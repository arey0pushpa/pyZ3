Total two pages

\begin{itemize}
\item \mukund{BIO motivation ( 1 paragraph ) }

\item \srivas{BIO to Graph problem ( a couple of paragraph) }

\item \srivas{Graph to previous encoding}

\end{itemize}

% \noindent


% What is vesicle traffic systems

vesicle traffic systems(VTS)

% Graph properties of the systems

% Search queries; why bio imp

% Using SAT/QBF solvers to encode


%

In this paper, we have developed a novel SAT encoding for
searching a VTS that satisfy the given properties.
%
We improved the encoding of some of the conditions for VTSs.
%
In the earlier implementation, reachability was encoded by
enumeration of paths.
%
In our encoding, we have recursively defined the reachability condition
such that that we avoid the exponential enumeration.
%
We have also encoded $k$-connectivity in the similar fashion and
avoided a blowup in the formula size.
%

We have implemented the encoding using Z3~\cite{z3} python API and searched for
VTS satisfying various properties upto size \ankit{15} nodes as compare
earlier tool that could only search graph upto size 8 nodes.
%
Since most VTSs in the cells are of \ankit{10} compartments,
our tool will be useful for biologists to study VTSs.
%
Furthermore, our encoding supports several variations of
the conditions.

The following are the contributions of this work:
\begin{itemize}
\item We have added novel encoding reachability and 3-4 connectivity
\item Direct encoding into the SAT solver
\item A user friendly and scalable tool based on well known SMT solver Z3
\end{itemize}

%--------------------- DO NOT ERASE BELOW THIS LINE --------------------------

%%% Local Variables:
%%% mode: latex
%%% TeX-master: "main"
%%% End:

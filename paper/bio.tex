%Again about two pages long
%
In this section, we will describe the model of VTSs and the properties
relevant to the biologists.

\textbf{The cell as a transport graph:} We consider a cell to be a
collection of compartments (nodes) and vesicles (edges), thus defining
a transport graph.
%
Compartments represent the large membrane-bound organelles, which are
connected by constant fluxes of small transport vesicles.
%
Every compartment or vesicle has a set of molecules associated with
it.
%
These include SNAREs and other biomolecules that regulate membrane
traffic, or that play the role of passive cargo.
%
Here we consider an abstract set of molecules and do
not explicitly associate them with particular protein varieties.

\textbf{Molecular flows and steady state:} The total amount of each
molecular type on each compartment can increase or decrease: this is
due to gain or loss of that molecular type driven by incoming or
outgoing vesicles.
%
Each edge is thus associated with a flux of all the molecular types
carried by the corresponding vesicle.
%
We assume the cell is in a steady state where each compartments
composition does not vary over time, meaning that all incoming and
outgoing fluxes are balanced for each molecular type at each
compartment.

\textbf{Vesicle targeting driven by molecular interactions:} The model
specification includes a description of how molecular properties
influence vesicle transport.
%
A vesicle can only contain a set of molecules that are present
on its source compartment.
%
Once a vesicle has budded out of the source, the molecules it carries
determine its properties.
%
In particular, for any given pair of a vesicle and a compartment, the
set of molecules that label the former and latter determine whether
the vesicle will fuse to that compartment.
%
Biophysically, fusion requires a direct physical interaction between
at least one molecular type on the vesicle and one molecular type on
the compartment.
%
The list of molecular pairs that can drive a fusion event is given an
a fusion pairing matrix.

\textbf{Molecular regulation:} The final layer of our framework
involves how the molecules are regulated.
%
We assume that for fusion to occur, the pair of molecular types
involved on the vesicle and compartment must both be in an
“active” state.
%
Whether these molecules are active or inactive depends on the
remaining molecules found on the vesicle or compartment, respectively.
%
This embodies the biological fact that other molecules can regulate the
fusion-driving molecules.
%
VTSs also require that if molecule is involved in fusion of some
vesicle and compartment, it must not be possible that the molecule 
can potentially fuse with some molecule in some other compartment.
%

We test many different versions of molecular
regulation.
%
Most generally, the activity state of a given molecule can be a
Boolean function of all the molecular types on a compartment or vesicles.
We have also tested a particularly simple regulation mechanism in which two
molecules that can pair to drive fusion inhibit one another.
%
This is
motivated by the idea that pairing must generate an inactive bi-molecular
complex.

\textbf{Synthesis:} We now combine all the above ingredients.
%
Given a particular
transport graph, a particular labeling of all the compartments and edges,
a particular fusion pairing matrix, and a particular regulatory model we do
the following.
\begin{enumerate}
\item We determine which molecules are ``active" on every
compartment or vesicle.
\item or every vesicle fusing to a compartment, we
determine whether there exists an active pair (one molecule on the vesicle,
one on the compartment) which drives that fusion event.
\item For every vesicle-compartment pair where the vesicle does not fuse to the
compartment, we verify that there is no pairing of active molecules on the
vesicle and compartment that could drive their fusion.
\item We verify that every molecular type entering a compartment also leaves the compartment,
and also that every molecular type entering a set of compartments also
leaves that set; this is the steady state condition.
%
This tells us that the
particular graph and molecular labeling does represent an allowed steady
state configuration of a VTS.
\end{enumerate}

Given the above conditions, the biologists search for VTSs that are
not $k$-connected, i.e., every pair of compartments remain reachable
after dropping $k$ vesicles.\ashu{why asking this question}
\ankit{Maximum connectivity [LGC, Least guarantee connectivity] and minimal connective [LRC (least required connectivity)] (rather than necc and suff cond) required by the graph.}



%--------------------- DO NOT ERASE BELOW THIS LINE --------------------------

%%% Local Variables:
%%% mode: latex
%%% TeX-master: "main"
%%% End:

Given a system that follows the VTS, we are interested in its connectivity (graph) property. We have implemented the encoding of the constraints in the python interface of $\zthree$. The basic idea is to constraint the variables such that vesicular transport rules are respected and using constraints (D0-D5) reason about the connectivity of the underlying structure (graph). We are interested in both least connectivity requirement (LRC) and least guarantee connectivity (LGC).
%LCR, that is required for a vesicular transport version to work and least connectivity that guarantees that vesicular transport system with that connectivity will always satisfies the rules.

% Whether a valid vesicular transport system is of certain connectiviity.   
%negation of the property .... STATE THE MEANING OF RESULT.  

%Besides using the default Z3 solvers we have used for these experiments \ankit{Z3, picoSAT, Lingeling}. The performance of Z3 was ...
%
% \Ankit{In other solver variations ...}
Our tool allows user to choose a model and the size of the network besides other parameters like connectivity and number of parallel edges. Our tool uses Z3 python interface to build the needed constraints and applies $\zthree$ on the constraints.

Our tool also translates the model found by $\zthree$ into a vesicle traffic system and presents a visual
output to the user. The output graph will satisfy the underlying rules of the system, so it is a valid vesicle transport network. The graph provides an overview of the underlying vesicle transport system with directed labeled edges and nodes providing the complete information about the system. Dashed lines are used in the resultant visual network output to specify the dropped edges to make the graph disconnected, which gives information about the connectivity of the graph.

\begin{table}[t]
  \centering
  \begin{tabular}[t]{|c|c|c|c|c|}\hline
    Size & Model & Connectivity &Formula building (in secs) & Solver (in sec) \\\hline
  \end{tabular}
  \caption{Runtimes for searching for models}
  \label{tab:qf-grabh}
\end{table}
  \todo{Make table for all sizes and variants of the tool}

%--------------------- DO NOT ERASE BELOW THIS LINE --------------------------

%%% Local Variables:
%%% mode: latex
%%% TeX-master: "main"
%%% End:

%\begin{table}[t]
  \centering
  \begin{tabular}[t]{|c|c|c|c|c|c|c|c|c|}\hline
    {\multirow{2}{*} \textbf{Size}}  & \multicolumn{2}{c|}{\textbf{Variant A}} & \multicolumn{2}{c|}{\textbf{Variant C}} & \multicolumn{2}{c|}{\textbf{Variant D}}  &  \multicolumn{2}{c|}{\textbf{Variant F}} \\\hline
   
   \cline{2-9}
    {} & {\textbf{Z3}} & {\textbf{CBMC}} & {\textbf{Z3}} & {\textbf{CBMC}} & {\textbf{Z3}} & {\textbf{CBMC}} & {\textbf{Z3}} & {\textbf{CBMC}} \\\hline
    
    2 & !0.085 & !2.43 & 0.15 & 2.12 & !0.13 & !1.89 & 0.35 & 5.12 \\\hline
    3 & !0.54 & !8.04 & 0.95  & 7.65 & 0.62 & 7.66  & 1.36 & 23.94\\\hline
    4 & !2.57 & !297.93 & 2.33 & 22.74 & 2.85 & 48.35  & 4.81 & 123.34\\\hline
    5 & !7.7 & !3053.8 & 7.60 & 500.03 & 10.27 & 890.84 & 33.36  & 2482.71 \\\hline
    6 & !22.98 & M/O & 19.52 & M/O & 30.81 & M/O  & 147.52 & M/O\\\hline
    7 & !57.07 & M/O & 81.89 & M/O & 82.94 & M/O & 522.26  & M/O \\\hline
    8 & !164.14 & M/O & 630.85 & M/O & 303.19 & M/O & 2142.76 & M/O\\\hline
    9 & !307.67 & M/O & 2203.45 & M/O & 971.01 & M/O & 4243.34 & M/O\\\hline
    10 & !558.34 & M/O & 7681.93 & M/O & 2274.30 & M/O & 7786.82 & M/O\\\hline
  \end{tabular}
  \caption{Run-times for searching for models (in secs).}
  \label{tab:qf-grabh}
\end{table}


%   \begin{tabular}[t]{|c|c|c|}\hline
%    \textbf{Size}  & {\textbf{Model B}}  &  {\textbf{Model E}} \\\hline
%    
%    2 & !0.09 & !0.12 \\\hline
%    3  & !0.54  & !0.75 \\\hline
%    4 &  !2.33 & !3.35 \\\hline
%    5 & !7.7 & !13.05 \\\hline
%    6 &  !20.05 & !40.64 \\\hline
%    7 & !51.86 & !152.78 \\\hline
%    8 & !112.24 &  !344.26 \\\hline
%    9 &  !236.31 & !880.73\\\hline
%    10 &  !531.56 &  !2133.89\\\hline
%  \end{tabular}
%  \caption{Run-times for searching for models (in secs).}
% % \label{tab:qf-grabh}
%\end{table} 

%\begin{table}[t]
%  \centering
%  \begin{tabular}[t]{|c|c|c|c|}\hline
%    \textbf{Size} &  \textbf{Model A}   & \textbf{Model B} &  \textbf{Model E} \\\hline
%   
%    2 & !0.0853381156921 & !0.091157913208 & !0.117037057877 \\\hline
%    3 & !0.540871858597 & !0.537098646164  & !0.754279375076 \\\hline
%    4 & !2.57536292076 & !2.33045578003 & !3.35159707069 \\\hline
%    5 & !7.70005106926 & !7.71436476707 & !13.0516757965 \\\hline
%    6 & !22.9898321629 & !20.0573630333 & !40.6696507931 \\\hline
%    7 & !57.0719909668 & !51.860738039 & !152.783704758 \\\hline
%    8 & !164.140100718 &  !112.248553038 &  !344.268831253 \\\hline
%    9 & !307.675467253 & !236.317871094 & !880.730427027\\\hline
%    10 & !558.342684269 &  !531.565055132 &  !2133.8986\\\hline
%  \end{tabular}
%  \caption{Runtimes for searching for models}
%  \label{tab:qf-grabh}
%\end{table}

%--------------------- DO NOT ERASE BELOW THIS LINE --------------------------

%%% Local Variables:
%%% mode: latex
%%% TeX-master: "main"
%%% End:

In Table 1. and Table 2.
%~\ref{tab:qf-graph},
we present the running times for the search of vesicle traffic systems of sizes 2 to 10 that satisfies the model variants. Formula for N = 10 returns in 45 minutes with a SAT result. In CBMC N = 10 results in OUT OF MEMORY. Hence with use of this novel encoding we are able to scale the system to much larger compartmentalized systems especially eukaryotics cells with ten compartments.  This additional cushion of scalability can provide us leverage to ask for more questions that were not previously possible for example what is the minimum number of molecules required to satisfy the vesicular traffic system in eukaryotes.

% REQUIRE COMPARISION WITH CBMC RUN TIME ?
The experiments were done on a machine with Intel(R) Core(TM) i3-4030U CPU @ 1.90GHz processor 
and 4GB RAM.
%
$\zthree$ was able to solve the constraints up to 18 nodes.
%

%table~\ref{tbl:qf-results} 

\begin{table}[!ht]
\centering
\def\arraystretch{1.6}
\caption{
{\bf Activity regulation of molecules and corresponding connectivity of the graph.}}
  \begin{tabular}{|c|c|c|c|}
    \hline
  {\textbf{Version}}  & {\textbf{Constraints}} &  {\textbf{Least graph connectivity}}  \\
    % \hline
    % \textbf{Inactive Modes} & \textbf{Description}\\
  %  \cline{3-4}
  %  {} & {} & \textbf{\textbf{Least}} & \textbf{Highest}\\
    %\hhline{~--}
    \hline
    
    A. & V1-V7 V8 V9 R1 R2 D1-D4 A\_nn A\_en & No graph  \\ \hline
B. & V1-V7 V8 V9 R1 R2 D1-D4 A\_nb A\_en & No graph  \\ \hline
C. & V1-V7 V8 V9 R1 R2 D1-D4 A\_nn A\_eb & 3-connected  \\  \hline
D. & V1-V7 V8 V9 R1 R2 D1-D4 A\_nn A\_ep & No graph  \\ \hline
E. & V1-V7 V8 V9 R1 R2 D1-D4 A\_nb A\_eb & 2-connected  \\ \hline
F. & V1-V7 V8 V9 R1 R2 D1-D4 A\_nb A\_ep & 4-connected  \\ \hline
G. & V1-V7 V8 V9 R1 R2 D1-D4 A\_nb A\_eb C\_ed & 3-connected \\ \hline

%A. & V1-V7 V8 V9 R1 R2 D1-D4 A\_nn A\_en & No graph & No graph \\ \hline
%B. & V1-V7 V8 V9 R1 R2 D1-D4 A\_nb A\_en & No graph & No graph \\ \hline
%C. & V1-V7 V8 V9 R1 R2 D1-D4 A\_nn A\_eb & 3-connected & 3-connected \\  \hline
%D. & V1-V7 V8 V9 R1 R2 D1-D4 A\_nn A\_ep & No graph & No graph \\ \hline
%E. & V1-V7 V8 V9 R1 R2 D1-D4 A\_nb A\_eb & 2-connected & 3-connected \\ \hline
%F. & V1-V7 V8 V9 R1 R2 D1-D4 A\_nb A\_ep & 4-connected & 4-connected \\ \hline
%G. & V1-V7 V8 V9 R1 R2 D1-D4 A\_nb A\_eb C\_ed & 3-connected & 3-connected \\ \hline
%H. & V1-V7 V8 V9 R1 R2 D1-D4 A\_nb  A\_eb + C\_es & No-idea & No-idea \\ \hline

% C_es :Self edges are allowed
% C_ed : Every edge is distinct 
  \end{tabular}
\label{table1}
\end{table}


%--------------------- DO NOT ERASE BELOW THIS LINE --------------------------

%%% Local Variables:
%%% mode: latex
%%% TeX-master: "main"
%%% End:

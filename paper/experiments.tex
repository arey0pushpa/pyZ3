We have implemented the encoding of the constraints
in the python interface of $\zthree$.
%
Our implementation supports several versions of the
models of vesicle traffic systems.
\todo{describe the variations}

\begin{enumerate}[label=\Alph*]
\item Every present molecule is considered to be active.
\item Activity of molecules on the nodes is based on Boolean function of presence of other molecules. 
\item Activity of molecules on the edges is based on Boolean function of presence of other molecules
\item Activity of molecules both on the edges and nodes is based on Boolean function of presence of other molecules.
\item Activity of molecules on the edges is driven by pairing inhibition.
\item Activity of molecules on the nodes is based on Boolean function of presence of other molecules and on edge by pairing inhibition.
\item Version F with additional constraint that every edge is distinct.
%\item Activity of molecules on the nodes is based on Boolean function of presence of other molecules and on edge by pairing inhibition with every edge is distinct.
%\item Activity of molecules on the nodes is based on Boolean function of presence of other moleculess and on edge by pairing inhibition with self-edges are allowed.
\item Version F but self-edges are allowed.
\end{enumerate}

We have applied the tool for the above variations
for searching the vesicle traffic systems that satisfies
their corresponding properties.


\begin{table}[t]
  \centering
  \begin{tabular}[t]{|c|c|c|c|c|}\hline
    Size & Model & Connectivity &Formula building (in secs) & Solver (in sec) \\\hline
  \end{tabular}
  \caption{Runtimes for searching for models}
  \label{tab:qf-grabh}
\end{table}
  \todo{Make table for all sizes and variants of the tool}

%--------------------- DO NOT ERASE BELOW THIS LINE --------------------------

%%% Local Variables:
%%% mode: latex
%%% TeX-master: "main"
%%% End:

In table~\ref{tbl:qf-results} 
\b PARTIAL ...

\begin{table}[!ht]
\centering
\def\arraystretch{1.6}
\caption{
{\bf Activity regulation of molecules and corresponding connectivity of the graph.}}
  \begin{tabular}{|c|l|l|c|c|}
    \hline
  \multirow{2}{*}{\textbf{Sr.No}}  & \multicolumn{2}{c|}{\textbf{Activity mechanism}} &  \multicolumn{2}{c|}{\textbf{Graph connectivity}}  \\
    % \hline
    % \textbf{Inactive Modes} & \textbf{Description}\\
    \cline{2-5}
    {} & \multirow{1}{*} {\bf{ On compartment}} & \multirow{1}{*} {\bf{On vesicle}}  & \textbf{\textbf{Least}} & \textbf{Highest}\\
    %\hhline{~--}
    \hline
E. & Boolean function & Boolean function & 2-connected & 3-connected \\ \hline
C. & None & Boolean function & 3-connected & 3-connected \\  \hline
F. & Boolean function & Pairing inhibition & 4-connected & 4-connected \\ \hline
D. & None & Pairing inhibition & No graph & No graph \\ \hline
B. & Boolean function & None & No graph & No graph \\ \hline
A. & None & None & No graph & No graph \\ \hline
G. & Boolean function & Every edge is distict & 3-connected & 3-connected \\ \hline
H. & Self edges are allowed & Boolean funcion & No-idea & No-idea \\ \hline

  \end{tabular}
\label{table1}
\end{table}

\todo{Make table for all sizes and variants of the tool}

%--------------------- DO NOT ERASE BELOW THIS LINE --------------------------

%%% Local Variables:
%%% mode: latex
%%% TeX-master: "main"
%%% End:

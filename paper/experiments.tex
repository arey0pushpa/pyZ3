We have implemented the encoding of the constraints
in the python interface of $\zthree$.
%
Our implementation supports several versions of the
models of vesicle traffic systems.
\todo{describe the variations}

\begin{enumerate}[label=\Alph*]
\item Every present molecule is considered to be active.
\item Activity of molecules on the nodes is based on Boolean function of presence of other molecules. 
\item Activity of molecules on the edges is based on Boolean function of presence of other molecules
\item Activity of molecules both on the edges and nodes is based on Boolean function of presence of other molecules.
\item Activity of molecules on the edges is driven by pairing inhibition.
\item Activity of molecules on the nodes is based on Boolean function of presence of other molecules and on edge by pairing inhibition.
\item Version F with additional constraint that every edge is distinct.
%\item Activity of molecules on the nodes is based on Boolean function of presence of other molecules and on edge by pairing inhibition with every edge is distinct.
%\item Activity of molecules on the nodes is based on Boolean function of presence of other moleculess and on edge by pairing inhibition with self-edges are allowed.
\item Version F but self-edges are allowed.
\end{enumerate}
%
Our tool allows user to choose a model and the size of
the network.
%
It builds the needed constraints and applies $\zthree$
on the constraints.
%
Our tool also translates the model found by $\zthree$
into a vesicle traffic system and presents a visual
output to the user.

We have applied the tool for the above variations
for searching the vesicle traffic systems that satisfies
their corresponding properties.


\begin{table}[t]
  \centering
  \begin{tabular}[t]{|c|c|c|c|c|}\hline
    Size & Model & Connectivity &Formula building (in secs) & Solver (in sec) \\\hline
  \end{tabular}
  \caption{Runtimes for searching for models}
  \label{tab:qf-grabh}
\end{table}
  \todo{Make table for all sizes and variants of the tool}

%--------------------- DO NOT ERASE BELOW THIS LINE --------------------------

%%% Local Variables:
%%% mode: latex
%%% TeX-master: "main"
%%% End:


In table~\ref{tab:qf-graph}, we present the running times
for the search of vesicle traffic systems of sizes 2 to 10
that satisfies the model variants.
%
The experiments were done on a machine with ... processor 
and ... RAM.
%

%
$\zthree$ was able to solve the constraints up to 15
nodes.
%

In table~\ref{tbl:qf-results} 



\begin{table}[!ht]
\centering
\def\arraystretch{1.6}
\caption{
{\bf Activity regulation of molecules and corresponding connectivity of the graph.}}
  \begin{tabular}{|c|c|c|c|c|}
    \hline
  \multirow{2}{*}{\textbf{Sr.No}}  & \multicolumn{2}{c|}{\textbf{Activity constraint}} &  \multicolumn{2}{c|}{\textbf{Graph connectivity}}  \\
    % \hline
    % \textbf{Inactive Modes} & \textbf{Description}\\
    \cline{2-5}
    {} & \multirow{1}{*} {\bf{ On compartment}} & \multirow{1}{*} {\bf{On vesicle}}  & \textbf{\textbf{Least}} & \textbf{Highest}\\
    %\hhline{~--}
    \hline
E. & A\_nb & A\_eb & 2-connected & 3-connected \\ \hline
C. & A\_nn & A\_eb & 3-connected & 3-connected \\  \hline
F. & A\_nb & A\_ep & 4-connected & 4-connected \\ \hline
D. & A\_nn & A\_ep & No graph & No graph \\ \hline
B. & A\_nb & A\_en & No graph & No graph \\ \hline
A. & A\_nn & A\_en & No graph & No graph \\ \hline
G. & A\_nb & A\_eb + C\_ed & 3-connected & 3-connected \\ \hline
H. & A\_nb & A\_eb + C\_es & No-idea & No-idea \\ \hline

% C_es :Self edges are allowed
% C_ed : Every edge is distinct 
  \end{tabular}
\label{table1}
\end{table}


%--------------------- DO NOT ERASE BELOW THIS LINE --------------------------

%%% Local Variables:
%%% mode: latex
%%% TeX-master: "main"
%%% End:

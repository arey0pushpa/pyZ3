%\ankit{related work: other SAT works related to bio; Graph problems; Models of VTS/GRN}
Modern day SAT solvers can handle millions of variables and due to the exhaustive nature of the searching, solving the combinatorial problem is a natural fit for the SAT solvers. There are many graphs related combinatorial work \cite{gay2013solving,wotzlaw2012generalized} that have used SAT solvers. Thanks to the improvement in SAT solver and formal techniques, tools like model checkers and theorem provers scalability and combinatorial solving capability are now being used to model complex biological systems \cite{heule2010exact,yordanov2013smt,mangla2010timing}.
% Many network questions are being reduced to solving a SAT question.
Many of these biological system uses model checkers to reason about graphs and networks rules with possible exhaustive search \cite{guerra2012reasoning,chin2008biographe}.  

Recently model checkers found the use of modeling and understanding the gene regulatory networks (GNR) \cite{giacobbe2015model,rosenblueth2014inference, batt2010efficient}. GNR is a complex system driven by many complex rules. We extended this work and have applied model checkers to the more complex transport network (VTS) \cite{mani2016stacking}. Model checkers are far scalable in comparison to simulations, but scaling it to cases which are interesting biologically is still a challenge. In this paper, we have achieved the scalability required for analyzing vesicular transport network for eukaryotic cells, of total ten compartments (N = 10). \\

%--------------------- DO NOT ERASE BELOW THIS LINE --------------------------

%%% Local Variables:
%%% mode: latex
%%% TeX-master: "main"
%%% End:

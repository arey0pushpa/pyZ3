%Total two pages

%\begin{itemize}
%\item \mukund{BIO motivation}

The conservation of basic molecular processes across all living cells,
including the genetic code, suggests that life on Earth arose only
once: all extant organisms are presumed to trace their lineages back
to a last universal common ancestor (LUCA).
%
However, there are still vast differences in cellular and molecular
architecture among the three kingdoms of life: bacteria, archaea, and
eukaryotes~\cite{embley2006eukaryotic}.
%
Eukaryotic cells have larger genomes and have more complex cellular
architectures than the prokaryotic bacteria and archaea.
%
In particular, the cytoplasm of all eukaryotic cells is broken up into
distinct membrane-bound locations known as organelles: the nuclear
membrane, endoplasmic reticulum (ER), Golgi apparatus, vacuoles and
lysosomes are examples of these.
%
Cargo is moved between these organelles in small membrane-bound
transporters known as vesicles \cite{stenmark2009rab}.

This organization resembles a transport logistics network, and is
commonly known as the membrane traffic system.
%
Membrane traffic underpins nearly every aspect of eukaryotic cellular
physiology, including human cellular physiology.
%
Defects in membrane traffic lead to a variety of disorders or even
cell death \cite{stenmark2009rab}.
%
Understanding how membrane traffic functions is therefore one of the
central challenges of cell biology.
%
Some progress has been made, as cell biologists have assembled a
“parts list” of molecules that drive vesicle
traffic~\cite{dacks2007evolution}.
%
The essential processes involve the creation of vesicles loaded with
cargo from source organelles, known as “budding”, and the depositing
of those vesicles and cargo into target organelles, known as “fusion”
\cite{munro2004organelle}.
%
Budding is regulated by proteins known as coats and adaptors that
select cargo.
%
Fusion is regulated by proteins known as SNAREs and tethers, that
ensure that vesicles fuse to the correct target
\cite{mani2016stacking}.
%
Although we know a great deal about molecular-level details, there has
been very little work done on how such processes are integrated across
scales to build the entire vesicle traffic system (VTS).
%
We have recently attempted to use ideas of graph theory to address
this question \cite{mani2016stacking,shukla}.

The SNARE proteins and the constraints that SNARE-to-SNARE regulations
impose on feasibile behaviors of VTSs are the focus of
our current work.
%
Abstractly, we can think of these proteins as labels that are
collected from the source compartment and taken to the target
compartment on vesicles.
%
There is a corresponding set of label molecules on the target compartments.
%
If the two sets of labels (on vesicles and targets) are compatible,
then the vesicle will fuse.
%
The question is: does this physical picture place any constrains on
the global topology of the traffic graph? We have previously shown
that one informative constraint is graph connectivity \cite{shukla}.
%
The intricacies of SNARE regulation are largely unknown.
%
How stringent SNARE regulation is can determine whether graph
topologies of particular level of connectivity are biologically
realizable.
%
For example, if SNAREs are completely unregulated and thus always
active, they cannot be used to generate a traffic network.
%
If these SNAREs are regulated by relatively simple rules then only
highly connected traffic networks can be generated using them.
%
The question is can we work toward a predictive analytical framework
of VTSs?

The analysis of VTSs is a difficult problem because
of the combinatorial scaling of possible traffic topologies and
regulatory rules.
%
Previous analyses of VTSs dealt with this combinatorial explosion by
using a sampling approach: simulating randomly generated regulatory
rules and taking averages over many samples~\cite{mani2016stacking}.
%
The outcome of such simulations is necessarily statistical rather than
rigorous and precise.
%
To make definite predictions about VTS topologies, we need to
efficiently check useful system properties over all possible graphs
and all possible variations of rules, preferably without enumerating
them.
%
Formal verification tools, such as model checkers
\cite{clarke1996symbolic, biere2003bounded, clarke2008birth,
cimatti2000nusmv, holzmann1997model} and SAT solvers
\cite{moskewicz2001chaff,een2004extensible} serve precisely this
purpose.
%
These tools encode the problem symbolically, and check various
properties without explicit enumeration of the graphs up to large but
finite size of the models.

%These tools are central to formal verification methods in computer science, and are increasingly being applied to understand biological systems, including gene regulatory networks [16-18]. 

In a recent work, we employed the model checker CBMC \cite{CKY03,
ckl2004} meant to analyze ANSI C programs for studying this problem.
%
In this work, VTSs and SNARE regulations were modeled as C code mostly
using arrays to represent graphs and boolean tables.
%
Queries about graph connectivity requirements for different variations
of SNARE Regulations were modeled as logical assertions to be checked
by the model checker.
%
CBMC uses its own built-in encodings to reduce the analysis problem to
boolean satisfiability problem which is solved using SAT solvers.
%
This work was successful in demonstrating the feasibility of this
technology to this domain.
%
However, the scalability of this approach was limited (up to VTSs
with 7-8 nodes) due to the following reasons:
\begin{enumerate}
\item The encoding used to model VTSs was not completely fine-tuned
to the structure of the problem.
\item The queries to be solved involve quantified formulas, i.e., we
needed to solve queries of the form ``\emph{for all} graphs does there
\emph{exist} SNARE regulation constraints that satisfy biological
steady state.
\end{enumerate} As CBMC does not directly support solving such
quantified formulas, we had to use a technique that combined
non-determinism and partial enumeration that is not as
efficient.\newline

%
In this paper, we have developed a novel SAT encoding for
searching a VTS that satisfy the given properties.
%
We improved the encoding of some of the conditions for VTSs.
%
In the earlier implementation, reachability was encoded by
enumeration of paths.
%
In our encoding, we have recursively defined the reachability condition
such that that we avoid the exponential enumeration.
%
We have also encoded $k$-connectivity in the similar fashion and
avoided a blowup in the formula size.
%

We have implemented the encoding using Z3~\cite{z3} python API and searched for
VTS satisfying various properties upto size 14-18 nodes as compare
earlier tool that could only search graph upto size 7-8 nodes.
%
Since most VTSs in the cells are of 10 compartments,
our tool will be useful for biologists to study VTSs.
%
Furthermore, our encoding supports several variations of
the conditions.\newline

The following are the contributions of this work:
\begin{itemize}
\item We have added novel encoding reachability and 3-4 connectivity
\item Direct encoding into the SAT solver
\item A user friendly and scalable tool based on well known SMT solver Z3
\end{itemize}

%--------------------- DO NOT ERASE BELOW THIS LINE --------------------------

%%% Local Variables:
%%% mode: latex
%%% TeX-master: "main"
%%% End:

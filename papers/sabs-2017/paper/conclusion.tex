In the experiments, we have demonstrated the power of SMT solvers and the value
of careful encoding of the problems into SMT queries.
%
We manage to scale the tool upto the size that makes tool
biologically relevant.
%
However, there are many further search problems or extensions
that are of interest.
%
For example, are there any k-connected graphs that can not be
annotated into a VTS?
%
This problem induces a quantifier alternation in an encoding.
%
Therefore, a simple call to SAT solver will not work.
%
We are planning to use QBF(quantified Boolean formulas) solvers or develop
iterative search algorithm for such queries.
%

One may be interested in counting the number of graphs that satisfy
a given property.
%
Exact counting of the graphs using SAT solvers may prove to be very difficult.
%
We are also planning to employ some methods for approximate counting of solutions.


% \ankit{interpret experiments (3-4 lines)}

 % \srivas{Future work(quantifier problem etc)}


%--------------------- DO NOT ERASE BELOW THIS LINE --------------------------

%%% Local Variables:
%%% mode: latex
%%% TeX-master: "main"
%%% End:

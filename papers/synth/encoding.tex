%
In this section, we will present a list of synthesis problems that may
arise from the partial available information about a VTS and our synthesis method
for the problems.

\subsection{Problem Statements}

We will assume that we are given a VTS, whose all components
are not specified.
%
Our objective is to find the missing parts.
%
The missing parts can be in any of the components of VTS. 
%
For example, some undiscovered edges or nodes, or insufficient
knowledge about the presence of molecules in some part of the VTS.
%
To cover most of the likely variations of this missing information,
we have encoded the following variants of VTS synthesis problem.

\begin{enumerate}
\item Fixing VTS by adding edges 
\item Fixing VTS by adding molecules to the labels
\item Fixing VTS by learning activity functions
  % \begin{enumerate}
  %   \item kcnf: low depth circuit.
  %   \item Boolean gates: And, Or.
  %   \item Boolean gates with linear combination.  
  % \end{enumerate}        
  % - Function dependence with var occurring once.
\item  Fixing VTS by both adding/deleting parts
\end{enumerate}


\subsection{Encoding Incomplete VTS}

In our synthesis method, we take a VTS $G =
(\nodes,\mols,\edges,\nlabel,\pairs,\edgef,\nodef)$ as input.
%
We allow activity functions not to be specified.
%
We construct the following constraints to encode the available information
about $G$.
%
\begin{align*}
  \texttt{PresentE} &= \land \{e_{i,j,q,m}|(i,M_1,j),...,(i,M_{q'},j) \in \edges \land q \leq q' \land m \in M_q \}\\
  \texttt{PresentN} &= \land \{n_{i,m}| m \in \nlabel(i) \land i \in \nodes \}\\
  \texttt{PresentP} &= \land \{p_{\{m_1,m_2,m_3,m_4\}}| \{m_1,m_2,m_3,m_4\} \in \pairs \} \\
  \texttt{KnownActiveN} &= \land \{ a_{i,m} = \texttt{NodeFun}_m (n_{i,1},\dots,n_{i,\mu}) | \nodef_m \text{ is defined.} \} \\
  \texttt{KnownActiveE} &= \land \{ b_{i,j,q,m} = \texttt{EdgeFun}_m(e_{i,j,q,1}, .., e_{i,j,q,\mu})
                   | \edgef_m \text{ is defined.} \} \\
  \texttt{PresentCons} & = \texttt{PresentE} \land \texttt{PresentN} \land 
  \texttt{PresentP} \land \texttt{KnownActiveE} \land \\
  &  \texttt{KnownActiveN}
\end{align*}
We also collect the variables that are not set to true in $\texttt{PresentCons}$.
\begin{align*}
  \texttt{AbsentELabel} &=
  \{e_{i,j,q,m}|(i,M_1,j),...,(i,M_{q'},j) \in \edges \land 
                          0 < q \leq q' \land m \not\in M_{q} \}\\
  \texttt{AbsentE} &= \{e_{i,j,q}|(i,M_1,j),...,(i,M_{q'},j) \in \edges \land 
                    q' < q \leq \pi \}\\
  \texttt{AbsentNLabel} &= \{n_{i,m}| m \not\in \nlabel(i) \land i \in \nodes \}\\
  \texttt{AbsentP} &= \{p_{\{m_1,m_2,m_3,m_4\}}| {\{m_1,m_2,m_3,m_4\}} \not\in \pairs \} \\
  \texttt{UnnownActive} &=  \Land \{ a_{i,m} = \nodef_m (n_{i,1},\dots,n_{i,\mu}) | \nodef_m \text{ is undefined.} \} \union \\
   &\quad\quad\quad  \{ b_{i,j,q,m} = \edgef_m(e_{i,j,q,1}, .., e_{i,j,q,\mu})
                   | \edgef_m \text{ is undefined.} \}
\end{align*}
We have defined $\texttt{AbsentELabel}$, $\texttt{AbsentE}$, $\texttt{AbsentN}$, and
$\texttt{AbsentP}$
as sets.
%
They will be converted into formulas depending
on the different usage in the synthesis problems. 

% Since not all the activity functions are defined some of the functions
% appearing in $\texttt{ActiveE}$ and $\texttt{ActiveN}$ are
% uninterpreted function symbols.

\subsection{Encoding synthesis property}

We will do synthesis against the following property that says the VTS
is stable and 3-connected.
%
% The property is designed to balance the search space such that the synthesis procedure does not
% succeed with simply adding too many edges. 
%
\begin{align*}
  \texttt{Property} =  \texttt{Stability} \land \texttt{Connected}(3) 
  %\land \texttt{DisConnected}(4)
\end{align*}
The property was proposed in~\cite{shukla2017discovering}.
%
However, the biological relevance of the property is debatable and open for change.
%
Our tool is easily modifiable to support any other property that may be deemed 
interesting by the biologists.

\subsection{Encoding synthesis constraints}

Now we will consider the encodings for the
listed synthesis problems.
%
The presented variations represent the encodings
supported by our tool.
%
Additionally, the combinations of the variation are also possible and
our tool easily supports them.
%
For simplicity of the presentation, we assume that if we are
synthesizing an aspect of VTS, then all other aspects are fully given.
%
Therefore, we will describe two kinds of constraints for synthesis
problems.
%
One will encode the variable part in the synthesis problem and
the other encodes the fixed parts.
%
Subsequently, the two constraints will be put together with 
$\texttt{Consistancy}$ and 
$\texttt{Property}$ to construct the constraints for synthesis.

\subsubsection{Fixing VTS by adding edges}
%
Now we will consider the case when we add new edges to VTS to satisfy the properties.
%
In the following, the pseudo-Boolean formula $\texttt{AddE}$ encodes
that at most $slimit$ new undeclared edges may be added in the VTS.
%
\texttt{FixedForEdge} encodes the parts of the VTS that are not allowed to change.
\begin{align*}
  &\texttt{AddE}(slimit) = \sum~\texttt{AbsentE} \leq slimit\\
  &\texttt{FixedForEdge} = \texttt{PresentCons} \land \texttt{UnknownActive} \land\\
   & \quad\quad\quad\quad \lnot \Lor \texttt{AbsentELabel} \;\union\;
                    \texttt{AbsentNLabel} \;\union\;
                    \texttt{AbsentP}
\end{align*}
We put together the constraints and obtain the following formula.
\begin{align*}
  & \texttt{SynthE}(slimit) =
       \texttt{Consistancy}\land \texttt{Property} \land
   \texttt{FixedForEdge} \land
  \texttt{AddE}(slimit)
  % \tag{SynthE}\label{eq:addedge1}
\end{align*}
Similar to what we have seen $\texttt{Consistancy}$ encodes the basic constraints about VTS,
$\texttt{Property}$\; encodes the goal, and
the rest two are defined just above.
%
A satisfying model of $\texttt{SynthE}$ will make 
some of the edges in $\texttt{AbsentE}$ true such that~$\texttt{Property}$ is satisfied.
%
We limit the addition of the edges, since we look for a fix that require minimum number
of changes in the given VTS.
%
We start with $slimit = 1$ and grow one by one until $\texttt{SynthE}(slimit)$
becomes satisfiable.

In the later synthesis problems, we will construct a similar QBF
formula with same first two parts and the last two are due the
requirements of the synthesis problem.

\paragraph{\bf Fixing VTS by adding molecules to the labels:}
The system may also be fixed only by modifying labels on the edges or the nodes instead
of adding edges.
%
Here let us consider only adding molecules to the labels of edges.
%
In the following, the formula encodes that only $slimit$ edge labels may be added.
\begin{align*}
  &\texttt{AddLabelEdge}(slimit) = 
    \sum~\texttt{UnknwonEdgeLabel}  \leq slimit\\
  & \texttt{FixedForLabel} = \texttt{PresentCons} \land \texttt{UnknownActive} \land\\
  & \quad \quad \lnot \Lor \texttt{AbsentE} \;\union\;
                    \texttt{AbsentNLabel} \;\union\;
                    \texttt{AbsentP}\\
  &\texttt{SynthLabel}(slimit) = \\
  & \quad\quad
    (  \texttt{Consistancy}\land \texttt{Property} \land
   \texttt{FixedForLabel} \land \texttt{AddLabelEdge}(slimit) )
\end{align*} 
Similar to the previous encoding, we solve the
satisfiability of the above formula to obtain additional molecules
that may be added to the edge labels to satisfy the properties.



\subsubsection{Fixing VTS by learning activity functions:}

Now we consider a scenario where some of the activity functions for some of the molecules
are missing.
%
The activity functions are $\mu$-input Boolean functions.
%
First, we choose a class of formulas for the candidate functions.
%
We encode the candidates in a formula with parameters.
%
By assigning different values for the parameters, a solver may select
different candidates for the activity functions.
%
We will illustrate only one class of formulas.
%
However, we support other classes of formulas for example $k$-CNF.

In the following, the formula \texttt{NNFTemplate} encodes a set of
negation normal form functions that take $y_1,..,y_\mu$ as input and
contain $\lambda$ literals.
%
We use $\texttt{Gate}$ to encode a gate that takes a parameter integer
$x$ to encode various gates.
%
We use $\texttt{Leaf} $ to encode the literal at some position.
%
Both are stitched to define \texttt{NNFTemplate}.
%
To encode the set of NNF formulas with $\lambda$ literals, it has
finite-range integer variables
$z_1,..,z_{2\lambda}$ as parameters.

% \paragraph{Encoding low depth cnf circuits}
% \paragraph{Encoding boolean gates}

\begin{align*}
  &\texttt{Gate}( x, w_1, w_2 ) = ( x  = 1 \limplies w_1  \land w_2 ) \land  
  ( x = 2  \limplies w_1  \lor w_2 ) \\
  &\texttt{Leaf}( x, [y_1,..,y_\mu] ) =
  \Land\limits_{l=1}^{\mu} ( x = 2l-1  \limplies y_{l}) \land ( x = 2l  \limplies \lnot y_{l})\\
  &\texttt{NNFTemplate}([z_1,..,z_{2\lambda}],[y_1,..,y_\mu] ) = \\
  &\exists w_1,..,w_{2\lambda}.\;w_1 \land \Land_{l=1}^{\lambda} w_{\lambda+l} = \texttt{Leaf}( z_{\lambda+l}, [y_1,..,y_\mu] ) \land
  w_{l} = \texttt{Gate}( z_l, w_{2l},w_{2l+1})
\end{align*}

Using the template we define the constraints $\texttt{FindFunctions}(z,\lambda)$
that encodes the candidate functions that satisfy the activity requirements,
where $z$ is the vectors of parameters for encoding parameters
for each molecule,
and $\lambda$ limits the size of the candidate functions. 
%
We fix the all other aspect of the VTS to be fixed via constraints
$\texttt{FixedForFunctions}$.
%
\begin{align*}
  &\texttt{FindFunctions}(z,\lambda) =\\
  &\Land \{ \bigwedge\limits_{i} n_{i,m} \limplies a_{i,m} = 
  \texttt{NNFTemplate}([z_{m,1},..,z_{m,2\lambda}],[n_{i,1},\dots,n_{i,\mu}] ) \\
  & \hspace{8cm}|
  \nodef_m \text{ is undefined}\} \\
  &\Land \{ \bigwedge\limits_{i,j,q} e_{i,j,q,m} \limplies b_{i,j,q,m} = 
  \texttt{NNFTemplate}([z_{i,j,q,1},..,z_{i,j,q,2\lambda}],[e_{i,j,q,1},..,e_{i,j,q,\mu}] ) \\
  & \hspace{8cm} | \edgef_m \text{ is undefined}
  \}\\
  & \texttt{FixedForFunctions} = \texttt{PresentCons} \land\\
  & \quad \quad
\lnot \Lor \texttt{AbsentE} \;\union\; \texttt{AbsentELabel} \;\union\;
                    \texttt{AbsentNLabel} \;\union\;
                    \texttt{AbsentP}\\
  &\texttt{SynthFunction}(z,\lambda) = \\
  & \quad
    (  \texttt{Consistancy}\land \texttt{Property} \land
   \texttt{FixedForFunctions} \land \texttt{FindFunctions}(slimit) )
\end{align*}
We construct $\texttt{SynthFunction}(z,\lambda)$ similar to the earlier
variations.
%
By reading of the values of $z$ in a satisfying model of the formula,
we learn the synthesized function.

\subsection{Fixing VTS by both adding/deleting parts:}
%
Now we will consider repairing of VTS by allowing not only addition but also
deletion of the molecules, edges, functions, or pairing matrix.
%
We have encoded the repairing in our tool by introducing flip bits
for each variable that is modifiable in the VTS.
%
We illustrate the repairing on one class of variables and rest can be
easily extended.
%
Let us consider repairing of node labels.
%
For each bit $n_{i,m}$, we create a bit $flip_{i,m}$.
%
We add constraints that take xor of VTS assigned values for  $n_{i,m}$
and $flip_{i,m}$.
%
We also limit the number of $flip_{i,m}$ that can be true, therefore
limiting the number of flips.
%
The above constraints are encoded in $\texttt{FlipN}(slimit)$.
\begin{align*}
    &\texttt{FlipN}(slimit,flip) = \Land \{n_{i,m}\lxor flip_{i,m}| m \in \nlabel(i) \land i \in \nodes \} \land \\
  & \quad \quad \quad
    \Land \{ \lnot n_{i,m}\lxor flip_{i,m}| m \not \in \nlabel(i) \land i \in \nodes \} \land  \sum\limits_{i,m} flip_{i,m} \leq slimit
\end{align*}
%
Similar to the earlier variations, we construct
$\texttt{SynthRepairNode}(slimit)$ for the repair.
%
In that, $\texttt{FixedForNodeRepair}$ encodes all the parts of VTS that do not change.
\begin{align*}
% \\ 
    & \texttt{FixedForNodeRepair} =  \texttt{PresentE} \land  
      \texttt{PresentP} \land \texttt{KnownActiveE} 
      \land \\
  & \quad \texttt{UnknownActive} \land \texttt{KnownActiveN} \land \lnot 
    \Lor \texttt{AbsentE} \;\union\; \texttt{AbsentNLabel} \;\union\;
    \texttt{AbsentP}\\
  &\texttt{SynthRepairNode}(slimit,flip) = \\
  & \quad\quad
    (  \texttt{Consistancy}\land \texttt{Property} \land
    \texttt{FixedForNodeRepair} \land \texttt{FlipN}(slimit,flip) )
\end{align*}
A satisfying model of $\texttt{SynthRepairNode}(slimit,flip)$ will assign some
$flip$ bits to true.
We will learn from the assignments the needed modifications in the VTS. 

%%% Local Variables:
%%% mode: latex
%%% TeX-master: "main"
%%% End:

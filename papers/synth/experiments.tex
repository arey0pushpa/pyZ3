
\begin{table}[t]
  \centering
  \begin{tabular}[t]{|c|c|c|c|c|c|c|c|c|c|c}\hline
    {\multirow{2}{*} \textbf{Size}}  & \multicolumn{2}{c|}{\textbf{Add edge}} & \multicolumn{2}{c|}{\textbf{Add molecules}} & \multicolumn{2}{c|}{\textbf{Low depth Cnf}}  &  \multicolumn{2}{c|}{\textbf{Gate function}} &  \multicolumn{2}{c|}{\textbf{VTS repair}} \\\hline
   
   \cline{2-9}
    {} & {\textbf{Free}} & {\textbf{Min tweak}} & {\textbf{Free}} & {\textbf{Min tweak}} & {\textbf{Free}} & {\textbf{Min tweak}} & {\textbf{Free}} & {\textbf{Min tweak}} & {\textbf{Free}} & {\textbf{Min tweak}} \\\hline
    
    plos1-dia[3/4C] & UNSAT & UNSAT & UNSAT & UNSAT & UNSAT & UNSAT & UNSAT & UNSAT & 0.0492 & 0.0\\\hline
    plos2-dia [4C] & 2.206 & \#0.0414 & 0.0424 & \#0.0407 & 0.190 & 1.333[1 n] & 2.192 & 2.327[1n 9e] & 0.0499 & \#0.042 \\\hline
%[M=8, N=2, Q=2]    
% n/A n/A activeN == UNSAT adding edge == UNSAT activeE == UNSAT presenceE == UNSAT
    mukund[3/4C] & 0.3809 & 185.06 & 0.3773 & 0.0 & 2.062 & 0.0 & !13.92/1.553 & 0.0 & 0.230 & 0.0 \\\hline
    node4[-] & 16.70 & 0.0 & 16.472 & 0.0 & UNSAT & UNSAT & UNSAT & UNSAT & 2.194 & 0.0\\\hline
    yeast-graph & 0.0 & 0.0 & 0.0 & 0.00 & 0.0 & 0.0 & 0.0  & 0.0 & 0.0 & 0.0 \\\hline
    mukund-comp & 0.0 & 0.0 & 0.0 & 0.0 & 0.0 & 0.0 & 0.0 & 0.0 & 0.0 & 0.0\\\hline
  \end{tabular}
  \caption{Run-times for searching for models (in secs).}
  \label{tab:qf-grabh}
\end{table}


We implemented the encodings in a tool called~\ourtool\footnote{{\url{https://github.com/arey0pushpa/pyZ3}}}.
%
The tool takes a partially defined VTS as input in its custom designed
input language. The input is then converted to the constraints over VTS. 
%
The tool can not only synthesize the above discussed query, but also their
combination.
%
For example, our tool can modify labels of nodes or edges while
learning activity functions.
%
Our tool is developed in C++ and uses~\zthree~\cite{z3} infrastructure for
processing formulas. 
%
Since, the formulas involve alternation of quantifiers over Boolean variables Z3 is not a suitable choice. We translate the formula created by Z3 tool into standard QDIMACS~\cite{qdimacs} format. Which can be used as an input for most of the QBF solvers. 

%Since the formulas are in QBF, we cannot use~\zthree.
%
We use~\depqbf~\cite{lonsing2010depqbf} for solving of QBF formulas. 
%
Our tool includes about 7000 lines of code.

We have applied~\ourtool~on six partially defined VTS.
%
The results are presented in table~\ref{tab:qf-graph}.
%
The experiments were conducted on ... machine,
with \ashu{?}GB memory and 900s timeout.
%
The first four VTSs are synthetic and inspire from literature for typical
motifs in VTSs. 
%
%
The fifth VTS is taken from "A complete set of SNAREs in yeast"
~\cite{burri2004complete} paper. It has been adapted from the paper by separating the v and the t SNAREs. 
%The t SNAREs are labelled within compartments and v SNAREs on the edges. %\ashu{Describe the source of two examples}.
%
The last VTS represent mammalian SNARE map created by studying the literature references.  

The table shows timing for various synthesis queries.
%
For each synthesis query, we have two columns.
%
One column reports the timing and the other reports the minimum changes
needed to obtain a valid VTS.
%
$\infty$ indicates that any number of changes with the synthesis query
search space can obtain the VTS.
%
\ashu{@ankit: please discuss all the synthesis queries in the table.}
%
%
Our experiments suggest that the synthesis problems are solvable by modern
solvers and the synthesis technology may be useful for biological research.
%


%%% Local Variables:
%%% mode: latex
%%% TeX-master: "main"
%%% End:

In this section, we will present the model of VTS from~\cite{PloS,SASB}.
%
We will also present the constrains and properties on the VTSs.   
%
We model a VTS as labelled graph along with assisting pairing matrices and
activating functions.

\begin{df}
  A VTS $G$ is a tuple $(\nodes,\mols,\edges,\nlabel,\pairs,\edgef,\nodef)$, where
  \begin{itemize}
  \item $\nodes$ is a set of nodes representing compartments in the VTS,
  \item $\mols$ is the set of molecules flowing in the system, 
  \item $\edges \subseteq \nodes \times (\powerset{\mols}-\emptyset) \times \nodes$ is the
    set of edges with molecule sets as labels,
  \item $\nlabel : \nodes \maps \powerset{\mols}$ defines the molecules present in the nodes,
  \item $\pairs \subseteq \mols \times \mols$ is pairing relation,
  \item $\nodef : \mols \maps \powerset{\mols} \maps \booleans$ is activity maps for nodes, and
  \item $\edgef : \mols \maps \powerset{\mols} \maps \booleans $ is activity maps for edges.
  \end{itemize}
\end{df}
$\pairs$ defines which molecules can fuse with which molecules.
%
Let $\pairs(M')$ denote $\{m|(m,m') \in P \text{ and } m' \in M'\}$.
%
$\nodef$ and $\edgef$ are used to define activity of molecules on
nodes and edges respectively.
%
A molecule $k$ is {\em active} at node $n$ if $k \in \nlabel(n)$ and
$\nodef(k,\nlabel(n))$ is true.
%
A molecule $k$ is {\em active} at edges $(n,M',n')$ if $k \in M'$ and
$\edgef(k,M')$ is true.
%
We call $G$ {\em well-structured} if molecules $M$ is divided into
two equal-sized partitions $M_1$ and $M_2$ such that
$P \subseteq M_1 \times M_2$, and
for each $(n,M',n') \in \edges$, $n \neq n'$, 
$M' \subseteq \nlabel(n) \intersection \nlabel(n')$.
%
In other words, a well-structured VTS has no self loops, and 
each edge carry only those molecules that are present in its source
and destination nodes. 

% We will also consider several variations of the model.
% %
% For example, unique edge between two nodes, activity of molecules is
% not constrained by $\nodef$ and $\edgef$, etc.
%

A {\em path} in $G$ is a sequence $n_1,...,n_\ell$ of nodes 
such that $(n_i,\_,n_{i+1}) \in \edges$ for each $ 0 < i < \ell$.
%
For a molecule $m \in M$,
an {\em $m$-path} in $G$ is a sequence $n_1,...,n_\ell$ of nodes 
such that $(n_i,M',n_{i+1}) \in \edges$ and $m \in M'$ for
each $ 0 < i < \ell$.
%
A node $n'$ is {\em ($m$-)reachable} from node $n$ in $G$ if there is a ($m$-)path
$n,...,n'$ in $G$.
%
% A node $n'$ is {\em $m$-reachable} from node $n$ in $G$ if there is a
% $m$-path $n,...,n'$ in $G$.
%
We call $G$ {\em stable} if for each $(n,M',n') \in \edges$ and $m \in M'$,
$n$ is $m$-reachable from $n'$.
%
An edge $(n,M',n') \in \edges$ {\em fuses} with its destination node $n'$
if there are molecules $m,m' \in \mols$ such that $m$ is active in
$(n,M',n')$, $m'$ is active in $n'$, and $(m,m') \in \pairs$.
%
We call $G$ {\em well-fused} if each edge $(n,M',n') \in \edges$ fuses
with non-empty fusing molecules $M'' \subseteq M'$
and $\pairs(M'')$ are not active in any other node.
%
We call $G$ {\em connected} if for each $n,n' \in \nodes$,
$n'$ is reachable from $n$ in $G$.
%
We call $G$ $k$-connected if for each $\edges' \subseteq \edges$ and $|\edges'| < k$,
VTS $(\nodes,\mols,\edges-E',\nlabel,\pairs,\edgef,\nodef)$ is connected.

%
% In the definition, we do not care about the paths to be $m$-connected for some $m$.  
% %
% A variant of the definition may be sensitive to the $m$-connectedness, but
% we are not considering the variation.


\subsection{Encoding VTS}

The conditions on the VTSs for a given size can be encoded as a QBF formula
with uninterpreted functions.
%
To encode the constraints, we need needs variables for each aspect of
VTS.
%
Let us suppose that the size of the graph is $\nu$ and number of
molecules is $\mu$.
%
Furthermore, we also limit the maximum number $\pi$ of edges present
between two nodes.
%
Here, we list the Boolean variables and uninterpreted function symbols
that encode the VTSs.
\begin{enumerate}

\item Boolean variable $n_{i,m}$ indicates if $m \in \nlabel(i)$
\item Boolean variable $e_{i,j,q}$ indicates if $q$th edge exists between $i$ and $j$.
\item Boolean variable $e_{i,j,q,m}$ indicates if $q$th edge between $i$ and $j$ contains $m$.
\item Boolean variable $p_{m,m'}$ indicates if $(m,m') \in \pairs$
\item uninterpreted Boolean functions $\nodef_m : \booleans^\mu \maps \booleans$
encoding $\nodef(m)$ map
\item uninterpreted Boolean functions $\edgef_m : \booleans^\mu \maps \booleans$
encoding $\edgef(m)$ map
\end{enumerate}



\subsection{Solvers}
Due the combinatorial nature of the graphs, the search space is huge
and often hard to enumerate naively.
%
We need sophisticated solvers 


%%% Local Variables:
%%% mode: latex
%%% TeX-master: "main"
%%% End:

\ashu{Discuss the model in Biological terms.}

%
Therefore, the model only has aspects of a network or a graph.\\

\ankit{Discuss the yeast and eukaryotes example graph}
\begin{enumerate}
\item Complex form using three Q-SNAREs and one R-SNARE. In order for a vesicle to fuse to it's corresponding target compartment this combination has to be present.
\item In the graph shown the R-SNAREs are always in active. Every combination is formed by three active Q-SNAREs and one active R-SNARE.
\item  Qai, Qbi, Qci and Ri are just dummy names, real biological names can be found in the below table. 
\item In the diagram white circle molecules are always active. Other edge labels are inactive molecules.
\item Pairing matrix restriction: ith R-SNARE can fuse with only the corresponding ith Q-SNARES and vice-versa. No other combination is allowed.
\item Concept of v and t-SNAREs is replaced by Q and R-SNAREs. 
\end{enumerate}


%%\begin{tabular}{|l|l|}
%%\hline
%%\multicolumn{2}{|c|}{Team sheet} \\
%%\hline
%%GK & Paul Robinson \\
%%LB & Lucus Radebe \\
%%DC & Michael Duberry \\
%%DC & Dominic Matteo \\
%%\hline
%%RB & Didier Domi \\
%%MC & David Batty \\
%%MC & Eirik Bakke \\
%%MC & Jody Morris \\
%%FW & Jamie McMaster \\
%%ST & Alan Smith \\
%%ST & Mark Viduka \\
%%\hline
%%
%%\caption{Run-times for searching for models (in secs).}
%%  \label{tab:vts-grabh}
%%\end{tabular}

Qa1 Qb1 Qc1 R1
Stx18 Sec20 Slt1 Sec22b



%%% Local Variables:
%%% mode: latex
%%% TeX-master: "main"
%%% End:

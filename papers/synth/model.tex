The molecules transported by the VTS are themselves its regulators. The molecules in a
compartment/vesicle may be active or inactive. The molecules that are responsible for vesicle fusion
are called SNARE proteins~\cite{jahn2006snares, wickner2008membrane}. Active SNAREs present on vesicles (v-SNAREs) bind with their
cognate active SNAREs on the target compartment (t-SNAREs) to enable vesicle fusion. A cell
contains multiple kinds of v- and t-SNAREs. Only specific pairs of v and t SNAREs can bind to
each other and participate in fusion. Fusion compatible v- and t- SNAREs are determined by
biological experiments. Different vesicle-compartment fusions in the cell are brought about by
different v- and t-SNARE pairs. A molecule that participates in a given fusion reaction must not
interfere with fusion at different compartments or vesicles. Therefore, SNAREs must be kept in an
inactive form in appropriate compartments/vesicles. The activity of molecules is regulated by the
other molecules, i.e., the presence and absence of the other molecules in a compartment or vesicle
may make the molecule active or inactive. We call this regulation as activity functions.
In the VTS model, we assume that the system is in steady state and the concentrations of the
molecules in the compartments do not change over time. We define SNARE pairing specificity by a
fusion pairing relation containing pairs of SNAREs and molecular regulation by activity Boolean
functions. Since the system is in steady state, we expect that any molecule that leaves a
compartment must come back via some path on the graph. We call this property of VTS as stability.

Our model is inspired by~\cite{shukla2017discovering}. On the timescales of minutes, our following assumptions reasonably capture the important aspects of the Rothman-Schekman-Sudhof (RSS) model~\cite{rothman2002machinery} of vesicle traffic system.
\begin{enumerate}
\item A cell is a set of compartments exchanging vesicles.
\item Compartments are neither created nor destroyed.
\item Each compartment is in steady state, gain and loss balance.
\item Molecules are neither created nor destroyed.
\item Molecules move via vesicles of uniform size.
\item Identical vesicles have identical target compartments.
\item Fusion of vesicles to compartments is driven by specific SNARE pairing.
\item The activity of a SNARE can be regulated by other molecules present on the same compartment
or vesicle.
\item An active SNARE pair is necessary and sufficient for fusion. 
\end{enumerate}
 
SNARE proteins are the agents of vesicle fusion in eukaryotic cells. When SNAREs on vesicles (v-SNAREs) encounter their cognate SNAREs on target compartments (t-SNAREs), they form SNARE complexes~\cite{jahn2006snares}, and a single SNARE complex releases enough energy to enable membrane fusion~\cite{van2010one}. SNAREs are identified by the presence of a conserved 60-70 stretch of amino acids called the SNARE motif. Based on amino acid sequence, SNARE motifs fall into 4 classes: Qa, Qb, Qc, and R~\cite{jahn2006snares}. Across all intracellular vesicle fusion reactions, the associated SNARE complexes contain one of each of the four kinds of SNARE motifs; the v-SNARE contributes a single SNARE motif, usually it is an R-SNARE (although, exceptions are known: Sec22b and Ykt6 are both R-SNAREs which form parts of t-SNAREs~\cite{hong2005snares}) and the rest of the three SNARE
motifs are contributed by the t-SNARE. In the cell, different vesicle fusion reactions are associated with distinct v- and t-SNARE pairs.

The paper~\cite{shukla2017discovering} consider three Q SNARES as a single molecule,  we have extended this model by considering each complex molecule as distinct. In contrast to the~\cite{shukla2017discovering}, we allow Q and R-SNARE type distribution across the whole system to be uneven. In our model fusion is driven by an active combination of three Q SNARE and one R SNARE molecule. We have relaxed the pairing matrix constraint to comply with this fact. For biological efficiency and optimality reasons, we do not allow self-edges to be present in the VTS. 
%\begin{enumerate} 
%\item Complex form using three Q-SNAREs and one R-SNARE. In order for a vesicle to fuse to its corresponding target compartment, this combination has to be present.
%\item In the graph shown the R-SNAREs are always inactive. Every combination is formed by three active Q-SNAREs and one active R-SNARE.
%\item  Qai, Qbi, Qci, and Ri are just dummy names, real biological names can be found in the below table. 
%\item In the diagram white circle molecules are always active. Other edge labels are inactive molecules.
%\item Pairing matrix restriction: ith R-SNARE can fuse with only the corresponding ith Q-SNARES and vice-versa. No other combination is allowed.
%\end{enumerate}


%%\begin{tabular}{|l|l|}
%%\hline
%%\multicolumn{2}{|c|}{Team sheet} \\
%%\hline
%%GK & Paul Robinson \\
%%LB & Lucus Radebe \\
%%DC & Michael Duberry \\
%%DC & Dominic Matteo \\
%%\hline
%%RB & Didier Domi \\
%%MC & David Batty \\
%%MC & Eirik Bakke \\
%%MC & Jody Morris \\
%%FW & Jamie McMaster \\
%%ST & Alan Smith \\
%%ST & Mark Viduka \\
%%\hline
%%
%%\caption{Run-times for searching for models (in secs).}
%%  \label{tab:vts-grabh}
%%\end{tabular}



%%% Local Variables:
%%% mode: latex
%%% TeX-master: "main"
%%% End:

% \newenvironment{psmallmatrix}
%   {\left(\begin{smallmatrix}}
%   {\end{smallmatrix}\right)}

\begin{figure}[t]
  \centering
  \begin{tikzpicture}[->,>=stealth',auto,node distance=5cm,
  thick,main node/.style={rectangle,draw,font=\sffamily\Large\bfseries}]
  \node[main node,text width=3cm] (ga) {GA{\small \^{Qa4} \^{Qa6} \^{Qb4} \^{Qb6} \^{Qc4} \^{R6}}};
  \node[main node] (ic) [below right of=ga,yshift=20mm,xshift=20mm] {IC {\small \^{Qa6} \^{Qb6} \^{R1}}};
  \node[main node] (er) [below right of=ga] {ER {\small \^{Qa1} \^{Qb1} \^{R1}}};
  \node[main node,text width=2.5cm] (pm) [above right of=ga] {PM {\small \^{Qa5} \^{Qa7} \^{Qbc2} \^{Qbc7}}};
  \node[main node] (ee) [below right of=pm] {EE {\small \^{Qa2} \^{Qb2/3} \^{Qc2/3}}};
  \node[main node] (le) [above of=ee,yshift=-13mm,xshift=18mm] {LE {\small \^{Qa8} \^{Qb8} \^{Qc8}}};

  \path (ic) edge node [below] {\^{Qc6}} (ga);
  \path (er) edge[bend right=20] node [right] {\^{Qc6}} (ic);


  %er <> ga
  \path (er) edge[bend right=20] node [left] {\^{Qc6}} (ga);
  \path (er) edge[bend left=20] node [left] {R6} (ga);
  \path (ga) edge[bend right=40] node [left] {\^{Qc1}} (er);


  %ga <-> ee
  \path (ga) edge[bend left=25] node [above] {Qb2 Qc2} (ee);
  \path (ga) edge[bend left=10] node [above] {Qbc2/3} (ee);
  \path (ee) edge[bend left=0] node [below] {Qb2/3,Qa2,R2,\^{R4},Qc2/3} (ga);

  %le <-> pm
  \path (le) edge[bend right=10] node [above] {Qb7 Qc7} (pm);
  %ee -> le
  \path (ee) edge[bend right] node [below,rotate=70] {\^{R8},Qa7,Qbc7,R7} (le);

  %pm <-> ee
  \path (ee) edge[bend left=10] node [above] {\^{R3}} (pm);
  \path (pm) edge[bend left=30] node [above,rotate=-45,text width = 2.5cm] {\^{R2} {Qa7} Qbc7, R7, Qc7, Qa2} (ee);

  %ga->pm
  \path (ga) edge[bend left=10] node [right] {{Qb2} Qc2} (pm);
  \path (ga) edge[bend left=80] node [above] {\^{R2}} (pm);
  \path (ga) edge[bend left=52] node [above] {\^{R7}} (pm);
  \path (ga) edge[bend left] node [above] {\^{R8}} (pm);
  \end{tikzpicture}
  \caption{A found-in-nature VTS}
  \label{fig:mukund-vts}
\end{figure}


% Qa1 Qb1 Qc1 R1
% Stx18 Sec20 Slt1 Sec22b

% Qa6 Qb5 Qc6 R1
% Stx5 Gs27 Bet1 Sec22b

% Qa2 Qbc2/3 R2
% Stx13 Snap25/29 Vamp2

% Qa5 Qbc2 R2
% Stx1 Snap25 Vamp2

% Qa5 Qbc2 R3
% Stx1 Snap25 Vamp3

% Qa4 Qb4 Qc4 R4
% Stx16 Vti1a Stx6 Vamp4

% Qa6 Qb6 Qc6 R6
% Stx5 Gs28 Bet1 Ykt6

% Qa6 Qb6 Qc5 R6
% Stx5 Gs28 GS15 Ykt6

% Qa7 Qbc7 R7
% Stx4 Snap23 Vamp7

% Qa8 Qb8 Qc8 R7
% Stx7 Vti1b Stx8 Vamp7

% Qa8 Qb8 Qc8 R8
% Stx7 Vti1b Stx8 Vamp8	
  
%t snares are labelled within compartmetns, v snares on the edges
%and snare complexes on the left-hand side
The figure~\ref{fig:mukund-vts} represent mammalian SNARE map created by studying the references~\cite{somya}. 
To construct the map, we have assumed that vesicles only contain a single
active v-SNARE, and we have attributed t-SNAREs and inactive v-SNAREs that
travel between compartments to one of the known vesicles that go between
the same source and target compartments.
In order to identify the active SNARE complex involved in any particular
vesicle fusion, two criteria were used: 
\begin{enumerate}
\item[a.] The SNARE complex is formed \textit{in vivo}. In most papers, this is determined by immunoprecipitation of the SNARE complex from the relevant cell fraction. 
\item[b.] Blocking SNARE complex formation (for example, using antibodies against these SNAREs, or
using cytosolic forms of these SNAREs) blocks the specific transport step.
\end{enumerate}


Note that these vesicles have been collected from multiple cell types, and
any given cell type is likely to contain only a subset of the vesicles in
the map.

In this diagram the gray circles represent compartments, the identities of compartments are
written within: ER=endoplasmic reticulum, ERGIC=ER-Golgi intermediate compartment,
RE=recycling endosome, EE=early endosome, LE=late endosome, LYS=lysosome, PM=plasma
membrane. Thin black arrows represent vesicle edges.
% and hollow arrows represent maturation edges. 
The set of SNAREs contained in these edges are written alongside each edge. Actual names
of these SNAREs are mentioned in the key alongside. The red labels represent the paths taken by
SNARES of the complex Stx-4-SNAP23-VAMP7. Here, the cycles for two SNARE molecules,
VAMP7 and SNAP23 are complete. The green labels represent the paths taken by SNAREs of the
complex Stx13-SNAP25-VAMP2. Here the cycle for VAMP2 is complete.The blue labels represent
the paths taken by SNARES of the complex Stx5-Gs28-Bet1/GS15-Ykt6. Here, cycles for none of
the SNAREs are complete. So, even though most edges known so far are present in the map, paths for molecules (SNAREs) across the cell are still not complete.

We have used our tool to synthesize possible fixes to complete the graph. Based on the variation we were able to synthesis the this graph within 600s. We have created a rank of minimum edits to complete the graph. Our experiments suggest that the synthesis problems are solvable by modern solvers and the synthesis technology may be useful for biological research.

%%% Local Variables:
%%% mode: latex
%%% TeX-master: "main"
%%% End:
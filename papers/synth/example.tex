% \newenvironment{psmallmatrix}
%   {\left(\begin{smallmatrix}}
%   {\end{smallmatrix}\right)}

\begin{figure}[t]
  \centering
  \begin{tikzpicture}[->,>=stealth',auto,node distance=5cm,
  thick,main node/.style={rectangle,draw,font=\sffamily\Large\bfseries}]
  \node[main node,text width=3cm] (ga) {GA{\small \^{Qa4} \^{Qa6} \^{Qb4} \^{Qb6} \^{Qc4} \^{R6}}};
  \node[main node] (ic) [below right of=ga,yshift=8mm,xshift=5mm] {IC {\small \^{Qa6} \^{Qb6} \^{R1}}};
  \node[main node] (er) [below of=ga] {ER {\small \^{Qa1} \^{Qb1} \^{R1}}};
  \node[main node,text width=2.5cm] (pm) [right of=ga] {PM {\small \^{Qa5} \^{Qa7} \^{Qbc2} \^{Qbc7}}};
  \node[main node] (ee) [right of=pm] {EE {\small \^{Qa2} \^{Qb2/3} \^{Qc2/3}}};
  \node[main node] (le) [below of=ee] {LE {\small \^{Qa8} \^{Qb8} \^{Qc8}}};

  \path (ic) edge[bend right] node [right] {\^{Qc6}} (ga);
  \path (er) edge[bend left] node [left] {R6} (ga);
  \path (er) edge[bend right] node [left] {\^{Qc6}} (ga);
  \path (er) edge[bend right] node [right] {\^{Qc6}} (ic);
  \path (ga) edge[bend right=80] node [right] {\^{Qc1}} (er);
  \path (ee) edge[bend left] node [left] {\^{R8},Qa7,Qbc7,R7} (le);

  %pm <-> ee
  \path (ee) edge[bend left=10] node [above] {\^{R3}} (pm);
  \path (pm) edge[bend left=10] node [above] {{Qb2} Qc2} (ee);

  %ga->pm
  \path (ga) edge[bend left=10] node [above] {{Qb2} Qc2} (pm);
  \path (ga) edge[bend left=80] node [above] {\^{R2}} (pm);
  \path (ga) edge[bend left=52] node [above] {\^{R7}} (pm);
  \path (ga) edge[bend left] node [above] {\^{R8}} (pm);
  \end{tikzpicture}
  \caption{found-in-nature VTS}
  \label{fig:mukund-vts}
\end{figure}

In figure~\ref{fig:mukund-vts}


The figure~\ref{fig:mukund} represent mammalian SNARE map created by studying the references []. In this diagram the gray circles represent compartments, the identities of compartments are
written within: ER=endoplasmic reticulum, ERGIC=ER-Golgi intermediate compartment,
RE=recycling endosome, EE=early endosome, LE=late endosome, LYS=lysosome, PM=plasma
membrane. Thin black arrows represent vesicle edges.
% and hollow arrows represent maturation edges. 
The set of SNAREs contained in these edges are written alongside each edge. Actual names
of these SNAREs are mentioned in the key alongside. The red labels represent the paths taken by
SNARES of the complex Stx-4-SNAP23-VAMP7. Here, the cycles for two SNARE molecules,
VAMP7 and SNAP23 are complete. The green labels represent the paths taken by SNAREs of the
complex Stx13-SNAP25-VAMP2. Here the cycle for VAMP2 is complete.The blue labels represent
the paths taken by SNARES of the complex Stx5-Gs28-Bet1/GS15-Ykt6. Here, cycles for none of
the SNAREs are complete. 

%%% Local Variables:
%%% mode: latex
%%% TeX-master: "main"
%%% End:

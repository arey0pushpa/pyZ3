Here we will present the two VTS collected from the literature.

\subsection{Mammalian VTS}
\begin{figure}[t]
  \centering
  \begin{tikzpicture}[->,>=stealth',auto,node distance=5cm,
  thick,main node/.style={rectangle,draw,font=\sffamily\Large\bfseries}]
  \node[main node,text width=3cm] (ga) {GA{\small \^{Qa4} \^{Qa6} \^{Qb4} \^{Qb6} \^{Qc4} \^{R6}}};
  \node[main node] (ic) [below right of=ga,yshift=20mm,xshift=20mm] {IC {\small \^{Qa6} \^{Qb6} \^{R1}}};
  \node[main node] (er) [below right of=ga] {ER {\small \^{Qa1} \^{Qb1} \^{R1}}};
  \node[main node,text width=2.5cm] (pm) [above right of=ga] {PM {\small \^{Qa5} \^{Qa7} \^{Qbc2} \^{Qbc7}}};
  \node[main node] (ee) [below right of=pm] {EE {\small \^{Qa2} \^{Qb2/3} \^{Qc2/3}}};
  \node[main node] (le) [above of=ee,yshift=-13mm,xshift=18mm] {LE {\small \^{Qa8} \^{Qb8} \^{Qc8}}};

  \path (ic) edge node [below] {\^{Qc6}} (ga);
  \path (er) edge[bend right=20] node [right] {\^{Qc6}} (ic);


  %er <> ga
  \path (er) edge[bend right=20] node [left] {\^{Qc6}} (ga);
  \path (er) edge[bend left=20] node [left] {R6} (ga);
  \path (ga) edge[bend right=40] node [left] {\^{Qc1}} (er);


  %ga <-> ee
  \path (ga) edge[bend left=25] node [above] {Qb2 Qc2} (ee);
  \path (ga) edge[bend left=10] node [above] {Qbc2/3} (ee);
  \path (ee) edge[bend left=0] node [below] {Qb2/3,Qa2,R2,\^{R4},Qc2/3} (ga);

  %le <-> pm
  \path (le) edge[bend right=10] node [above] {Qb7 Qc7} (pm);
  %ee -> le
  \path (ee) edge[bend right] node [below,rotate=70] {\^{R8},Qa7,Qbc7,R7} (le);

  %pm <-> ee
  \path (ee) edge[bend left=10] node [above] {\^{R3}} (pm);
  \path (pm) edge[bend left=30] node [above,rotate=-45,text width = 2.5cm] {\^{R2} {Qa7} Qbc7, R7, Qc7, Qa2} (ee);

  %ga->pm
  \path (ga) edge[bend left=10] node [right] {{Qb2} Qc2} (pm);
  \path (ga) edge[bend left=80] node [above] {\^{R2}} (pm);
  \path (ga) edge[bend left=52] node [above] {\^{R7}} (pm);
  \path (ga) edge[bend left] node [above] {\^{R8}} (pm);
  \end{tikzpicture}
  \caption{A found-in-nature VTS. Nodes and edges are labelled with sets of molecules. \^{} indicates that the molecule is active.}
  \label{fig:mukund-vts}
\end{figure}

%%% Local Variables:
%%% mode: latex
%%% TeX-master: "main"
%%% End:


The figure~\ref{fig:mukund-vts} represent mammalian SNARE map
created by studying the wide array of literature.
% 
To construct the map, we have assumed that vesicles only contain a
single active v-SNARE, and we have attributed t-SNAREs and inactive
v-SNAREs that travel between compartments to one of the known vesicles
that go between the same source and target compartments.
%
In order to identify the active SNARE complex involved in any
particular vesicle fusion, we used two criteria.
%
The SNARE complex is formed \textit{in vivo}. In most papers, this is
determined by immunoprecipitation of the SNARE complex from the
relevant cell fraction.
%
Blocking SNARE complex formation (for example, using antibodies
against these SNAREs, or using cytosolic forms of these SNAREs) blocks
the specific transport step.
%
Note that these vesicles have been collected from multiple cell types, and
any given cell type is likely to contain only a subset of the vesicles in
the map.

In this figure, the rectangles represent compartments, the identities
of compartments are written within ER=endoplasmic reticulum,
ERGIC=ER-Golgi intermediate compartment, RE=recycling endosome,
EE=early endosome, LE=late endosome, LYS=lysosome, PM=plasma
membrane. The arrows represent vesicle edges.


% \ashu{Adopt }
% The set of SNAREs contained in these edges are written alongside each
% edge.
% %
% Actual names of these SNAREs are mentioned in the key
% alongside.
% %
% The red labels represent the paths taken by SNARES of the
% complex Stx-4-SNAP23-VAMP7.
% %
% Here, the cycles for two SNARE molecules,
% VAMP7 and SNAP23 are complete.
% %
% The green labels represent the paths
% taken by SNAREs of the complex Stx13-SNAP25-VAMP2.
% %
% Here the cycle for
% VAMP2 is complete. The blue labels represent the paths taken by SNARES
% of the complex Stx5-Gs28-Bet1/GS15-Ykt6.
% %
% Here, cycles for none of the
% SNAREs are complete.
% %
% So, even though most edges known so far are
% present in the map, paths for molecules (SNAREs) across the cell are
% still not complete.


\subsection{Yeast VTS}

\begin{figure}[t]
  \centering
  \begin{tikzpicture}[->,>=stealth',auto,node distance=4.5cm,
  thick,main node/.style={rectangle,draw,font=\sffamily\Large\bfseries}]
  \node[main node,,text width=2.5cm] (golgi) [] {GOLGI {\small \^{Qa2} \^{Qb2} \^{Qc2} \^{Qa5} \^{Qb5} \^{Qc5}}};
  \node[main node,text width=2.5cm] (pvac) [right of=golgi] {PRE-VAC {\small \^{Qa3} \^{Qb2} \^{R3} \^{R4}}};
  \node[main node,text width=2cm] (vac) [right of=pvac] {VAC {\small \^{Qa4} \^{Qb2} \^{R3} \^{R4}}};
  \node[main node,text width=2.5cm] (pm_up) [above right of=golgi] {PM {\small \^{Qa1} \^{Qbc1}}};
  \node[main node] (pm_down) [below right of=golgi] {PM {\small \^{Qa6} \^{Qb6} \^{R6} \^{R4}}};

  \path (golgi) edge[bend right] node [right] {\^{R1}} (pm_up);
  \path (pm_up) edge[bend right] node [right] {\^{R1}} (golgi);

  \path (golgi) edge[bend right] node [right] {\^{Qc6}} (pm_down);
  \path (pm_down) edge[bend right] node [right] {\^{R1}} (golgi);

  \path (golgi) edge[] node [below] {\^{Qc3}} (pvac);
  \path (pvac) edge[bend left=20] node [above] {\^{Qc4}} (vac);

  \end{tikzpicture}
  \caption{Yeast VTS}
  \label{fig:mukund-vts}
\end{figure}

%%% Local Variables:
%%% mode: latex
%%% TeX-master: "main"
%%% End:


In figure~\ref{fig:yeast-vts}, we present the yeast VTS.
%
We have borrowed the VTS from~\cite{burri2004complete}.
%
It has been adapted from the paper by
separating the v and the t SNAREs. 
%
It is clear that it is an incomplete description of the VTS.
%
For example, the inactive molecules were not reported in the reference.
%
We are currently searching for more literature that can help us complete
all known information about the VTS.
%


% It is important to note that it is not trivial collect all the information
% about such systems.

%%% Local Variables:
%%% mode: latex
%%% TeX-master: "main"
%%% End:

Vesicle Traffic Systems(VTS) are the material transport mechanisms
among the compartments inside the biological cells~\cite{vtsIntro}.
%
\ashu{Describe VTS in detail}
%
Vesicle Traffic Systems are the transport mechanisms inside the cells.
%
Different compartments are viewed as nodes and transport channels are
nodes between the graphs.

\ashu{Make a case for incomplete knowledge of VTSs.}
%Now we will work on
Since VTSs are partially known, 
%
the labels may not have been fully observed.


\ashu{Make a case for expected properties!!}
Furthermore, it is a matter of debate what properties the VTSs should
have, such as stability, i.e., every chemical that is leaving a
compartment comes back.


\ashu{Speculative: Make a case for completing partial parts of VTSs!!}

We model a VTS as a labelled graph.
%
Each compartment of VTS is a node of the graph and and the
transport channels between the nodes is an edge.
%
The set of chemicals present in a compartment is the label of of
the compartment.
%
Similarly, the set of transported chemicals carried by an edge is the label
on the edge.
%
%


There are several constraints that limits 

There are several constraints that encode the interaction.


The labels on the nodes 



%%% Local Variables:
%%% mode: latex
%%% TeX-master: "main"
%%% End:

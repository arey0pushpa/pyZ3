Vesicle Traffic Systems(VTS) are the material transport mechanisms
among the compartments inside the biological cells~\cite{vtsIntro}.
%
\ashu{Describe VTS in detail}
%
Vesicle Traffic Systems are the transport mechanisms inside the cells.
%
Different compartments are viewed as nodes and transport channels are
nodes between the graphs.

\ashu{Make a case for incomplete knowledge of VTSs.}
%Now we will work on
Since VTSs are partially known, 
%
the labels may not have been fully observed.


\ashu{Make a case for expected properties!!}
Furthermore, it is a matter of debate what properties the VTSs should
have, such as stability, i.e., every chemical that is leaving a
compartment comes back.


\ashu{Speculative: Make a case for completing partial parts of VTSs!!}

We model a VTS as a labelled graph.
%
Each compartment of VTS is a node of the graph and and the
transport channels between the nodes is an edge.
%
The set of molecules present in a compartment is the label of
the compartment.
% 
Similarly, the set of transported molecules carried by an edge is the label
on the edge.
%
A molecule in a node or edge may be active or inactive. 
%
The constraints discussed on the VTSs that allow a VTS to be feasible
are combinatorial in nature, for example stability or 3-connectedness.
%
The problem of synthesizing parts of a VTS that are unobserved such that 
the modified VTS 

It is appropriate that we translate 
We turn the problem of synthesizing parts of 
%
We consider several versions of the synthesis problem involving different
parts of VTSs that can be synthesized, such as modifying labels,
adding/deleting edges, learning activity functions, and adding nodes.
%
In order to synthesize the parts of a VTS such that it satisfies the constraints, 
we encode the synthesis problem into a satisfiability of QBF with uninterpreted
Boolean functions. 




Since an available, 

on the labels.


There are several constraints that limits 

There are several constraints that encode the interaction.


The labels on the nodes 



%%% Local Variables:
%%% mode: latex
%%% TeX-master: "main"
%%% End:

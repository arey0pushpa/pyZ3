
%Motivate VTS
Eukaryotic cells, including human cells, consist of multiple membrane-bound compartments.
%
Material is transported among these compartments by the vesicle
transport system (VTS).
%
Briefly, the source compartment produces a membrane-bound packet of
molecules called a vesicle.
%
After release, this vesicle specifically recognizes the correct target
compartment within the cell, and fuses with it~\cite{alberts2013essential}.
%
A lot of information about the molecules that form the machinery of
the VTS has been discovered, including their regulatory interaction
with each other~\cite{bonifacino2004mechanisms}. 
% %
% \todo{connect to function }{We also know a lot about the specific functions} of these molecules and how these molecules interact amongst each other.
%
In spite of this detailed knowledge at the level of the molecules, the
structure of the VTS network, or the road-map of the eukaryotic cell,
is far from complete.
%
For example, although the localization of various SNAREs --a class of
molecules that participate in the control of VTS-- in the cell is
known, and also their site of action~\cite{hong2014tethering}, for
most SNAREs, how they first reached the compartments they reside in is
not known.
%
The current knowledge of the network is put together from a patchwork
of biological experiments and is scattered across several publications.
%
Even after this information is collected and put together, we find
that the network obtained is still not complete; new vesicles and new
contents in previously known vesicles are constantly being discovered
(some new discoveries include~\cite{chanaday2017you, d2017tethering,
rodepeter2017indication, zhao2017conserved}).
%
The synthesis for the unknown pieces may be assisted by computation on
the graph model of VTSs.
%
% 
%In this work, we use a model described in [] to find out, given a partial
%VTS graph and details about interactions amongst its molecules, what the complete graph of the
%VTS could look like. (Please look at Appendix A for the detailed discussion on pros and cons of the
%model.
%
%Eukaryotic cells, including human cells, consists of multiple compartments.
%
%Vesicle Traffic Systems(VTSs) are the material transport mechanisms among the compartments inside the cells~\cite{vtsIntro}.
%
%Almost all subsystems of the cells depend on VTSs.
%
%Therefore, understanding how VTSs functions is one of the key
%questions of cell biology.
%
In this paper, we are looking at the computational questions 
arising from the VTSs.


%
VTSs are regulated by the same molecules that they transport.
%
For the purpose of this paper, the VTS molecules we focus on are the transmembrane SNARE proteins.
%
SNAREs drive the recognition of the target compartment by vesicles and
their subsequent fusion.
%
The SNAREs can be divided into v-SNAREs (which are present
on vesicles) and t-SNAREs (which are present on compartments). A
vesicle fuses with a compartment if its v-SNARE can form a complex
with the t-SNARE present on that compartment.
%
Not all v- and t- SNARE combinations can form complexes; this
constraint forms part of the basis for the specificity of vesicle
traffic~\cite{jahn2006snares}.
%
The activity of SNARE proteins is further regulated by their
interaction with other traffic molecules \ashu{ [do we need the following], such as SM
proteins~\cite{sudhof2009membrane} and proteins of tethering
complexes~\cite{hong2014tethering}. }
\ankit{From our papers prospective NO. But Biologically Yes. Mukund have written this earlier then stated that "In this paper we focus only on SNARE's ... But somya has did this in reverse. Similar was the case many places, a few of which I fixed. I can give one more try."}
%

We use the model of VTSs that has been presented
in~\cite{shukla2017discovering}.
%
Please look at Appendix~\ref{sec:model} for a detailed discussion on
pros and cons of the model.
%
We model the system as a labeled graph, where compartments are nodes
and transport vesicles are edges.
%
The molecular compositions of the compartments and vesicles are the
node and edge labels respectively.
%
The molecules can be active or inactive on any a compartment or
vesicle.
%
The activity states of molecules are also included in the labels.
%
Due to the biology of SNAREs of the VTSs our interest, a vesicle is enabled
by a set of {\em four} molecules such that one part of the set occurs in the
vesicle and the other part occurs in the target of the vesicle compartment.
%
The partition always divides the set in the set of three and one molecules.
%
% Describe VTS in detail
%Molecules on compartments and on vesicles can be in either an active or an inactive state.
% the term vesicle has already been introduced.. using two terms for the same object seems confusing. 
%
% (The molecules are transported to the other compartments via
%unidirectional channels.) cannot comment on unidirectionality of vesicles. they may be undergoing multiple cycles of back-fusion for all we %know.  
%
%The active molecules regulate the vesicle they occur on.
%
%A vesicle is enabled by a set of molecules such that one part of the set occurs in the vesicle and the other part occurs in the target compartment.
% use either target or destination consistently throughout..i prefer target because that's what the t in t-SNARE stands for.
%
The enabling molecules must be active in the vesicle
and target compartment respectively.
%
The pairs are called {\em fusing} sets and analogously
the vesicle is considered to be {\em fused} with the
destination compartment.
%
Not all sets of molecules can participate in the fusion; in the
biological cells, fusogenic SNARE complexes are discovered through
experiments.
%
%Not all molecule sets can participate in the fusion, the
%potential fusion pairs are usually determined by biological experiments.
%
Generally, the fusing pairs are found to be distinct for distinct vesicle-compartment fusions.
%
To ensure that a molecule that has participated in a fusion does not
interfere with fusion at compartments and vesicles, in the model, we
require that the molecule is inactive on appropriate compartments.
%
The activity of molecules is regulated by the other molecules, i.e.,
the presence and absence of the other molecules in a compartment or
vesicle may make the molecule active or inactive.
%
We call this regulation as {\em activity functions}.
% Defending our model
%
%In this VTS model, we assume that the system is in steady state and the
%concentrations of the molecules in the compartments do not change over
%time.
%
%We model the system as a labelled graph, where compartments are nodes and
%transport vesicles are edges.
%
%The set of molecules present and their activity in a compartment or
%a vesicle is the label of the respective node or edge.
%
The regulation controls are defined by a fusion pairing relation
containing pairs of molecules and activity
Boolean functions.

%
In the model, we assume that the system is in steady state and the
concentrations of the molecules in compartments do not change over
time.
%
Since our system is in steady state, we expect that any molecule that
leaves a compartment must come back via some path on the graph.
%
We call this property of VTS as {\em stability}.

% Partial info defence
%
As we have discussed earlier, our understanding of VTSs is partial.
%
% It largely means that the current knowledge is due to the patchwork
% of biological experiments~\cite{model} for identifying the compartments,
% channels, molecules, and their activity and regulatory control.
% %
% Often the information is scattered around in several publications.
% %
% Even after with some effort we may collate all the known information about
% a VTS, we may still have unknown pieces of the system; new vesicles and new contents in previously known vesicles are constantly being discovered.
%
%{Speculative: Make a case for completing partial parts of VTSs!!}
%
%For example, it is possible that some edges may not have been
%discovered in a VTS that does not satisfy the stability condition.
% nature of missing information has already been elaborated upon in the first paragraph
%
The synthesis for the unknown pieces may be {\em assisted} by computation on
the graph model of VTSs.
%
In this paper, we consider several versions of the synthesis problem
involving different parts of VTSs that can be synthesized, such as
modifying labels, adding/deleting edges, and learning activity functions.
%
We also consider variation on the properties against which we do
synthesis, namely stability, and $k$-connectedness that states that the VTS remains connected after removing any $k-1$ edges.
%
We have assumed that the given partial VTS is always well-fused whereas properties like stability and k-connectedness may not hold in the partial VTS.
%
In order to synthesize the parts of a VTS such that it satisfies the
constraints, we encode the synthesis problem into one of satisfiability of
quantified Boolean formulas(QBFs). 
%

We have implemented the encoding in a flexible tool,
which can handle a wide range of synthesis queries.
%
We have applied our tool on several VTSs including two found-in-nature
VTSs.
%
% \begin{figure}[t]
  \centering
  \begin{tikzpicture}[->,>=stealth',auto,node distance=5cm,
  thick,main node/.style={rectangle,draw,font=\sffamily\Large\bfseries}]
  \node[main node,text width=3cm] (ga) {GA{\small \^{Qa4} \^{Qa6} \^{Qb4} \^{Qb6} \^{Qc4} \^{R6}}};
  \node[main node] (ic) [below right of=ga,yshift=20mm,xshift=20mm] {IC {\small \^{Qa6} \^{Qb6} \^{R1}}};
  \node[main node] (er) [below right of=ga] {ER {\small \^{Qa1} \^{Qb1} \^{R1}}};
  \node[main node,text width=2.5cm] (pm) [above right of=ga] {PM {\small \^{Qa5} \^{Qa7} \^{Qbc2} \^{Qbc7}}};
  \node[main node] (ee) [below right of=pm] {EE {\small \^{Qa2} \^{Qb2/3} \^{Qc2/3}}};
  \node[main node] (le) [above of=ee,yshift=-13mm,xshift=18mm] {LE {\small \^{Qa8} \^{Qb8} \^{Qc8}}};

  \path (ic) edge node [below] {\^{Qc6}} (ga);
  \path (er) edge[bend right=20] node [right] {\^{Qc6}} (ic);


  %er <> ga
  \path (er) edge[bend right=20] node [left] {\^{Qc6}} (ga);
  \path (er) edge[bend left=20] node [left] {R6} (ga);
  \path (ga) edge[bend right=40] node [left] {\^{Qc1}} (er);


  %ga <-> ee
  \path (ga) edge[bend left=25] node [above] {Qb2 Qc2} (ee);
  \path (ga) edge[bend left=10] node [above] {Qbc2/3} (ee);
  \path (ee) edge[bend left=0] node [below] {Qb2/3,Qa2,R2,\^{R4},Qc2/3} (ga);

  %le <-> pm
  \path (le) edge[bend right=10] node [above] {Qb7 Qc7} (pm);
  %ee -> le
  \path (ee) edge[bend right] node [below,rotate=70] {\^{R8},Qa7,Qbc7,R7} (le);

  %pm <-> ee
  \path (ee) edge[bend left=10] node [above] {\^{R3}} (pm);
  \path (pm) edge[bend left=30] node [above,rotate=-45,text width = 2.5cm] {\^{R2} {Qa7} Qbc7, R7, Qc7, Qa2} (ee);

  %ga->pm
  \path (ga) edge[bend left=10] node [right] {{Qb2} Qc2} (pm);
  \path (ga) edge[bend left=80] node [above] {\^{R2}} (pm);
  \path (ga) edge[bend left=52] node [above] {\^{R7}} (pm);
  \path (ga) edge[bend left] node [above] {\^{R8}} (pm);
  \end{tikzpicture}
  \caption{A found-in-nature VTS. Nodes and edges are labelled with sets of molecules. \^{} indicates that the molecule is active.}
  \label{fig:mukund-vts}
\end{figure}

%%% Local Variables:
%%% mode: latex
%%% TeX-master: "main"
%%% End:

% %
% In figure~\ref{fig:mukund-vts}, we present one of the two for
% mammalian cells obtained by studying the
% literature.\footnote{Please read appendix~\ref{sec:ex-vts} for more biological details.}
% %
% The VTS has six nodes and 55 molecules.
% %
% We can easily check that stability condition is not satisfied for
% many molecules for example ???.
% %
% On the example, our synthesis tool reports that minimum ?? edges needed
% to be added to make the VTS stable.
% %
% The dashed edges are the suggested edges from our tool.
% %

Our experiments suggest that some of the synthesis problems are
solvable by modern solvers and the synthesis technology may be useful
for biological research.
%
% We have also applied our tool on various synthetic examples to
% demonstrate that our tool scale up to the graph size nodes, which
% is the size of typical VTSs in a eukaryotic cell.

% The following are the contributions of this work:
% \begin{itemize}
% \item We have identified an interesting application of
%   the synthesis technology.
% \item We have developed encoding of the synthesis problem in QBF.
% \item A user-friendly and scalable tool based on QBF solvers
% \end{itemize}

The rest of the paper is organized as follows.
%
% In section~\ref{sec:example}, we present a motivating example.
% %
In section~\ref{sec:prelim}, we present the graph model of VTSs and encoding of several
constraints on VTSs.
%
In section~\ref{sec:encoding}, we present the synthesis problems and their
encoding into QBF satisfiability.
%
In section~\ref{sec:experiments}, we present our implementation and experimental results.
%
We discuss related work in~\ref{sec:related} and conclude in section~\ref{sec:conclusion}.

%%% Local Variables:
%%% mode: latex
%%% TeX-master: "main"
%%% End:

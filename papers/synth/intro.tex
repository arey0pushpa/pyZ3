%Motivate VTS
Eukaryotic cells, including human cells, consists of multiple compartments.
%
Vesicle Traffic Systems(VTSs) are the material transport mechanisms
among the compartments inside the eukaryotic cells~\cite{vtsIntro}.
%
Almost all subsystems of eukaryotic cells depend on VTSs.
%
Therefore, understanding how VTSs functions is one of the key
questions of cell biology.
%
In this paper, we are looking at the computational questions 
arise from the VTSs.
%
In the following, we use the model of VTSs that has been presented
in~\cite{VTS}.
%
Please look at appendix~\ref{sec:model}
for the detailed discussion on pros and cons of the model.

% Describe VTS in detail
While molecules are transported by VTSs, VTSs are also
regulated by the molecules.
%
Each compartment contains a set of molecules.
%
The molecules are transported to the other compartments via
unidirectional channels.  
%
The molecules in a compartment may be active or not active.
%
Similarly, a molecule can be active along in a channel. 
%
The active molecules are the regulators of the channels.
%
A channel is enabled by a pair of molecules
such that one occurs in the channel and the other
occurs in the destination compartment.
%
Both the molecules must be active in the channel
and compartment respectively.
%
The pairs are called {\em fusing} molecules and analogously
the channel is considered to be {\em fused} with the
compartment.
%
Not all the pairs of molecules can participate in the fusion, it is
usually determined by wet experiments the potential fusion pairs.
%
To ensure effective regulation, the fusing molecule pairs must be
unique for each fused channel-compartment pair.
%
Since a molecule has to be active in order to participate in fusion,
the molecule may be present in the other places without being active
and does not interfere with regulation of the other fused
channel-compartment pairs.
%
The activity of molecules is regulated by the other molecules, i.e.,
the presence and absence of the other molecule in a compartment or
channel may make the active or inactive.
%
We call this regulation {\em activity functions}.


Although the VTSs are very important, our understanding 


In the VTS model, we assume that the system is in study state and
the concentration of molecules does not change over time.
%
Therefore, there is no aspect of 

%
They are known 

Some of the molecules that are pre
%
%
Different compartments are viewed as nodes and transport channels are
nodes between the graphs.

\ashu{Make a case for incomplete knowledge of VTSs.}
%Now we will work on
Since VTSs are partially known, 
%
the labels may not have been fully observed.


\ashu{Make a case for expected properties!!}
Furthermore, it is a matter of debate what properties the VTSs should
have, such as stability, i.e., every chemical that is leaving a
compartment comes back.


\ashu{Speculative: Make a case for completing partial parts of VTSs!!}

We model a VTS as a labelled graph.
%
Each compartment of VTS is a node of the graph and and the
transport channels between the nodes is an edge.
%
The set of molecules present in a compartment is the label of
the compartment.
% 
Similarly, the set of transported molecules carried by an edge is the label
on the edge.
%
A molecule in a node or edge may be active or inactive. 
%
The constraints discussed on the VTSs that allow a VTS to be feasible
are combinatorial in nature, for example stability or 3-connectedness.
%
The problem of synthesizing parts of a VTS that are unobserved such that 
the modified VTS 

It is appropriate that we translate 
We turn the problem of synthesizing parts of 
%
We consider several versions of the synthesis problem involving different
parts of VTSs that can be synthesized, such as modifying labels,
adding/deleting edges, learning activity functions, and adding nodes.
%
In order to synthesize the parts of a VTS such that it satisfies the constraints, 
we encode the synthesis problem into a satisfiability of QBF with uninterpreted
Boolean functions. 


Since an available, 

on the labels.


There are several constraints that limits 

There are several constraints that encode the interaction.


The labels on the nodes 



%%% Local Variables:
%%% mode: latex
%%% TeX-master: "main"
%%% End:

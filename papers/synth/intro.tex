%Motivate VTS
Eukaryotic cells, including human cells, consists of multiple compartments.
%
Vesicle Traffic Systems(VTSs) are the material transport mechanisms
among the compartments inside the eukaryotic cells~\cite{vtsIntro}.
%
Almost all subsystems of eukaryotic cells depend on VTSs.
%
Therefore, understanding how VTSs functions is one of the key
questions of cell biology.
%
In this paper, we are looking at the computational questions 
arise from the VTSs.
%
In the following, we use the model of VTSs that has been presented
in~\cite{VTS}.
%
Please look at appendix~\ref{sec:model}
for the detailed discussion on pros and cons of the model.

% Describe VTS in detail
While molecules are transported by VTSs, VTSs are also
regulated by the molecules.
%
Each compartment contains a set of molecules.
%
The molecules are transported to the other compartments via
unidirectional channels.  
%
The molecules in a compartment may be active or not active.
%
Similarly, a molecule can be active along in a channel. 
%
The active molecules are the regulators of the channels.
%
A channel is enabled by a pair of molecules
such that one occurs in the channel and the other
occurs in the destination compartment.
%
Both the molecules must be active in the channel
and compartment respectively.
%
The pairs are called {\em fusing} molecules and analogously
the channel is considered to be {\em fused} with the
compartment.
%
Not all the pairs of molecules can participate in the fusion, it is
usually determined by wet experiments the potential fusion pairs.
%
To ensure effective regulation, the fusing molecule pairs must be
unique for each fused channel-compartment pair.
%
Since a molecule has to be active in order to participate in fusion,
the molecule may be present in the other places without being active
and does not interfere with regulation of the other fused
channel-compartment pairs.
%
The activity of molecules is regulated by the other molecules, i.e.,
the presence and absence of the other molecule in a compartment or
channel may make the active or inactive.
%
We call this regulation as {\em activity functions}.

% Defending our model
%
In the VTS model, we assume that the system is in study state and
the concentrations of the molecules do not change over time.
%
Therefore, the system only has aspects of a network or a graph.
%
We model the system as a labelled graph, where compartments are nodes and
transport channels are edges.
%
The set of molecules present and their activity in a compartment or
a channel is the label of the respective node or edge.
%
The regulation controls are defined by a fusion pairing matrix
containing Boolean values for each pair of molecules and activity
Boolean functions.
%
Since the system is in study state, we expect that any molecule that
leaves a compartment must come back via some path on the graph.
%
We call this property of VTS as {\em stability}.

% Partial info defence
%
Although the VTSs are important, our understanding of VTSs are partial.
%
It largely means patch work of wet experiments~\cite{model} for
identifying the compartments, channels, molecules, their activity, and
regulatory control.
%
Often the information is scattered around in several publications.
%
Even after with some effort we may collate all the known information about
a VTS, we may still have unknown pieces of the system.
%
The search for the unknown pieces may be {\em assisted} by computation on
the graph model of VTSs.
%
For example, we suspect that some edges are missing from the graph, since
we expect the must satisfy stability.
%
Due the combinatorial nature of graphs, the search space is huge and often
hard to enumerate naively.
%
We need specialized methods to perform the search.

%
We may translate the problem into a synthesis question, in which 
we have a partial system  and a given property that
the system needs to satisfy.
%
The synthesis method completes the system such that the property
is satisfied by it.
%


In synthesis, we may have a partial

\ashu{Make a case for expected properties!!}
Furthermore, it is a matter of debate what properties the VTSs should
have, such as stability, i.e., every chemical that is leaving a
compartment comes back.


\ashu{Speculative: Make a case for completing partial parts of VTSs!!}

The constraints discussed on the VTSs that allow a VTS to be feasible
are combinatorial in nature, for example stability or 3-connectedness.
%
The problem of synthesizing parts of a VTS that are unobserved such that 
the modified VTS 

It is appropriate that we translate 
We turn the problem of synthesizing parts of 
%
We consider several versions of the synthesis problem involving different
parts of VTSs that can be synthesized, such as modifying labels,
adding/deleting edges, learning activity functions, and adding nodes.
%
In order to synthesize the parts of a VTS such that it satisfies the
constraints, we encode the synthesis problem into a satisfiability of
QBF with uninterpreted Boolean functions. 


Since an available, 

on the labels.


There are several constraints that limits 

There are several constraints that encode the interaction.


The labels on the nodes 



%%% Local Variables:
%%% mode: latex
%%% TeX-master: "main"
%%% End:

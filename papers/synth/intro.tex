%Motivate VTS
Eukaryotic cells, including human cells, consists of multiple compartments.
%
Vesicle Traffic Systems(VTSs) are the material transport mechanisms
among the compartments inside the cells~\cite{vtsIntro}.
%
Almost all subsystems of the cells depend on VTSs.
%
Therefore, understanding how VTSs functions is one of the key
questions of cell biology.
%
In this paper, we are looking at the computational questions 
arise from the VTSs.
%
In the following, we use the model of VTSs that has been presented
in~\cite{VTS}.
%
Please look at appendix~\ref{sec:model}
for the detailed discussion on pros and cons of the model.

% Describe VTS in detail
While VTSs transports molecules, VTSs are also
regulated by the molecules.
%
Each compartment contains a set of molecules.
%
The molecules are transported to the other compartments via
unidirectional channels.  
%
The molecules in a compartment may be active or not active.
%
Similarly, a molecule can be active in a channel. 
%
The active molecules are the regulators of the channels.
%
A channel is enabled by a pair of molecules
such that one occurs in the channel and the other
occurs in the destination compartment.
%
Both the molecules must be active in the channel
and compartment respectively.
%
The pairs are called {\em fusing} molecules and analogously
the channel is considered to be {\em fused} with the
destination compartment.
%
Not all the pairs of molecules can participate in the fusion, the
potential fusion pairs are usually determined by wet experiments.
%
To ensure effective regulation, the fusing molecule pairs must be
unique for each fused channel-compartment pair.
%
A molecule that have participated in a fusion must not interfere
with fusion at different compartments or channels.
%
Therefore, the molecule must be inactive in appropriate compartments.
%
The activity of molecules is regulated by the other molecules, i.e.,
the presence and absence of the other molecule in a compartment or
channel may make the active or inactive.
%
We call this regulation as {\em activity functions}.

% Defending our model
%
In the VTS model, we assume that the system is in study state and
the concentrations of the molecules in the compartments do not change over time.
%
We model the system as a labelled graph, where compartments are nodes and
transport channels are edges.
%
The set of molecules present and their activity in a compartment or
a channel is the label of the respective node or edge.
%
The regulation controls are defined by a fusion pairing relation
containing pairs of molecules and activity
Boolean functions.
%
Since the system is in study state, we expect that any molecule that
leaves a compartment must come back via some path on the graph.
%
We call this property of VTS as {\em stability}.

% Partial info defence
%
Although the VTSs are important, our understanding of VTSs are partial.
%
It largely means that the current knowledge is due to the patch work
of wet experiments~\cite{model} for identifying the compartments,
channels, molecules, their activity, and regulatory control.
%
Often the information is scattered around in several publications.
%
Even after with some effort we may collate all the known information about
a VTS, we may still have unknown pieces of the system.
%
%{Speculative: Make a case for completing partial parts of VTSs!!}
%
For example, we may suspect that some edges are missing in a VTS that
does not satisfy the stability condition.
%
The synthesis for the unknown pieces may be {\em assisted} by computation on
the graph model of VTSs.
%

In this paper, we consider several versions of the synthesis problem
involving different parts of VTSs that can be synthesized, such as
modifying labels, adding/deleting edges, learning activity functions,
and adding nodes.
%
We also consider variation on the properties against which we do synthesis,
namely stability, and $k$-connectedness that states
that VTS remains connected after removing any $k-1$ edges.
%
In order to synthesize the parts of a VTS such that it satisfies the
constraints, we encode the synthesis problem into a satisfiability of
quantified Boolean formulas(QBFs). 
%

We have implemented the encoding in a highly flexible tool,
which can handle a wide range of synthesis query.
%
We have applied our tool on two found-in-nature VTSs, namely
M1~\cite{} and S1~\cite{}.
%
We have also applied our tool on various synthetic examples to
demonstrate that our tool scale upto the graph size ??, which
is the size of typical VTSs in a eukaryotic cell.

The following are the contributions of this work:
\begin{itemize}
\item We have identified an interesting application of
  the synthesis technology.
\item We have developed encoding of the synthesis problem in QBF.
\item A user friendly and scalable tool based on well known SMT solver Z3
\end{itemize}

The rest of the paper is organized as follows.
%
In section~\ref{sec:prelim}, we present the graph model of VTSs and encoding of several
constraints on VTSs.
%
In section~\ref{sec:prelim}, we present the synthesis problems and their
encoding into QBF satisfiability.
%
In section~\ref{sec:experiments}, we present our implementation and experimental results.
%
We discuss related work in~\ref{sec:related} and conclude in section~\ref{sec:conclusion}.



%%% Local Variables:
%%% mode: latex
%%% TeX-master: "main"
%%% End:

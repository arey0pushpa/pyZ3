\usepackage{mathtools}
\usepackage{makecell}
\usepackage{graphicx}
\usepackage{amsmath}
\usepackage{pdflscape}
% \usepackage[top=0.85in,left=2.75in,footskip=0.75in]{geometry}

% I ADDED FOR THE CHANGE IN ENUMERATE TO ALPHABBETICAL
\usepackage{enumitem}


% amsmath and amssymb packages, useful for mathematical formulas and symbols
\usepackage{amsmath,amssymb}

\renewcommand{\figurename}{Fig.{}}

% Use adjustwidth environment to exceed column width (see example table in text)
\usepackage{changepage}

% Use Unicode characters when possible
\usepackage[utf8x]{inputenc}

% textcomp package and marvosym package for additional characters
\usepackage{textcomp,marvosym}

% cite package, to clean up citations in the main text. Do not remove.
\usepackage{cite}

% Use nameref to cite supporting information files (see Supporting Information section for more info)
\usepackage{nameref,hyperref}

% line numbers
\usepackage[right]{lineno}

% ligatures disabled
\usepackage{microtype}
\DisableLigatures[f]{encoding = *, family = * }

% color can be used to apply background shading to table cells only
\usepackage[table]{xcolor}

\usepackage{todonotes}

% array package and thick rules for tables
\usepackage{array}

% Use package Listing to add code in our Manuscript
\usepackage{listings} 

% Added for the sub-pictures
\usepackage{subcaption}

% Added for the multi-column 
\usepackage[british]{babel}
\usepackage{hhline}
\usepackage{multirow}
\usepackage[figurename=Fig]{caption}

%ADEDED BY ANKIT
\usepackage{tkz-orm}
\usepackage{lineno}
\linenumbers
%%%<
%\usepackage[active,tightpage]{preview}
%\PreviewEnvironment{tikzpicture}

\usepackage{verbatim}
\usepackage{pgfplots}
\newcommand*{\equal}{=}
 \usepackage{tikz}
 \usetikzlibrary{arrows}
 \usepackage{xparse}
\usetikzlibrary{matrix,backgrounds}
\pgfdeclarelayer{myback}
\pgfsetlayers{myback,background,main}

\tikzset{mycolor/.style = {line width=1bp,color=#1}}%
\tikzset{myfillcolor/.style = {draw,fill=#1}}%


\NewDocumentCommand{\highlight}{O{blue!40} m m}{%
\draw[mycolor=#1] (#2.north west)rectangle (#3.south east);
}

\NewDocumentCommand{\fhighlight}{O{blue!40} m m}{%
\draw[myfillcolor=#1] (#2.north west)rectangle (#3.south east);
}
 \usetikzlibrary{matrix,decorations.pathreplacing, calc, positioning}



% create "+" rule type for thick vertical lines
\newcolumntype{+}{!{\vrule width 2pt}}
\renewcommand{\thesubfigure}{\Alph{subfigure}}

% create \thickcline for thick horizontal lines of variable length
\newlength\savedwidth
\newcommand\thickcline[1]{%
  \noalign{\global\savedwidth\arrayrulewidth\global\arrayrulewidth 2pt}%
  \cline{#1}%
  \noalign{\vskip\arrayrulewidth}%
  \noalign{\global\arrayrulewidth\savedwidth}%
}

\usepackage{array}
\newcolumntype{L}[1]{>{\raggedright\let\newline\\\arraybackslash\hspace{0pt}}m{#1}}
\newcolumntype{C}[1]{>{\centering\let\newline\\\arraybackslash\hspace{0pt}}m{#1}}
\newcolumntype{R}[1]{>{\raggedleft\let\newline\\\arraybackslash\hspace{0pt}}m{#1}}


% Remove comment for double spacing
%\usepackage{setspace} 
%\doublespacing

% Text layout
% \raggedright
% \setlength{\parindent}{0.5cm}
% \textwidth 5.25in 
% \textheight 8.75in
% create "+" rule type for thick vertical lines
% \newcolumntype{+}{!{\vrule width 2pt}}
\renewcommand{\thesubfigure}{\Alph{subfigure}}

% \thickhline command for thick horizontal lines that span the table
\newcommand\thickhline{\noalign{\global\savedwidth\arrayrulewidth\global\arrayrulewidth 2pt}%
\hline
\noalign{\global\arrayrulewidth\savedwidth}}

% \raggedright
% \setlength{\parindent}{0.5cm}
% \textwidth 5.25in 
% \textheight 8.75in


% \usepackage[aboveskip=1pt,labelfont=bf,labelsep=period,justification=raggedright,singlelinecheck=off]{caption}
% \renewcommand{\figurename}{Fig}
% \usepackage{epstopdf}
% \AppendGraphicsExtensions{.tif}

\newcommand{\booleans}{\mathbb{B}}
\newcommand{\naturals}{\mathbb{N}}
\newcommand{\integers}{\mathbb{Z}}
\newcommand{\ordinals}{\mathbb{O}}
\newcommand{\numarals}{\mathbb{I}}

\newcommand{\maps}{\rightarrow}

\newcommand{\union}{{\cup} }
\newcommand{\Union}{{\bigcup} }
\newcommand{\powerset}[1]{2^{#1}}
\newcommand{\intersection}{{\cap} }
\newcommand{\intersect}{\intersection}
\newcommand{\Intersection}{{\bigcap} }
\newcommand{\compose}{{\circ} }


\newcommand{\ltrue}{\mathbf{tt}}
\newcommand{\lfalse}{\mathbf{ff}}
\newcommand{\limplies}{\Rightarrow}
\newcommand{\lxor}{\oplus}
\newcommand{\Land}{\bigwedge}
\newcommand{\Lor}{\bigvee}
\newcommand{\Lxor}{\bigoplus}
\newcommand{\lequiv}{\Leftrightarrow}
\newcommand{\landplus}{\mathrel{:\hspace{-3pt}\land\hspace{-3pt}=}}
\newcommand{\lorplus}{\mathrel{:\hspace{-3pt}\lor\hspace{-3pt}=}}

\newcommand{\nodes}{N}
\newcommand{\mols}{M}
\newcommand{\nlabel}{L}
\newcommand{\edges}{E}
\newcommand{\pairs}{\mathcal{P}}
\newcommand{\nodef}{f}
\newcommand{\edgef}{g}


\newcommand{\lorem}{{\bf LOREM}}
\newcommand{\ipsum}{{\bf IPSUM}}

\newcommand{\zthree}{\textsc{Z3}}
\newcommand{\ourtool}{\textsc{VTSSynth}}
\newcommand{\depqbf}{\textsc{DepQBF}}
\newcommand{\ashu}[1]{ {\textcolor{magenta} {Ashutosh: #1}} }
\newcommand{\mukund}[1]{ {\textcolor{red} {Mukund: #1}} }
\newcommand{\srivas}[1]{ {\textcolor{blue} {Srivas: #1}} }
\newcommand{\ankit}[1]{ {\textcolor{green!50!black}{Ankit: #1}} }

\newtheorem{df}{Definition}

%--------------------- DO NOT ERASE BELOW THIS LINE --------------------------

%%% Local Variables:
%%% mode: latex
%%% TeX-master: "main"
%%% End:

\begin{table}[!ht]
%\begin{adjustwidth}{-2.25in}{0in} % Comment out/remove adjustwidth environment if table fits in text column.
\centering
\def\arraystretch{1.6}
\caption{
{\bf SNARE regulation and graph connectedness.}}
\begin{tabular}{|c|l|l|c|}
\hline
\bf{Sr.No} & \multicolumn{1}{c|}{\bf{Regulation on node}} & \bf{Regulation on edge} & \bf{Connectedness}\\ \thickhline
1. & Boolean function & Boolean function & N = 2C S = 3C \\ \hline
2. & None & Boolean function & N \& S = 3C\\ \hline
3. & Boolean function & SNARE-SNARE inhibition & N \& S = 4C\\ \hline
4. & None & SNARE-SNARE inhibition & No graphs found\\ \hline
5. & Boolean function & None & No graphs found\\ \hline
6. &  None & None & No graphs found\\ \hline

%6. & Arbitrary Boolean & Specificity matrix and every edge is distinct & N \& S = 3C\\ \hline

\end{tabular}
\label{tab:smt-grph}
%\end{adjustwidth}
\end{table}

\begin{table}[t]
	\centering
	% \def\arraystretch{1.6}
	\begin{tabular}{|c|c|c|c|}
		\hline
		{\multirow{2}{*}{\textbf{Variant}}}  &
		\multicolumn{2}{c|}{\textbf{Constraints}} &
		{\multirow{2}{*}{\textbf{Graph connectivity}}}
		% \multicolumn{2}{c|}{\textbf{Graph connectivity}}
		\\
		\cline{2-3}
		&  \textbf{Rest} & \textbf{Activity} &  % & \textbf{Sampling: Guarantee}
	\\ \hline
	
	A. & \multirow{5}{*}{
		\makecell{ \ref{eq:f0}--\ref{eq:fuse2},\\
			\ref{eq:reach1},\ref{eq:reach2},\\
			\ref{eq:drop1}--\ref{eq:drop4}}
	}
	& \ref{eq:ann},\ref{eq:aen} & No graph     \\\cline{3-4}
	B. & & \ref{eq:anb},\ref{eq:aen} & No graph     \\\cline{3-4}
	C. & & \ref{eq:ann},\ref{eq:aeb} & 3-connected  \\\cline{3-4}
	D. & & \ref{eq:anb},\ref{eq:aeb} & 2-connected  \\\cline{3-4}
	E. & & \ref{eq:ann},\ref{eq:aep} & No graph     \\\cline{3-4}
	F. & & \ref{eq:anb},\ref{eq:aep} & 4-connected  \\\hline
	
	% C_es :Self edges are allowed
	% C_ed : Every edge is distinct 
\end{tabular}
\caption{{\bf Activity regulation of molecules vs. graph connectivity.
}}
\label{tab:smt-grph}
\end{table}

\begin{sidewaysfigure}[t]
	\centering
	\begin{tabular}[t]{|c@{}|@{}c@{}|@{}c@{}|@{}c@{}|@{}c@{}|@{}c@{}|@{}c@{}|@{}c@{}|@{}c@{}|@{}c@{}|@{}C{4cm}@{}|}\hline
		% \begin{table}[t]
		%   \centering
		%   \begin{tabular}[t]{|c@{}|@{}c@{}|@{}c@{}|@{}c@{}|@{}c@{}|@{}c@{}|@{}c@{}|@{}c@{}|@{}c@{}|@{}c@{}|@{}c@{}|}\hline
		{\multirow{2}{*} \textbf{}}  & \multicolumn{2}{c|}{\textbf{Add}} & \multicolumn{2}{c|}{\textbf{Add}} & \multicolumn{2}{c|}{\textbf{Learning NNF}}  &  \multicolumn{2}{c|}{\textbf{Learning}} &  \multicolumn{2}{c|}{\textbf{Add/Delete}} \\
		{\multirow{2}{*} \textbf{Table a}}  & \multicolumn{2}{c|}{\textbf{edge}} & \multicolumn{2}{c|}{\textbf{molecules}} & \multicolumn{2}{c|}{\textbf{(only $\land$ and $\lor$)}}  &  \multicolumn{2}{c|}{\textbf{k-CNF}} &  \multicolumn{2}{c|}{\textbf{parts}} \\
		\cline{2-11}
		{} & {\textbf{Time}} & {\textbf{\#C}} & {\textbf{Time}} & {\textbf{\#C}} & {\textbf{Time}} & {\textbf{\#C}} & {\textbf{Time}} & {\textbf{\#C}} & {\textbf{Time}} & {\textbf{\#C}} \\
		\hline
		
		plos1-dia[3C]& 0.326 &$\infty$& 0.312 &$\infty$& 0.669 & $\infty$ & 0.966 &$\infty$& 0.277 & -1 E, -1 AE, -1 AN. +1 E, +1 N. \\\hline
		plos2-dia[4C] & 0.266 & 0   & 0.322 & 0  & 1.409  & 0 & 2.114 & 0 &  0.337 & 0 \\\hline
		sub-mammal[3C]  & 0.767 & 1 E  & 1.049 & 5 PE & 3.523 & 1E & 4.961 & 1E & 1.172  & -1 E, -2 PE, -1 AN. +1 E, +4 PE, +4 N, +2 AN, +2 AE. \\\hline
		node4[3C]  & 1.554  & 1 E   &  3.859 & 12 PE  &  5.286  & $\infty$ & 4.502 &$\infty$& 2.194  & -2 E, -2 PE, -1 N, -1 AN, -1 AE. +12 N, +8 E, +1 PE.\\\hline
		%   yeast-graph[3C]   & 95.016    & 1 E   & 94.520   & 1 E   & 169.430  & 1 E & 172.35   & 1E   & 107.43  &  -1 E, -2 N, -2 AE, -2 AN. +2 E, 12 PE, 7 N. \\\hline
		yeast-graph[3C]   & 95.016    & 2 E  &   timeout  & N/A   & 1571.42  & 2 E  & 530.210   & 2 E & 72.316  &  -1 E, -1 N, -1 AE, -1 AN, -1PE. +2 E, 7 PE, 8 N. \\\hline
		
		mammal-graph[3C]  &  timeout     & N/A  &  timeout     & N/A    &  timeout         & N/A      &  timeout    &  N/A    &  timeout     & N/A\\\hline
		%    & 0.0    & 0.0    & 0.0    & 0.0    & 0.0         & 0.0      & 0.0   & 0.0    & 0.0    & 0.0\\\hline
	\end{tabular}
	% \caption{Run-times for searching for models (in secs). \#C  stands for minimum changes.
	% Time is reported in seconds.}
	% \label{tab:qf-graph}
	% \end{table}
	\begin{tabular}[t]{|c@{}|@{}c@{}|@{}c@{}|@{}c@{}|@{}c@{}|@{}c@{}|@{}c@{}|@{}c@{}|@{}c@{}|@{}c@{}|@{}C{4cm}@{}|}\hline
		{\multirow{2}{*} \textbf{}}  & \multicolumn{2}{c|}{\textbf{Add}} & \multicolumn{2}{c|}{\textbf{Add}} & \multicolumn{2}{c|}{\textbf{Learning NNF}}  &  \multicolumn{2}{c|}{\textbf{Learning}} &  \multicolumn{2}{c|}{\textbf{Add/Delete}} \\
		{\multirow{2}{*} \textbf{Table b}}  & \multicolumn{2}{c|}{\textbf{edge}} & \multicolumn{2}{c|}{\textbf{molecules}} & \multicolumn{2}{c|}{\textbf{(only $\land$ and $\lor$)}}  &  \multicolumn{2}{c|}{\textbf{k-CNF}} &  \multicolumn{2}{c|}{\textbf{parts}} \\
		\cline{2-11}
		{} & {\textbf{Time}} & {\textbf{\#C}} & {\textbf{Time}} & {\textbf{\#C}} & {\textbf{Time}} & {\textbf{\#C}} & {\textbf{Time}} & {\textbf{\#C}} & {\textbf{Time}} & {\textbf{\#C}} \\
		\hline
		
		plos1-dia & 0.041&$\infty$& 0.320 &$\infty$& 0.225 & $\infty$ & 0.33&$\infty$& 3.74 & -1 E, -1 PE, - 1 N, -1 PE. +1 AE, +1 PE, +1 N\\\hline
		plos2-dia & 3.97 & 0 &  2.647 & 0  & 5.941 & 0 & 5.680 & 0 & 3.56 & 0 \\\hline
		sub-mammal & 3.483 & 1 E  & 4.379 & 5 PE  & 29.980 & 1 E  & 10.405 & 1 E & 3.650  & -1 E, -2 PE, -1 AN. +1 E, +4 PE, +4 N, +2 AN, +2 AE \\\hline
		node4  & 4.150  & 1 E  & 10.562  & 12 PE & 3.401  & $\infty$ & 4.760 &$\infty$&  5.05  & -2 E, -2 PE, -1 N, -1 AN, -1 AE. +12 N, +8 E, +1 PE \\\hline
		yeast-graph & 40.225  & 2 E  &   timeout  & N/A   & 1393.84  & 2 E  & 468.161   & 2 E & 69.81  &  -1 E, -1 N, -1 AE, -1 AN, -1PE. +2 E, 7 PE, 8 N. \\\hline
		mammal-graph   &  timeout     & N/A  &  timeout     & N/A    &  timeout         & N/A      &  timeout    &  N/A   &  timeout     & N/A\\\hline
	\end{tabular}
	\caption{Run-times for synthesis queries. \#C stands for minimum changes in the synthesized VTS in comparison with the given partial VTS. Time is reported in seconds. (a) The solver used is DepQBF (b) The solver used is Z3. The sub-mammal is a subgraph of the complete mammal-graph. In the Add/Delete parts column, ‘+’n sign is used to show the addition of n number of the molecules, similarly ‘-’n is used to show the removal of n number of molecules. In the table, N is node labels, AN is active node molecules, E is edges, PE is molecule presence on the edge and AE is active molecules on the edge. The [kC] stands for k graph connectedness which is part of only DepQBF experiments.}
	
	% \caption{Run-times for searching for Z3 models (in secs). \#C  stands for minimum changes.
	% Time is reported in seconds.}
	\label{tab:qbf-graph}
\end{sidewaysfigure}

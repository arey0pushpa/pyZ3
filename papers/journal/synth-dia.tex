\begin{figure}[t]
	\centering
	%\begin{minipage}{0.45\linewidth
 \begin{subfigure}[b]{0.49\linewidth}
%\begin{tikzpicture}[node distance = 30mm]

%\begin{tikzpicture}[->,>=stealth',shorten >=1pt,auto,node distance=2cm,
%thick,main node/.style={circle,draw,font=\sffamily,minimum size=0.8cm}]
%\begin{scope}[]
%
%\node[main node] (1) {$s_{1}$};
%\node[main node] (2) [below left of=1] {$s_{2}$};
%\node[main node] (3) [below right of=1] {$s_{3}$};
%
%\path[every node/.style={font=\sffamily\small}]
%(3) edge node [left] {} (1)
%(1) edge [loop above] node {} ()
%(2) edge [loop above] node {} ()
%(1) edge node [right] {} (2)
%(2) edge node[right] {} (3)
%(3) edge[bend left, below] node{} (2);
%%(3) edge [bend right] node[right] {} (2)
%\end{scope}
 \begin{tikzpicture}[->,>=stealth', auto , node distance=3.1cm,
	  thick,main node/.style={rectangle,draw}]
	 % 	\begin{scope}[]
	  \node[main node, text width=1.5cm] (pm) {\textbf{PM:} { \^{Qa5} 
	  		                     \^{Qa7} \^{Qbc2} \^{Qbc7}}};
	  \node[main node, text width=1.5cm] (ee) [below right of=pm] {\textbf{EE:} { \^{Qa2} \^{Qb2/3} \^{Qc2/3}}};
	  \node[main node, text width=1.3cm] (le) [above of=ee,yshift=-10mm,xshift=20mm] {\textbf{LE:} { \^{Qa8} \^{Qb8} \^{Qc8}}};
	
	  %le <-> pm
	  \path (le) edge[bend right=55] node [above] {Qb7 Qc7} (pm);
	  %ee -> le
	  \path (ee) edge[bend right=32, text width=2.0cm] node [below,rotate=55] {\^{R8},Qa7, Qbc7,R7} (le);
	
	  %pm <-> ee
	  \path (ee) edge[bend left=32] node [below, rotate=-15] {\^{R3}} (pm);
	  \path (pm) edge[bend left=30, text width=2.0cm] node [above,rotate=-45] {\^{R2} {Qa7} Qbc7, R7, Qc7, Qa2} (ee);
	\end{tikzpicture}
	\caption{Input partial VTS} 
		 \label{fig:synth-vts1}
  \end{subfigure}%
 % \end{scope}
%  \vspace{1.2cm}
  \begin{subfigure}[b]{0.49\linewidth}
	  \begin{tikzpicture}[->,>=stealth',auto,node distance=3.5cm,
	  thick,main node/.style={rectangle,draw}]
	 % \begin{scope}[xshift=3.5cm,yshift=0.0cm]
	  \node[main node,text width=1.5cm] (pm) {\textbf{PM:} {\small \^{Qa5} \^{Qa7} \^{Qbc2} \^{Qbc7 ...}}};
	  \node[main node, text width=1.5cm] (ee) [below right of=pm] {\textbf{EE:} {\small \^{Qa2} \^{Qb2/3} \^{Qc2/3} ... }};
	  \node[main node, text width=1.5cm] (le) [above of=ee,yshift=-10mm,xshift=23mm] {\textbf{LE:} {\small \^{Qa8} \^{Qb8} \^{Qc8 ...}}};
	  
	  %le <-> pm
	  \path (le) edge[bend right=45] node [above] {Qb7 Qc7,{\color{blue} { \textbf{R7}}}} (pm);
	  
	  %le <-> ee
	  {\color{red} { \path (le) edge[bend right] node [below,rotate=50] {\color{blue} { {\textbf{\^{R8},Qa7}}}} (ee);
	  }}
	  %ee -> le
	  \path (ee) edge[bend right] node [text width=1.5cm, below,rotate=50] {\^{R8},Qa7, Qbc7,R7} (le);	  
	  
	  %pm <-> ee
	  \path (ee) edge[bend left=30] node [below, text width=2.0cm] {\^{R3, {\color{blue} { \textbf{\^{Qa2}, \^{Qa7}, \^{R2}}}}}} (pm);
	  \path (pm) edge[bend left=30] node [above,rotate=-45,text width = 2.5cm] {\^{R2} {Qa7} Qbc7, R7, Qc7, Qa2} (ee);	    
	 % \end{scope}

	 \end{tikzpicture}
	 \caption{Output complete VTS} 
	 \label{fig:synth-vts2}
  \end{subfigure}%
  \caption{An example of synthesis of edge and molecules in the partial VTS.} \label{fig:synth-vts}
\end{figure}

  

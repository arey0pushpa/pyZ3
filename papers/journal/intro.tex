A characteristic feature of eukaryotic cells is the presence of multiple internal membrane bound organelles. 
%
These organelles exchange molecules amongst themselves in small membrane bound packets called vesicles. 
%
The recognition of cargo molecules to be loaded onto these vesicles and the recognition of the correct target organelles for each vesicle is performed by an intricate network of molecules called the vesicle transport system []. 
%
We call the set of internal organelles, together with the molecular routes that connect them through vesicle exchange, the vesicle transport network. 
%
Although vesicle mediated traffic was discovered decades ago [17], the picture of the vesicle transport network is far from complete; we do not yet know how many of the cellular proteins reach their resident organelles within the cell, and new vesicle routes are being discovered every year [1, 3]. 

Completing the vesicle traffic network is a difficult task for many reasons: (1) the core vesicle transport network, which consists of a secretory and an endocytic route, is conserved across all eukaryotes, but the traffic network in different organisms [6, 12, 14], and even in different cell types within an organism can be different [10, 11], (2) although the basic traffic machinery is the same for all vesicle fusion events, the details of regulation can be different [15, 16], (3) behaviour of traffic molecules in vitro is different from their behaviour inside cells [9], (4) molecules can have redundant routes within the cell [7,8], and (5) it can be difficult to distinguish between the direct and indirect effects of experiments involving knock-downs or knock outs of traffic molecules [2, 13].

Nonetheless, completing the vesicle traffic network for different organisms is very useful. Many diseases are caused by the malfunction of the vesicle traffic network. Knowledge of the complete network would be helpful in identifying the root causes.[4,5]. At a more basic level, having complete pictures of the vesicle traffic system for various organisms and various cell-types would allow comparative studies, and therefore would allow the deduction of modes of evolution of the form of the traffic network, and would also allow us to decipher features such as parts of the traffic system that are unchanging, and therefore likely to be its core, and parts that are more plastic [14].

In this paper, using constraints on global vesicle traffic network topology due to local molecular interactions, we take incomplete pictures of vesicle traffic networks as inputs and output various completed versions which can then be tested experimentally.

%%% Local Variables:
%%% mode: latex
%%% TeX-master: "main"
%%% End:


\noindent 
%A characteristic feature of eukaryotic cells is the presence of multiple internal membrane-bounded compartments. 
Eukaryotic cells contain distinct membrane-bound objects known as organelles: endoplasmic reticulum (ER), Golgi apparatus, vacuoles, and lysosomes are examples of these.
%These compartments exchange molecules amo\-ngst themselves in small membrane-bounded packets called vesicles~\cite{alberts2002molecular}. 
%
%%The molecular interaction network that controls the vesicle traffic system involves dozens of varieties of molecules that regulate the flow of cargo. 
%
%%Due to this, obtaining a complete understanding of the VTS through physical experimentation alone is a hard problem.
%
%%Highly reduced models of vesicle traffic have previously provided insight into how the chemical composition of various cellular compartments can emerge out of molecular regulation and cargo flow. 
%
%%However, the high complexity of problems in VTS restricts the use of traditional methods like simulation and statistical reasoning~\cite{mani2016stacking}. 
%The high complexity of problem restricts the use of traditional methods like simulation and statistical reasoning and the requirement of precise analysis makes formal methods an ideal candidate. 
%%In this paper, using formal methods, we solve the problem of completing incomplete VTS graphs while also considering constraints due to molecular interactions, thereby demonstrating the effectiveness of such methods.
%
%
Some of the material of the cell needs to move
between these organelles in small membrane-bound
transporters known as vesicles \cite{stenmark2009rab}.
%
This organization resembles a transport logistics network
and is commonly known as the membrane traffic system.
%
Membrane traffic underpins nearly every aspect of eukaryotic cellular
physiology, including human cellular physiology.
%
Defects in membrane traffic lead to a variety of disorders or even
cell death \cite{stenmark2009rab}.
%
Understanding how membrane traffic functions are, therefore, one of the
central challenges of cell biology.
%

The cell biologists have made some progress by assembling a
“parts list” of molecules that drive vesicle
traffic~\cite{dacks2007evolution}.
%
The essential processes involve the creation of vesicles loaded with
cargo from source organelles, known as “budding”, and the depositing
of those vesicles and cargo into target organelles, known as “fusion”
\cite{munro2004organelle}.
%
Proteins that are called coats and the adaptors that select the cargo regulate budding.
%
The proteins that are known as SNAREs and tethers that ensure that vesicles
fuse to the correct target regulate fusion
\cite{mani2016stacking}.
%
Although we know a great deal about molecular-level details, there has
been very little work done on how such processes are integrated across
scales to build the entire vesicle traffic system (VTS).
%
We have recently attempted to use the ideas of graph theory to address
this question \cite{mani2016stacking,shukla2017discovering}.

The SNARE proteins and the constraints that SNARE-to-SNARE regulations
impose on the feasible behaviors of VTSs are the focus of
our current work.
%
Abstractly, we can think of these proteins as labels that are
collected from the source compartment and taken to the target
compartment on vesicles.
%
There is a corresponding set of label molecules on the target compartments.
%
If the two sets of labels (on vesicles and targets) are compatible,
then the vesicle will fuse.
%
The question is: does this physical picture place any constraints on
the global topology of the traffic graph? We have previously shown that one informative constraint is graph connectivity \cite{shukla2017discovering}.
%
The intricacies of SNARE regulation are largely unknown.
%
How stringent SNARE regulation is can determine whether the graph
topologies of a particular level of connectivity are biologically
realizable.
%
For example, if SNAREs are completely unregulated and thus always
active, they cannot be used to generate a traffic network.
%
If relatively simple rules regulate the SNAREs, then only
highly connected traffic networks can be generated using them.
%
The question is, can we work toward a predictive analytical framework
of VTSs?

Another interesting question is the completion of the VTS \cite{synthesisGupta}. 
%
The current picture of the VTS is far from complete; we do not yet know how many of the cellular proteins reach their resident compartments within the cell, and
researchers are discovering new vesicle routes every
year~\cite{nickel2018unconventional,weill2018toolbox}. 
%
Completing the VTS is a difficult task for many important reasons: 
\begin{enumerate}
	\item The core VTS is similar across eukaryotes, the traffic network in different organisms \cite{richardson2015evolutionary, barlow2017seeing}, and even in different cell types within an organism can be different~\cite{stoops2014trafficking,zhou2015arp2}.
	\item Although the basic molecular machinery is the same for all vesicle fusion events, the details of regulation can be different~\cite{davletov2007regulation,di2010calcium}.
	\item Molecules can have redundant routes within the cell~\cite{shimizu2014compensatory,nakatsukasa2014nutrient}.
	\item The behavior of traffic molecules in vitro is different from their behavior inside cells~\cite{furukawa2014multiple}. 
	\item It can be difficult to distinguish between the direct and indirect effects of experiments involving knock-downs or knockouts of traffic molecules \cite{hirst2004epsinr,mishev2013small}.
        \item Lastly, interpreting experimental results can be difficult, since the behavior of traffic molecules in vitro is different from their behavior inside cells.
\end{enumerate}

Nonetheless, completing the VTS for different organisms is very useful. 
%
The malfunction of the VTS causes many diseases.
%
The knowledge of the complete network would help identify the root causes of
the diseases~\cite{bexiga2013human,gissen2007cargos}. 
%
At a more basic level, having complete pictures of the VTS for various organisms and various cell-types would allow comparative studies, and therefore would allow the deduction of modes of evolution of the form of the traffic network. 
%
The network knowledge would also allow us to decipher features such as parts of the traffic system that is unchanging and therefore likely to be its core, and more plastic parts~\cite{barlow2017seeing}.

%\subsection{Synthesis of VTS:}
%
The underlying cell biology dictates the set of basic constraints the VTS should follow.
%
In order to attempt to complete a given partial VTS, we have to first agree on the properties that every complete VTS should follow in spite of the basic constraints. 
%
We use our graph connectivity results as a basis for this investigation.   
%
The incomplete VTS might not respect these constraints. 
%
In this paper, using constraints on global vesicle traffic network topology due to local molecular interactions, we take incomplete pictures of vesicle traffic networks as inputs.
%
We output various completed versions of VTS against these properties, which we can then test experimentally.
%

%\subsection{Analysis is difficult problem}
The analysis of VTSs is a difficult problem because
of the combinatorial scaling of possible traffic topologies and
regulatory rules.
%
An analysis of a large number of graphs~\footnote{There are total 68715 simple (no parallel edges) 3-connected graphs with 8 nodes} jointly with combinatorially many possible regulatory rules (all compatible labels of edges and nodes etc.) will be difficult to handle by any tool.
%
The same reason would hinder the precise analysis of the properties of the VTS. 

A previous work used a sampling approach. 
%
The study used a rule-based approach~\cite{mani2016stacking}.
Given an initial condition of the cell and rules for molecular traffic,
the approach updates the state of each compartment in the cell until the cell reaches a steady-state. 
%
With this study, statistical claims can be made, but improbable albeit biologically possible solutions could have been missed.
%
Also, there are many ways in which the dynamics of vesicle transport can be played out, while a single one was explored in this work. 
%
It is possible that a different scheme for the dynamics could lead to a different set of solutions.

%Previous analyses of vesicle traffic networks dealt with this combinatorial
%explosion by using a sampling approach~\cite{mani2016stacking}. 
%
%In these analyses, vesicle traffic is modelled as an update system. 
%
%The traffic rule specifies how the system transitions from one
%time point to the next. 
%
%Given a traffic rule and an initial condition, the system updates molecular flux of the compartments one step a time choosing one among the possibly exponential choices of the rule.
%
%The update rule defines the flux transition for the next step. 
%
%This process is performed until a steady state is reached. 
%
%
%By studying a large sample of randomly generated update rules, such analyses can make statistical claims about vesicle traffic. 
%

In contrast, here we seek to understand the properties of vesicle traffic networks
over all possible update rules, not just for a sampled subset.
%
Formal verification tools like model checkers, SAT (Boolean satisfiability), SMT (satisfiability modulo theories) and QBF (quantified Boolean formula) solvers serve precisely this purpose. 
%
%Formal verification tools, such as model checkers
%\cite{clarke1996symbolic, biere2003bounded, clarke2008birth,
%	cimatti2000nusmv, holzmann1997model} and SAT solvers
%\cite{moskewicz2001chaff,een2004extensible} serve precisely this
%purpose.
%%
The tools model the computation symbolically for the graphs of finite size and can predict the hypothesis for all possible states of the system without explicit enumeration of rules. 
%
In recent work \cite{shukla2017discovering}, we employed the model checker CBMC \cite{ckl2004} meant to analyze ANSI C programs for studying this problem. 
%
VTSs and SNARE regulations were modeled as C code using arrays to represent graphs and Boolean tables. 
%
The graph connectivity question for different variations of SNARE regulations was modeled as logical assertions to be checked by the model checker. 
%
Although the performance was satisfactory, the encoding was non-optimal and hence scalability of the method was insufficient. The tool was able to handle the system with just eight compartments.
%
Another interesting question of property involving quantifiers (was termed as sufficient condition i
n \cite{shukla2017discovering}; in this paper, we call it universal condition) was partly handled
using sampling and left open.
%
In this paper, we present both the SMT and QBF encodings to address the two problems.
%


%
%These tools are central to formal verification methods in computer science, and are increasingly being applied to understand biological systems, including gene regulatory networks [16-18]. 
%\subsection{Previous work and our Contribution}
%
%The following are the contributions of this work:
%\begin{itemize}
%	\item We have added novel encoding reachability and 3-4 connectivity
%	\item Direct encoding into the SAT solver
%	\item A user friendly and scalable tool based on well known SMT solver Z3
%\end{itemize}


\par We have implemented the encodings in a flexible tool, which can handle both connectivity questions and a wide range of synthesis queries. 
%
We have applied our synthesis tool on several VTSs, including
two found-in-nature VTSs.
%
We manage to scale the tool to the size that makes the tool biologically relevant. 
%
Our experiments suggest that some of the synthesis problems are solvable by modern solvers and the synthesis technology may be useful for biological research.
%

%
We organized the rest of the paper as follows. 
%
In Section~\ref{sec:bio}, we present the biological background and details to understand the VTS. 
%
In Section~\ref{sec:prelim}, we present the notations we use in this paper. 
%
In Section~\ref{sec:model}, we present a formal model of VTSs, and in section~\ref{sec:problem}, we present the property of interest with their formal problem statements of different search problems of VTS.
%
In Section~\ref{sec:encoding}, we present the encodings of corresponding search problems. 
%
In Section~\ref{sec:experiments}, we present our implementation and experimental results. 
%
In Section~\ref{sec:related}, we discuss related work and conclude in Section~\ref{sec:conclusion}.
%
%%% Local Variables:
%%% mode: latex
%%% TeX-master: "main"
%%% End:
~        

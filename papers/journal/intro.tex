%\textbf{Components of the vesicle transport network}:
A characteristic feature of eukaryotic cells is the presence of multiple internal membrane-bound organelles called compartments.
% 
These compartments exchange molecules amongst themselves in small membrane-bound packets called vesicles. 
%
The recognition of cargo molecules to be loaded onto these vesicles and the recognition of the correct target compartments for each vesicle is performed by an intricate network of molecules called the vesicle transport system~\cite{alberts2002molecular}. 
%
We call the set of internal compartments, together with the molecular routes that connect them through vesicle exchange, the vesicle transport network. 
%
Vesicle traffic underpins nearly every aspect of eukaryotic cellular physiology, including
human cellular physiology.
%
Understanding how this system function is, therefore,
one of the central challenges of cell biology.
%
The high complexity of the problems~\cite{mani2016wine, mani2016stacking} in VTS restricts the use of traditional methods like simulation and statistical reasoning. 
%The high complexity of problem restricts the use of traditional methods like simulation and statistical reasoning and the requirement of precise analysis makes formal methods an ideal candidate. 
The requirement of precise analysis makes formal methods ideal candidate in this case. 

First, we will give an overview of the VTS mechanism and then will sketch out a few interesting problems that are key to the understanding of the VTS.

\subsection{Vesicle traffic system}
\todo{SA:One para of BIO BACKGROUND related to VTS. Why they are Important history etc.}
%
%\textbf{Components of the vesicle transport network}:
The vesicle transport network is used by cells to transport molecules across compartments.
% 
The mechanism of vesicle transport can roughly be broken down into two big steps:
% budding of vesicles from the source compartment and their fusion with the target compartment.
%
the process of formation of a vesicle by budding from the membrane called as vesicle budding and the fusion of these transport vesicle with its target compartment called as vesicle fusion. 

\todo{SA:Need a Diagram}Vesicle budding process involves series of steps, sorting and packaging of the molecules, membrane deformation and budding out from the compartment.
%
First the sorting of cargo molecules takes place by cytosolic adaptor molecules. 
%
Cargo adaptors bind to short linear stretches of amino acids in the cargo molecules called sorting signals and thereby sequester them for packaging. 
%
Then the different adaptor proteins act at different compartments, for eg., sec23/24 is required for packaging vesicles that form at the ER, the AP2 adaptor is used for packaging vesicles at the plasma membrane, etc.
%
Thereafter, adaptors are recruited to the source compartments through interactions with the lipids that form the source compartment's membrane, and interactions with Arf/Sar GTPases which signal the initiation of vesicle formation~\cite{paczkowski2015cargo}.

The next step is membrane deformation, which involves cytosolic coat proteins.
Coat proteins are recruited to the forming vesicle through their interaction with adaptor proteins. 
%
These are naturally curved proteins, and their assembly on membranes leads to membrane deformation.
%
A single kind of coat protein can bind to multiple kinds of adaptors, thereby allowing the packaging of multiple types of cargo within the same vesicle. 
%
Clathrin, COP2 and COP1 are the three major kinds of coat proteins in eukaryotic cells. 
%
The clathrin coat operates at multiple compartments, and can interact with multiple adaptors proteins, whereas the COP2 and COP1 coat are only involved in vesicle formation at a single compartment; the ER and the Golgi respectively~\cite{faini2013vesicle}.
	
The final step is budding of the vesicles from the source compartment.
% by dynamin.
%	The final step in vesicle formation is its pinching off from the source compartment.
	%
This process requires ATP and is performed by the cytosolic dynamin proteins, which are recruited to the vesicle neck by the coat proteins~\cite{cocucci2014dynamin}.
%
The transport vesicle is now ready for the transportation.

Vesicle fusion involves two types of events, recognition of the correct target compartment and fusion of the transport vesicle with it's target compartment.

	 \textbf{Recognition of the correct target compartment} is a complex process involving Rab
	GTPases and tethering complexes.
	% 
	 Rab GTPases are small proteins which occur in two forms: the membrane associated and active
		GTP bound form, and the cytosolic and inactive GDP bound forms. 
		%
		Different compartments of the
		cell are associated with a different type of Rab protein. In their active form, Rabs recruit many
		downstream proteins, including tethering complexes, fusion regulators, motor proteins, sorting
		adaptors, etc, to the membrane they are associated with. 
		%
		The proteins that are recruited by active
		Rabs are called Rab effectors.
		%
		Rab activity is controlled by two proteins: GEFs turn Rabs on and
		GAPs turn them off. 
		%
		Each Rab has its own specific GEF and GAP. In some cases, GEFs and GAPs
		themselves can be Rab effectors, thus generating feed-forward and feedback loops. 
		%
		In their cytosolic, inactive form, Rabs are complexed with GDI proteins, and it is in this form that Rabs are presented to membrane bound GEFs for activation~\cite{muller2018molecular}
		
	 \textbf{Tethering complexes} sequester vesicles to their target compartments and regulate SNARE
		proteins, which enable vesicle fusion. 
		%
		Tethers are believed to bind to SNAREs and function as
		chaperones for SNARE complex assembly which is the ultimate step in membrane fusion. 
		%
		There are two types of tethering complexes: multi subunit tethers and long coiled-coil tethers:
%		
 Coiled coil tethers are long, single subunit proteins. 
			%
			These tethers are believed to bind the
			vesicle on one end and the target compartment on the other end, thereby bridging the two before
			fusion.
			%
			A clear mechanism for action of these tethers is missing, but based on their interactions with
			other proteins of the traffic machinery, some hypotheses as to their mechanism of action have been
			put forth. 
			%
			Golgins, coiled coil tethers that are anchored at the Golgi by a transmembrane domain, contain a large number of binding sites for Rab GTPases. 
			%
			Thus, golgins might capture vesicles by
			binding vesicle-associated Rab proteins. 
			%
			Golgins also contain domains which can sense membrane
			curvature, which they could be using to recognize vesicles. 
			%
			They might also be interacting directly
			with vesicle SNAREs~\cite{baker2016chaperoning}.
		%	
			Multi-subunit tethering complexes (MTCs) are composed of three or more different
			subunits.
			%
			MTCs are known to interact with vesicle coat proteins, Rab GTPases, SNAREs and SM proteins.
			%
			For example, the HOPS complex, which is required for homotypic (fusion of two identical
			membranes) and heterotypic fusion in the endo-lysosomal system, was shown to tether membranes
			through its interactions with the membrane-associated Rab GTPase Ypt7, acidic phospholipids and
			SNAREs. 
			%
			The HOPS complex binds both individual SNAREs and SNARE complexes. It also
			seems to protect assembling trans-SNARE complexes from premature disassembly. 
			%
			Another example is given by the Dsl1 complex, which is anchored to the ER membrane through interactions
			with t-SNAREs. 
			%
			At the other end, the Dsl1 complex contains multiple binding sites for COPI
			(vesicle coat that is present on vesicles produced by the Golgi). 
			%
			This structure suggests that the Dsl1
			complex functions as a tether connecting COPI-coated vesicles to their target organelle, the ER~\cite{baker2016chaperoning}.
			

	
	 \textbf{Fusion of vesicle with the target compartment} is brought about by SNARE proteins. 
	%
	SNAREs
	are defined by a 60- to 70-residue SNARE motif. Most SNAREs are anchored to the membrane by
	their C-terminal transmembrane helices. 
	%
	The formation of a productive SNARE complex requires
	the formation of a four-helix bundle containing four different SNARE motifs. Usually, one of these
	motifs is contributed by the SNARE on the vesicle membrane (v-SNAREs), and the other three
	motifs by SNAREs on the membrane of the tartget compartment (t-SNAREs). 
	%
	Alternatively, they
	can be defined by their amino acid sequences as R-, Qa-, Qb-, and Qc-SNAREs. A few SNARE
	proteins, such as SNAP25, contain both Qb- and Qc-SNARE motifs.
	%
	Most v-SNAREs are RSNAREs, and most t-SNAREs are Q-SNAREs~\cite{yoon2018snare}.
	
	


Many SNAREs also have N-terminal domains that regulate SNARE complex assembly and/or
interact with other parts of the vesicle fusion machinery. 
%
Qa-SNAREs have an amino terminal
domain called the Habc domain, which can fold back on the SNARE domain and hold the QaSNARE
in an inactive state. 
%
This inactivated state is stabilized by interactions with SM proteins~\cite{yoon2018snare}.	      

A subset of R-SNAREs (the longin SNAREs) possess an amino-terminal domain called the longin domain. 
%
In addition to regulation of SNARE activity as in the case of the Qa-SNARE Habcdomains,
these longin domains also regulate the localization of longin SNAREs through their
interaction with adaptor proteins~\cite{daste2015structure}.

Different vesicle-target compartment pairs in the cell are associated with unique SNARE
complexes.
%
Membrane fusion converts the trans-SNARE complex (v- and t- SNAREs on opposite
membranes) into a cis-SNARE complex (both v- and t- SNAREs on the same membrane).
Disassembly of cis-SNARE complexes requires Sec17 and Sec18 (the yeast SNAP and NSF
protein, respectively). 
%
This process requires energy which is released by the hydrolysis of ATP.

\textbf{SNAREs are regulated by SM proteins}, which have three modes of action: SM proteins can hold
the Qa-SNAREs in an autoinhibited state and prevent or postpone their assembly into SNARE
complexes.
%
Secondly, some SM proteins have been seen to act as a template upon which a halfzippered
complex between the Qa- and R-SNAREs (an early SNARE complex intermediate) can
form. 
%
In this mode, by choosing an R-SNARE located on the opposite membrane, SM proteins
could be imposing a filter inhibiting the formation of futile cis-SNARE complexes and promoting
trans-SNARE complexes.
%
Finally, SM proteins bind the four-helix bundles formed by assembled
SNARE complexes.
%
This mode of binding of SM proteins might protect assembled SNARE
complexes from premature disassembly by NSF and SNAPs and/or might stimulate fusion directly~\cite{baker2016chaperoning}.
%
SM proteins have also been shown to interact with tethers~\cite{yoon2018snare}.

\textbf{Major paths in the vesicle transport network}:
Molecules traverse the cell in a series of vesicles, and there are two major such routes of transport in all eukaryotic cells.
%
	The secretory route takes proteins from the ER, their site of production, to the plasma membrane, from where they are secreted out of the cell.
	%
	 Cargo proteins leave the ER in COP2 coated vesicles which fuse with the Golgi apparatus.
	 %
	  The golgi apparatus is where proteins get modified, for example by the addition of carbohydrate side-chains. 
	  %
	  Subsequently, proteins destined for secretion are packaged into clathrin coated vesicles which fuse with the plasma membrane, thus releasing their contents to the outside of the cell\cite{alberts2002molecular}.
	  %
 The endocytic routetakes proteins from the outside of the cell through the plasma membrane to the endocytic compartments, where they are digested. 
 %
 Cargo from the outside of the cell is taken in in the form of clathrin coated vesicles which form at the plasma membrane. 
 %
 These vesicles fuse amongst themselves in a process called homotypic fusion to generate the early endosomal compartments. 
 %
 Early endosomes undergo changes in their composition due to ongoing vesicle traffic and maturee into late endosomes. 
 %
 This maturation is concomitant with a switch the Rab protein they are associated with; the Rab5 associated early endosomes mature into Rab7 associated late endosomes\cite{rink2005rab}. Late endosomes then fuse with lysosomes, where all its contents get digested\cite{pryor2009delivery}.
%
Other paths are used for cross talk between the secretory and the endocytic routes. For example, vesicles are sent from the TGN (Trans-Golgi network) to the early and the late endosomes to transport enzymes, and vesicles are sent back from the endocytic compartments to the TGN to recycle sorting receptors\cite{progida2016bidirectional}. Also recently, evidence for unconventional vesicle-mediated secretory routes have been found, which bypass the Golgi. The molecules involved in these routes are as yet unknown\cite{nickel2018unconventional}. 
%

%\begin{table}
%	\begin{center}
%		\begin{tabular}{|c|c|c|c|c|c|c|c|c|c|c|}
%			\hline
%			Nodes & 1 & 2 & 3 & 4 & 5 & 6 & 7 & 8 & 9 & 10 \\ \hline
%			Graphs & 0 & 0 & 0 & 1 & 2 & 15 & 121 & 2159 & 68715 & 3952378 \\ 
%			\hline
%		\end{tabular}
%		\label{tab:threec}
%		\caption{Number of simple 3-edge-connected unlabelled N-node graphs.}
%	\end{center}	
%\end{table}
\subsection{Abstraction of VTS as a graph problem}

%In this section, we will describe the basic constraints imposed by cell biology. These are all incorporated into an abstract model of a VTS, whose properties will then be explored using SMT solvers.

In this section, we will abstract from the biological description of the VTS and represent the whole network as an annotated transport graph. 
%
The constraints imposed by cell biology is incorporated in the  annotations of the graph. 

\textbf{The cell as a transport graph:} 
%We consider a cell to be a collection of compartments (nodes) and vesicles (edges), thus defining a transport graph. Every compartment or vesicle has a set of molecular labels, such as SNARE proteins, associated with it.
The cell can be abstractly represented as an annotated graph. 
Every compartment in the cell can be represented as a node in the graph. 
%
The set of molecules present in the compartment, for example, SNARE proteins associated with it, are represented as a label to the corresponding nodes.
%
The label of each node will be unique molecules present on the compartment, i.e we abstract over the quantity of the molecule present in a compartment.
%This represents the flux of the molecule type present at that compartment.
%
The target vesicle flowing from source compartment to the target compartment is represented by a labeled directed edge. 
% 
The label represents the associated flux of all molecular types carried by the corresponding vesicle.
%
%

\textbf{Steady state:} 
The vesicle transfer will change the molecular composition (distinct molecule count) on both the source compartment and the target compartment. 
%
In our abstracted model, we will assume that cell is in a steady state. 
%
The compartment's composition does not evolve over the short times' scales.
%
Therefore, the incoming and outgoing flux is balanced for each of the molecular types at each compartment.
%
This abstraction is supported by the cell biology as on the scale of seconds and minutes the molecular composition actually remains the same. 
%
In this paper, we refer to the steady state of the cell as stability condition over the annotated graph.


%Each edge is associated with a flux of all the molecular types carried by the corresponding vesicle. The total amount of each molecular type on each compartment can therefore increase or decrease. We assume the cell is in a steady state where each compartment’s composition does not vary over short time scales. Therefore, incoming and outgoing fluxes are balanced for each molecular type at each compartment; it is the \textit{stability cond1ition}.

\textbf{Vesicle fusion:}
%Based on the earlier discussion of fusion, the vesicle targeting is driven by molecular interactions.  
%
%Particularly, molecules composition (SNARE proteins etc) present on the budded vesicle determines it's properties, which 
% 
%The molecular composition and hence the properties of the transfer vesicle 
%is the crucial factor to which target compartment the vesicle will fuse to.
%
%For any given pair of a vesicle and a compartment, the SNARE proteins present on both influence the fusion of the vesicle to that compartment.
%  
%Biophysically, fusion of a vesicle to the target compartment requires a direct physical interaction between at least one SNARE type on the vesicle and one SNARE type on the compartment.
%
%Once a vesicle has budded out of the source, the molecules it carries determine its properties. In particular, for any given pair of a vesicle and a compartment, the set of SNARE proteins that label the former and latter influence whether the vesicle will fuse to that compartment. Biophysically, fusion requires a direct physical interaction between at least one SNARE type on the vesicle and one SNARE type on the compartment. 
%
%SNAREs are of two types (known as Q and R in the cell biology iterature) and 
Vesicle fusion requires a pairing of a Q-SNARE with an R-SNARE between transfer vesicle and target compartment. 
%For any given pair of a vesicle and a compartment, vesicle fusion requires a pairing of a Q-SNARE with an R-SNARE on 
%the SNARE proteins present on both influence the fusion of the vesicle to that compartment.
%
%The list of molecular pairs that can drive a fusion event is given in a pairing matrix between Q and R SNAREs. 
%
Only a particular pair of Q and R SNAREs are allowed to fuse with each other.
%
Without loss of generality, we assume equal numbers of Q and R SNARE types.
%
Therefore, we can abstract the underlying cell biology by labeling the nodes and edges with an equal number of Q and R SNAREs. 
% equal number of Q and R SNAREs as a part of node and edge label.
%
%
Given a relation of all allowed fusion SNARE pairings, we can computationally determine whether a particular transfer vesicle will fuse to a particular target compartment (the edge between two compartments) based on the above condition.  

\textbf{Molecular regulation:} 
Fusion takes place only if the SNARE types involved in the vesicle and compartment must both be in an active state.
%We assume that for fusion to occur, the pair of 
%
The activity of these SNAREs is dependent on the presence of other molecules on the vesicle or compartment, respectively.
%
In our abstract model, we create a set of variations of different kind of molecular regulations.
%
Most generally, the activity state of a given SNARE can be a Boolean function of all the molecular types on a compartment or vesicle. 
%
We have also tested \cite{shukla} a particularly simple regulation mechanism in which two SNAREs that can pair to drive fusion to inhibit one another; this is the \textit{pairing inhibition}. 
%
This is motivated by the idea that pairing must generate an inactive bi-molecular complex.
%

%We assume that for fusion to occur, the pair of SNARE types involved on the vesicle and compartment must both be in an active state. Whether these SNAREs are active or inactive depends on the other molecules found on the vesicle or compartment, respectively. We test many different versions of this kind of molecular regulation. Most generally, the activity state of a given SNARE can be a Boolean function of all the molecular types on a compartment or vesicle. We have also tested \cite{shukla} a particularly simple regulation mechanism in which two SNAREs that can pair to drive fusion inhibit one another; this is the \textit{pairing inhibition}. This is motivated by the idea that pairing must generate an inactive bi-molecular complex.

\textbf{Difficulty of the analysis:}
The analysis of vesicle traffic systems is a difficult problem
because of the combinatorial scaling of possible traffic topologies and regulatory rules. 
%
For example, we might want to check some conjecture of interest for all graphs of a certain structure. 
%
The number of graphs with that specific structure with a combination of the possible regulatory rule will be large to handle by any tool.
%
Precise analysis of the properties of the VTS would be hindered by the same reason. 
%
%Properties of the VTS would be hindered by the same reason. 
%3-connected graphs and all possible variations of SNARE regulation rules. 
%
%The number of graphs of specified connectivity grows exponentially with the number of nodes: Table~\ref{tab:threec} shows how many 3-edge-connected graphs~\cite{a052448-oeis} exist (without parallel or self edges) as node number N increases.

\subsection{Interesting properties of VTS}
%\textbf{Properties of a VTS that satisfies all cell-biological constraints:} 
Suppose we are given a particular transport graph, a particular labeling of all the compartments and edges, a particular fusion pairing matrix, and a particular regulatory model. This information is then sufficient to check the following properties, which summarise the cell-biological constraints described above:
\begin{enumerate}
	\item We can determine which molecules are active on every compartment or vesicle.
	\item For every vesicle fusing to a compartment, we can determine whether there exists an active pair (one molecule on the vesicle, one on the compartment) which drives that fusion event.
	\item For every vesicle-compartment pair where the vesicle does not fuse to the compartment, we can verify that there is no pairing of active molecules on the
	vesicle and compartment that could drive their fusion.
	\item We can verify that every molecular type entering a compartment also leaves the compartment, and also that every molecular type entering a set of compartments also leaves that set. This is the steady state condition. In the biological literature this is often referred to as “homeostasis” and is a widespread and well-accepted assumption about cellular behaviour, at least over timescales of minutes to hours \cite{mani2016stacking}.
\end{enumerate}

The biological problem often boils down to such an analysis. A cell biologist might determine which molecules flow between which set of compartments, and biochemical experiments could be used to see how these molecules activate one another. We can then ask: is this a complete and consistent description? That is, do all the required properties listed above hold, given what the experimentalists have told us? It is often the case that biological data is missing. For example, only a few of the dozens of molecules involved in real VTSs have been mapped out. Therefore, it is extremely likely that the description provided by the cell biologist is incomplete. We can use our model to find out which properties have failed to hold, and thus prescribe new experiments in order to fill in the missing information.

Can we find a simple test to see whether any information is missing, given the experimental data? We have shown that graph k-connectedness furnishes precisely such a test \cite{shukla}. If the data provided by cell-biologists, suitably represented as a graph, does not have a certain degree of connectivity, this implies that some biological data has been missed. (The converse is not true: even if the required connectivity does hold, there might of course be more information missing.)

Our result about k-connectedness being a useful test of missing information \cite{shukla} was obtained using SAT solvers for graphs up to a certain size, and a certain number of molecules. This was due to limitations in how we encoded the problem. Here we present a much more natural encoding that allows our result to be extended to graph sizes and molecule numbers that are typical of those found in real cells.


%\subsection{Modeling and Symbolic Analysis of VTS: An Overview}
\label{subsec:graphmodel}
%
Since a VTS is a transport graph, it is but natural to formally model
VTS as graphs (as used in computer science) with their nodes denoting
compartments and labeled edges denoting transport vesicles with labels
denoting the set of molecules being transported. The pairing mechanism
can be represented as matching tables over sets of molecules.
% i.e., formally as a boolean function that given requisite labels of nodes and vesicles that returns true if the required regulations are met. 
%
Given such a graph model of VTSs and their properties, such as
stability condition and fusion rules, can be formally defined as
constraints over graphs and uninterpreted Boolean functions.
%
% Note that the formulas define among other things constraints over
% paths between nodes in the graph model of VTS. Similarly, one can
% define fusion rules as constraints over the boolean functions modeling
% the regulations.  
\
% For example, the steady state condition, described informally earlier, can be defined by the formula shown in Listing~1.1.

%\begin{itemize}
%\item \srivas{Detailed BIO to Graph problem}
%\end{itemize}


%Given definitions of fusion and budding rules and the steady state conditions, whether a VTS meets maximum connectivity requirement, i.e. the LGC property, can be verified by checking if the formula show in Listing~1.2 is valid.  Here we have defined the property by checking over existence of any fusion/budding rules.  We can eliminate the existential quantifier if we are interested in checking the property for a particular fusion/budding rule.
%
%\subsection{Converting the Graph Problem into Boolean SAT problem}
%\label{subsec:satproblem}
%To convert the graph problem described in sec~\ref{subsec:graphmodeling}, into a boolean SAT problem, we need to define a scheme to represent graphs and boolean functions using a suitable set of propositional variables.  In our earlier work, we modeled the graph problem in C using arrays to model graphs and boolean function.  We then used the CBMC model checker to convert the graph problem into a SAT problem.  One of the challenges we had in our earlier work is dealing with quantifiers.  Note that the connectedness property defined in Listing~1.2 has quantifier alternation.  Even if we were to eliminate the existential quantifier by instantiating the problem for a fixed set of fusion rules, we would have embedded quantifiers in the antecedent of implications.   CBMC supports only a quantifier-free logic n its assertion language.  In our earlier work we used a combination of explicit enumeration at the C-level and clever use f non-determinism to eliminate alternation of quantifiers.  This enumeration was one source of bottleneck for scaling in our earlier work.  In the current work we eliminate this bottleneck by modeling the problem directly as a SAT instance using uninterpreted functions and recursive relations.  The details of te hnew encding will described in subsequent sections.

%A VTS is {\em $k$-connected} if every pair of compartments remain reachable after dropping $k-1$ vesicles.
%
%This property of VTSs has been considered informative and studied by~\cite{shukla}.
%
%Here we have build an {\em efficient} tool that studies properties of VTSs that are not $k$-connected, from some $k$. 

\subsection{Synthesis of VTS:}
Although vesicle-mediated traffic was discovered decades ago~\cite{wells2005discovery}, the picture of the vesicle transport network is far from complete; we do not yet know how many of the cellular proteins reach their resident organelles within the cell, and new vesicle routes are being discovered every year ~\cite{nickel2018unconventional,weill2018toolbox}. 
%
Completing the vesicle traffic network is a difficult task for the following reasons: 
\begin{enumerate}
	\item The core vesicle transport network, which consists of a secretory and an endocytic route, is conserved across all eukaryotes, but the traffic network in different organisms \cite{richardson2015evolutionary,nishimoto2009differential,barlow2017seeing}, and even in different cell types within an organism can be different ~\cite{stoops2014trafficking,zhou2015arp2}.
	
	\item Although the basic traffic machinery is the same for all vesicle fusion events, the details of regulation can be different~\cite{davletov2007regulation,di2010calcium}.
	
	\item Behaviour of traffic molecules in vitro is different from their behaviour inside cells ~\cite{furukawa2014multiple}.
	
	\item Molecules can have redundant routes within the cell ~\cite{shimizu2014compensatory,nakatsukasa2014nutrient}.
	
	\item It can be difficult to distinguish between the direct and indirect effects of experiments involving knock-downs or knock outs of traffic molecules \cite{hirst2004epsinr,mishev2013small}.
\end{enumerate}

Nonetheless, completing the vesicle traffic network for different organisms is very useful. 
%
Many diseases are caused by the malfunction of the vesicle traffic network. 
%
Knowledge of the complete network would be helpful in identifying the root causes~\cite{bexiga2013human,gissen2007cargos}. 
%
At a more basic level, having complete pictures of the vesicle traffic system for various organisms and various cell-types would allow comparative studies, and therefore would allow the deduction of modes of evolution of the form of the traffic network, and would also allow us to decipher features such as parts of the traffic system that is unchanging, and therefore likely to be its core, and parts that are more plastic \cite{barlow2017seeing}.

%
The set of basic constraints the VTS should follow is dictated by underlying cell biology.   
%
In order to even attempt to complete a given partial VTS we have to agree on the properties that every complete VTS should follow in spite of basic constraints. 
%
We can use our stability and k-connectivity constraints as a basis for this investigation.   
%
The incomplete VTS might not respect these constraints. 
%
In this paper, using constraints on global vesicle traffic network topology due to local molecular interactions, we take incomplete pictures of vesicle traffic networks as inputs.
%
We output various completed versions of VTS against these properties which can then be tested experimentally.
%

We have implemented the encoding in a flexible tool, which can handle a wide range of synthesis queries. 
%
We have applied our tool on several VTSs including
two found-in-nature VTSs.
%
Our experiments suggest that some of the synthesis problems are solvable by modern solvers and the synthesis technology may be useful for biological research.
%

The rest of the paper is organized as follows. 
%
In Sect.~\ref{sec:prelim}, we present the notations we use in this paper. 
%
In Sect.~\ref{sec:model}, we present graph model of VTSs and  properties of interest in Sect.~\ref{sec:property}.
%
%encoding of several constraints on VTSs. 
%
%In , we present the
%
In Sect.~\ref{sec:problem}, we present the complete formal problem statement of the verification properties.
%
In Sect.~\ref{sec:encoding}, we present the encoding of BMC, SMT, QBF and Synthesis into SAT, SMT and QBF satisfiability. 
%
In Sect.~\ref{sec:experiments}, we present our implementation and experimental results. 
%
In Sect.~\ref{sec:related}, we discuss related work and conclude in Sect.~\ref{sec:conclusion}.

%Previous analyses of VTS have mostly been based on sampling approach~\cite{mani2016wine, mani2016stacking}. 
%%
%In these analyses, vesicle traffic is
%modeled as a dynamical system. 
%%
%The traffic rule specifies how the system transitions from one
%time point to the next. 
%%
%Given a traffic rule and an initial condition, the system is evolved over time until a steady state is reached. By studying a large sample of randomly generated traffic
%rules, such analyses can make statistical claims about vesicle traffic.
%%
%Hence making precise general prediction about the properties of vesicle traffic networks over all possible traffic rules, not just for a sampled subset is not possible.
%%
%SAT and SMT solvers are a perfect fit for this situation, solving combinatorial problems with their exhaustive nature of searching. 
%%
%These tools check a specified logical property
%on every possible state of a model constructed using variables ranging over Boolean or any
%finite discrete type.

%%% Local Variables:
%%% mode: latex
%%% TeX-master: "main"
%%% End:


%\textbf{Components of the vesicle transport network}:
A characteristic feature of eukaryotic cells is the presence of multiple internal membrane-bound organelles called compartments.
% 
These compartments exchange molecules amongst themselves in small membrane-bound packets called vesicles. 
%
The recognition of cargo molecules to be loaded onto these vesicles and the recognition of the correct target compartments for each vesicle is performed by an intricate network of molecules called the vesicle transport system~\cite{alberts2002molecular}. 
%
We call the set of internal compartments, together with the molecular routes that connect them through vesicle exchange, the vesicle transport network. 
%
Vesicle traffic underpins nearly every aspect of eukaryotic cellular physiology, including
human cellular physiology.
%
Understanding how this system function is, therefore,
one of the central challenges of cell biology.
%
The high complexity of the problems~\cite{mani2016wine, mani2016stacking} in VTS restricts the use of traditional methods like simulation and statistical reasoning. 
%The high complexity of problem restricts the use of traditional methods like simulation and statistical reasoning and the requirement of precise analysis makes formal methods an ideal candidate. 
The requirement of precise analysis makes formal methods ideal candidate in this case. 

First, we will give an overview of the VTS mechanism and then will sketch out a few interesting problems that are key to the understanding of the VTS.
%
%\textbf{Components of the vesicle transport network}:
The vesicle transport network is used by cells to transport molecules across compartments.
% 
The mechanism of vesicle transport can roughly be broken down into two steps: budding of vesicles from the source compartment and their fusion with the target compartment.
%
Vesicle budding involves three steps:
\begin{enumerate}
	\item  \textbf{Sorting of cargo molecules by cytosolic adaptor molecules}. 
	Cargo adaptors bind to short linear stretches of amino acids in the cargo molecules called sorting signals and thereby sequester them for packaging. 
	%
	Different adaptor proteins act at different compartments, for eg., sec23/24 is required
	for packaging vesicles that form at the ER, the AP2 adaptor is used for packaging vesicles at the plasma membrane, etc. 
	%
	Adaptors are recruited to the source compartments through interactions with
	the lipids that form the source compartment's membrane, and interactions with Arf/Sar GTPases which signal the initiation of vesicle formation~\cite{paczkowski2015cargo}.
	
	\item \textbf{Membrane deformation, which involves cytosolic coat proteins}.
	Coat proteins are recruited to the forming vesicle through their interaction with adaptor proteins. 
	%
	These are naturally curved proteins, and their assembly on membranes leads to membrane deformation.
	%
	A single kind of coat protein can bind to multiple kinds of adaptors, thereby allowing the packaging of multiple types of cargo within the same vesicle. 
	%
	Clathrin, COP2 and COP1 are the three major kinds of coat proteins
	in eukaryotic cells. 
	%
	The clathrin coat operates at multiple compartments, and can interact with
	multiple adaptors proteins, whereas the COP2 and COP1 coat are only involved in vesicle formation at a single compartment; the ER and the Golgi respectively~\cite{faini2013vesicle}
	
	
	\item \textbf{Pinching of vesicles from the source compartment by dynamin}.
	The final step in vesicle formation is its piching off from the source compartment.
	%
	This process requires ATP and is performed by the cytosolic dynamin proteins, which are recruited to the vesicle neck by the coat proteins~\cite{cocucci2014dynamin}
\end{enumerate}

Vesicle fusion involves two steps:
\begin{enumerate}
	\item \textbf{Recognition of the correct target compartment}. 
	This is a complex process involving Rab
	GTPases and tethering complexes:
	
	\begin{enumerate}[label=(\roman*)]
		\item \textbf{Rab GTPases}  are small proteins which occur in two forms: the membrane associated and active
		GTP bound form, and the cytosolic and inactive GDP bound forms. 
		%
		Different compartments of the
		cell are associated with a different type of Rab protein. In their active form, Rabs recruit many
		downstream proteins, including tethering complexes, fusion regulators, motor proteins, sorting
		adaptors, etc, to the membrane they are associated with. 
		%
		The proteins that are recruited by active
		Rabs are called Rab effectors.
		%
		Rab activity is controlled by two proteins: GEFs turn Rabs on and
		GAPs turn them off. 
		%
		Each Rab has its own specific GEF and GAP. In some cases, GEFs and GAPs
		themselves can be Rab effectors, thus generating feed-forward and feedback loops. 
		%
		In their cytosolic, inactive form, Rabs are complexed with GDI proteins, and it is in this form that Rabs are presented to membrane bound GEFs for activation~\cite{muller2018molecular}
		
		\item \textbf{Tethering complexes} sequester vesicles to their target compartments and regulate SNARE
		proteins, which enable vesicle fusion. 
		%
		Tethers are believed to bind to SNAREs and function as
		chaperones for SNARE complex assembly which is the ultimate step in membrane fusion. 
		%
		There are two types of tethering complexes: multi subunit tethers and long coiled-coil tethers:
		
		\begin{enumerate}[label=(\alph*)]
			\item \textbf{Coiled coil} tethers are long, single subunit proteins. 
			%
			These tethers are believed to bind the
			vesicle on one end and the target compartment on the other end, thereby bridging the two before
			fusion.
			%
			A clear mechanism for action of these tethers is missing, but based on their interactions with
			other proteins of the traffic machinery, some hypotheses as to their mechanism of action have been
			put forth. 
			%
			Golgins, coiled coil tethers that are anchored at the Golgi by a transmembrane domain, contain a large number of binding sites for Rab GTPases. 
			%
			Thus, golgins might capture vesicles by
			binding vesicle-associated Rab proteins. 
			%
			Golgins also contain domains which can sense membrane
			curvature, which they could be using to recognize vesicles. 
			%
			They might also be interacting directly
			with vesicle SNAREs~\cite{baker2016chaperoning}.
			
			\item \textbf{Multi-subunit tethering complexes (MTCs)} are composed of three or more different
			subunits.
			%
			MTCs are known to interact with vesicle coat proteins, Rab GTPases, SNAREs and SM proteins.
			%
			For example, the HOPS complex, which is required for homotypic (fusion of two identical
			membranes) and heterotypic fusion in the endo-lysosomal system, was shown to tether membranes
			through its interactions with the membrane-associated Rab GTPase Ypt7, acidic phospholipids and
			SNAREs. 
			%
			The HOPS complex binds both individual SNAREs and SNARE complexes. It also
			seems to protect assembling trans-SNARE complexes from premature disassembly. 
			%
			Another example is given by the Dsl1 complex, which is anchored to the ER membrane through interactions
			with t-SNAREs. 
			%
			At the other end, the Dsl1 complex contains multiple binding sites for COPI
			(vesicle coat that is present on vesicles produced by the Golgi). 
			%
			This structure suggests that the Dsl1
			complex functions as a tether connecting COPI-coated vesicles to their target organelle, the ER~\cite{baker2016chaperoning}.
			
		\end{enumerate}
		
	\end{enumerate}
	
	\item \textbf{Fusion of vesicle with the target compartment} is brought about by SNARE proteins. 
	%
	SNAREs
	are defined by a 60- to 70-residue SNARE motif. Most SNAREs are anchored to the membrane by
	their C-terminal transmembrane helices. 
	%
	The formation of a productive SNARE complex requires
	the formation of a four-helix bundle containing four different SNARE motifs. Usually, one of these
	motifs is contributed by the SNARE on the vesicle membrane (v-SNAREs), and the other three
	motifs by SNAREs on the membrane of the tartget compartment (t-SNAREs). 
	%
	Alternatively, they
	can be defined by their amino acid sequences as R-, Qa-, Qb-, and Qc-SNAREs. A few SNARE
	proteins, such as SNAP25, contain both Qb- and Qc-SNARE motifs.
	%
	Most v-SNAREs are RSNAREs, and most t-SNAREs are Q-SNAREs~\cite{yoon2018snare}.
	
	
\end{enumerate}

Many SNAREs also have N-terminal domains that regulate SNARE complex assembly and/or
interact with other parts of the vesicle fusion machinery. 
%
Qa-SNAREs have an amino terminal
domain called the Habc domain, which can fold back on the SNARE domain and hold the QaSNARE
in an inactive state. 
%
This inactivated state is stabilized by interactions with SM proteins~\cite{yoon2018snare}.	      

A subset of R-SNAREs (the longin SNAREs) possess an amino-terminal domain called the longin domain. 
%
In addition to regulation of SNARE activity as in the case of the Qa-SNARE Habcdomains,
these longin domains also regulate the localization of longin SNAREs through their
interaction with adaptor proteins~\cite{daste2015structure}.

Different vesicle-target compartment pairs in the cell are associated with unique SNARE
complexes.
%
Membrane fusion converts the trans-SNARE complex (v- and t- SNAREs on opposite
membranes) into a cis-SNARE complex (both v- and t- SNAREs on the same membrane).
Disassembly of cis-SNARE complexes requires Sec17 and Sec18 (the yeast SNAP and NSF
protein, respectively). 
%
This process requires energy which is released by the hydrolysis of ATP.

\textbf{SNAREs are regulated by SM proteins}, which have three modes of action: SM proteins can hold
the Qa-SNAREs in an autoinhibited state and prevent or postpone their assembly into SNARE
complexes.
%
Secondly, some SM proteins have been seen to act as a template upon which a halfzippered
complex between the Qa- and R-SNAREs (an early SNARE complex intermediate) can
form. 
%
In this mode, by choosing an R-SNARE located on the opposite membrane, SM proteins
could be imposing a filter inhibiting the formation of futile cis-SNARE complexes and promoting
trans-SNARE complexes.
%
Finally, SM proteins bind the four-helix bundles formed by assembled
SNARE complexes.
%
This mode of binding of SM proteins might protect assembled SNARE
complexes from premature disassembly by NSF and SNAPs and/or might stimulate fusion directly~\cite{baker2016chaperoning}.
%
SM proteins have also been shown to interact with tethers~\cite{yoon2018snare}.

\textbf{Major paths in the vesicle transport network}:
Molecules traverse the cell in a series of vesicles, and there are two major such routes of transport in all eukaryotic cells:
\begin{enumerate}
	\item The secretory route takes proteins from the ER, their site of production, to the plasma membrane, from where they are secreted out of the cell. Cargo proteins leave the ER in COP2 coated vesicles which fuse with the Golgi apparatus. The golgi apparatus is where proteins get modified, for example by the addition of carbohydrate side-chains. Subsequently, proteins destined for secretion are packaged into clathrin coated vesicles which fuse with the plasma membrane, thus releasing their contents to the outside of the cell\cite{alberts2002molecular}.
	\item the endocytic routetakes proteins from the outside of the cell through the plasma membrane to the endocytic compartments, where they are digested. Cargo from the outside of the cell is taken in in the form of clathrin coated vesicles which form at the plasma membrane. These vesicles fuse amongst themselves in a process called homotypic fusion to generate the early endosomal compartments. Early endosomes undergo changes in their composition due to ongoing vesicle traffic and maturee into late endosomes. This maturation is concomitant with a switch the Rab protein they are associated with; the Rab5 associated early endosomes mature into Rab7 associated late endosomes\cite{rink2005rab}. Late endosomes then fuse with lysosomes, where all its contents get digested\cite{pryor2009delivery}.
\end{enumerate}

\textbf{Existential and universal condition:}
Other paths are used for cross talk between the secretory and the endocytic routes. For example, vesicles are sent from the TGN (Trans-Golgi network) to the early and the late endosomes to transport enzymes, and vesicles are sent back from the endocytic compartments to the TGN to recycle sorting receptors\cite{progida2016bidirectional}. Also recently, evidence for unconventional vesicle-mediated secretory routes have been found, which bypass the Golgi. The molecules involved in these routes are as yet unknown\cite{nickel2018unconventional}. 
%

\begin{table}
	\begin{center}
		\begin{tabular}{|c|c|c|c|c|c|c|c|c|c|c|}
			\hline
			Nodes & 1 & 2 & 3 & 4 & 5 & 6 & 7 & 8 & 9 & 10 \\ \hline
			Graphs & 0 & 0 & 0 & 1 & 2 & 15 & 121 & 2159 & 68715 & 3952378 \\ 
			\hline
		\end{tabular}
		\label{tab:threec}
		\caption{Number of simple 3-edge-connected unlabelled N-node graphs.}
	\end{center}	
\end{table}

\textbf{Difficulty of the problem:}
The analysis of vesicle traffic systems is a difficult problem
because of the combinatorial scaling of possible traffic topologies and regulatory rules. 
%
For example, we might want to check some conjecture of interest for all 3-connected graphs and
all possible variations of SNARE regulation rules. 
%
The number of graphs of specified connectivity grows exponentially with the number of nodes: Table~\ref{tab:threec} shows how many 3-edge-connected graphs~\cite{a052448-oeis} exist (without parallel or self edges) as node number N increases.

\textbf{Synthesis of VTS:}
Although vesicle-mediated traffic was discovered decades ago~\cite{wells2005discovery}, the picture of the vesicle transport network is far from complete; we do not yet know how many of the cellular proteins reach their resident organelles within the cell, and new vesicle routes are being discovered every year ~\cite{nickel2018unconventional,weill2018toolbox}. 

Completing the vesicle traffic network is a difficult task for many reasons: (1) the core vesicle transport network, which consists of a secretory and an endocytic route, is conserved across all eukaryotes, but the traffic network in different organisms \cite{richardson2015evolutionary,nishimoto2009differential,barlow2017seeing}, and even in different cell types within an organism can be different \cite{stoops2014trafficking,zhou2015arp2}, (2) although the basic traffic machinery is the same for all vesicle fusion events, the details of regulation can be different \cite{davletov2007regulation,di2010calcium}, (3) behaviour of traffic molecules in vitro is different from their behaviour inside cells \cite{furukawa2014multiple}, (4) molecules can have redundant routes within the cell \cite{shimizu2014compensatory,nakatsukasa2014nutrient}, and (5) it can be difficult to distinguish between the direct and indirect effects of experiments involving knock-downs or knock outs of traffic molecules \cite{hirst2004epsinr,mishev2013small}.

Nonetheless, completing the vesicle traffic network for different organisms is very useful. Many diseases are caused by the malfunction of the vesicle traffic network. Knowledge of the complete network would be helpful in identifying the root causes\cite{bexiga2013human,gissen2007cargos}. At a more basic level, having complete pictures of the vesicle traffic system for various organisms and various cell-types would allow comparative studies, and therefore would allow the deduction of modes of evolution of the form of the traffic network, and would also allow us to decipher features such as parts of the traffic system that is unchanging, and therefore likely to be its core, and parts that are more plastic \cite{barlow2017seeing}.

In this paper, using constraints on global vesicle traffic network topology due to local molecular interactions, we take incomplete pictures of vesicle traffic networks as inputs and output various completed versions which can then be tested experimentally.
%

%Previous analyses of VTS have mostly been based on sampling approach~\cite{mani2016wine, mani2016stacking}. 
%%
%In these analyses, vesicle traffic is
%modeled as a dynamical system. 
%%
%The traffic rule specifies how the system transitions from one
%time point to the next. 
%%
%Given a traffic rule and an initial condition, the system is evolved over time until a steady state is reached. By studying a large sample of randomly generated traffic
%rules, such analyses can make statistical claims about vesicle traffic.
%%
%Hence making precise general prediction about the properties of vesicle traffic networks over all possible traffic rules, not just for a sampled subset is not possible.
%%
%SAT and SMT solvers are a perfect fit for this situation, solving combinatorial problems with their exhaustive nature of searching. 
%%
%These tools check a specified logical property
%on every possible state of a model constructed using variables ranging over Boolean or any
%finite discrete type.

%%% Local Variables:
%%% mode: latex
%%% TeX-master: "main"
%%% End:


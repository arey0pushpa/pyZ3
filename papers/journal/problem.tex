\noindent %\improve{The weakest part of the paper; Need a serious rewrite!}
We analyze the model against our desired properties.
%
%Since in biology there are no declared specifications or properties, we conjecture some properties that appear to be 
%true in  relevant inall the known biological systems.
%
In this paper, we choose graph connectedness to be a key property.
%
%We categorize the search problem into three different categories (Fig.~\ref{fig:vts-search}), search problem for the existential condition, search problem for the universal condition, and performing synthesis on the incomplete VTS. 
%
%We will present two encodings for the existential condition and one each for the universal condition and synthesis.

\begin {figure}[!t]
\centering
\begin{adjustbox}{width=0.9\columnwidth}
	{\Large
		%{	\scalefont{1.0}
		%\forestset{parent color/.style args={#1}{
		%		{fill=#1},
		%		for tree={fill/.wrap pgfmath arg={#1!##1}{1/level()*80},draw=#1!80!darkgray}},
		%	root color/.style args={#1}{fill={{#1!60!gray!25},draw=#1!80!darkgray}}
		%}
		
		\begin{forest}
			/tikz/every node/.append style={font=\sffamily,minimum size=1.0cm},
			for tree={
				edge+={->,thick},% uncomment for arrows
				draw,
				rounded corners,
				node options={align=center,},
				text width=1.7cm,
				calign=fixed edge angles, calign primary angle=-84,calign secondary angle=85,
				%	anchor=center,	
				%	calign=center,
			},
			where level=0{%
				parent anchor=children,
			}{%
				folder,
				grow'=0,
				if level=1{% this changes the edges from level 0 to nodes at level 1
					before typesetting nodes={child anchor=north},
					edge path'={(!u.parent anchor) -- ++(0,-5pt) -| (.child anchor)},
				}{},
			}
			[\textbf{VTS Search \\ Problems}, fill=white!20, parent, s sep=1mm,
			[\textbf{1.Existential Condition}, for tree={fill=fv, child}, 
			[BMC, for tree={fill=ai, child}]
			[SMT, for tree={fill=ai, child}]
			]
			[\textbf{2.Universal Condition}, for tree={fill=fv, child},
			[QBF, for tree={fill=ai, child}]
			]
			[\textbf{3.VTS Synthesis}, for tree={fill=fv,child}, 
			[Synthesis (QBF), for tree={fill=ai, child}]
			]
			]
		\end{forest}
	}
\end{adjustbox}
\vspace{0.01cm}

\begin{tikzpicture}
\begin{customlegend}[legend columns=-1,legend style={draw=none,column sep=1ex},legend entries={ \small Search Problems, \small Encodings}]
\addlegendimage{fill=fv,area legend} 
\addlegendimage{fill=ai,area legend} 
%\addlegendimage{fill=ar,area legend}
%\addlegendimage{red,fill=black!50!red,ybar,ybar legend}
\end{customlegend}
\end{tikzpicture}

\caption{An overview of the VTS search problems with their respective encodings}
\label{fig:vts-search}
\end{figure}

\vspace{0.2cm}
\noindent We seek to answer three kinds of questions given the property (Fig.~\ref{fig:vts-search}).
% %
% In this paper,
% our conjecture relates graph-connectedness to a particular variation of SNARE pairing and
% regulation rules of the VTS.
% %
% We are interested in the following two type of properties of the transport graph:
\begin{enumerate}
\item {\em Existential condition}:
if there is a VTS that satisfies the connectedness property. 

\item {\em Universal condition}:
  if there is a connectedness property that is satisfied by each valid VTS.
\item{\em Synthesis on incomplete VTS}:
    given an incomplete VTS, if we can predict missing parts.
\end{enumerate}
The goal is to find the minimum graph connectivity properties for each of the condition. We will present two encodings for the existential condition and one each for the universal condition and synthesis.
% In our program, the rules VTS needs to follow 
% %
% are represented as constrained over the transport graphs (G), label on the nodes and edges and activity of those labels (determined by a Boolean function f). 
% %
% Each of these constrained functions is defined to be TRUE for a G and f: if and only if the corresponding condition holds for the given G and f. 
% %
% We use these functions to define the assertion that characterizes the corresponding condition to be checked. 
% %
% We will show in later sections how we encode these Boolean functions. 
\subsection{Search for existential condition} 
%
% To ensure that k-connectedness is an existential condition,
%
We need to find a $k$-connected graph that satisfies all the rules (and constraints) of the system. 
%
We also aim to find the minimum $k$. 
%
%Let's fix our conjecture: ``k-connectedness is a necessary condition for the system regulated by a Boolean function on the node and SNARE-SNARE inhibition on the edge". 
%
To find a $k$-connected graph we specify our property as a query for the existence
of a model for a conjunction of the formula representing stability, fusion and the specific
graph connectivity constraint. 
%
% \begin{align*}
%   &\texttt{LabeledVTS} = \texttt{Stability} \, \land \, \texttt{Fusion} \land \, \texttt{MinConnectivity(k)}
%  \tag{E}\label{eq:existcond}
% \end{align*}
%
We will start with the value of $k = 1$ and check for the satisfiability of the formula.
%
In case the formula is satisfiable the procedure terminates and we report the value of $k$. 
%
If the formula is not satisfiable we increment $k$ by $1$ and repeat the same procedure. 
%
In this way, we ensure that the reported $k$ is the minimum connectivity of the VTS that
satisfies the existential condition.
%

The implementation of the $k$-connectedness property is challenging as it requires reasoning about each combination of $k - 1$ edge removal. 
%
%The implementation of the k-connectedness property is challenging as it requires reasoning about every possible k − 1 edge removals.
%
So instead of checking for the graph to be $k$-connected, the encoding (and the implementation) checks whether the graph is not $k+1$-connected, i.e., search for a $k$ edge removal that disconnects the graph.
%
Note that this suffices to find the exact $k$, as we are starting from the smallest value of $k$ and subsequently incrementing it by one.
%Note that this suffice to find the correct $k$ as we are starting from smallest $k$ ($k = 1$) and subsequently increment it by one.     
% Used E as edge and L_e L_n as label for node and the edge,

We employ two techniques to examine the existential condition: first, using BMC and second using SMT solving.
%
Throughout the paper, we will address these two subcategories as $\fbmc$ and $\fsmt$ search problems.
%
%Here, we present both the $\fbmc$ and $\fsmt$ search problem for the existential condition.
%
%\subsubsection{$\fbmc$ search problem.}
%For a given $k$, size $\nu$, molecule number $\mu$, and unwinding depth $w$,
%we are searching for well-structured, stable, and well-fused VTS
%$G = (\nodes,\mols,\edges,\nlabel,\pairs,\edgef,\nodef)$ such that
%$|\nodes| = \nu$, $|\mols| = \mu$, and $G$ is not $k$-connected, within $w$ depth of unwinding.    
%
%\subsubsection{$\fsmt$ search problem.}
%For a given $k$, size $\nu$, and molecule number $\mu$,
%we are searching for well-structured, stable, and well-fused VTS
%$G = (\nodes,\mols,\edges,\nlabel,\pairs,\edgef,\nodef)$ such that
%$|\nodes| = \nu$, $|\mols| = \mu$, and
%$G$ is not $k$-connected.    


\subsection{Search for universal condition}
%
% Similarly, for the conjecture that k-connectedness is a universal condition for the VTS,
For some $k>0$,
we have to ensure that every $k$-connected graph satisfies all
the rules of the system.
%
We also aim to find the minimum such $k$. 
% We also have to ensure that the particular k is the least one. 
%
This condition can be specified as: for every $k$-connected graph, there exists a satisfiable assignment following the rules of the system. 
%
% \begin{align*}
%   & \forall G \, \texttt{Connectivity(k)} \implies  \exists
%                         f,p: \texttt{Stability} \, \land \, \texttt{Fusion}  
%   \tag{U}\label{eq:univcond}
% \end{align*}
%
We will use the same procedure as the existential condition to ensure that $k$ is minimized.
%

%In the implementation, to conclude that a given $k$-connectedness is not a universal condition it is suffice to search for a $k$-connected graph that has no satisfying assignment of all the rules of the system, i.e., checking the negation of the condition is unsatisfiable.
%In the implementation, to conclude that a given $k$-connectedness is not a universal condition we search for a $k$-connected graph that has no satisfying assignment of all the rules of the system, i.e., checking the negation of the condition is unsatisfiable.
%
 
The formula for the problem has quantifier alternation. 
%
Most SAT solvers can check (at least efficiently) only quantifier-free formulas.
%
%
We need a QBF solver for such queries. 
%
We refer the corresponding search problem as $\fqbf$ search problem.
%

For an efficient implementation of the specification, instead of checking the satisfiability of the formula, we check whether the negation of the formula is unsatisfiable. 
%
That is, to conclude that a given $k$-connectedness is not a universal condition, we search for a $k$-connected graph that has no satisfying assignment to all the rules of the system.
%
%In case no such graph exists for a specific $k$ we report the $k$.
%%\ankit{This is a dummy template. need a rewrite.} 
%\subsubsection{$\fqbf$ search problem.} For a given $k$, size $\nu$, and molecule number $\mu$, we are searching for a $k$-connected graph such that there is no assignment that makes it a well-structured, stable, and well-fused VTS $G = (\nodes,\mols,\edges,\nlabel,\pairs,\edgef,\nodef)$ with $|\nodes| = \nu$, $|\mols| = \mu$.    

%
%In this section, we will present a list of synthesis problems that may
%arise from the partially available information about a VTS and our synthesis method
%for the problems.

%\subsection{Problem Statements}
\subsection{Synthesis of VTS}
%
\noindent We consider another variant of analysis of VTSs.
%
We will assume that we are given a VTS, all of whose components
are not specified.
%
Our objective is to find the missing components.
%
The missing component can
% be in any of the components of VTS. 
%
%For example, 
be undiscovered edges or nodes, or insufficient
knowledge about the presence of molecules in some part of the VTS.
%
To cover most of the likely variations of this missing information,
we have encoded the following variants of the VTS synthesis problem.

\begin{enumerate}
	\item Fixing VTS by adding edges 
	\item Fixing VTS by adding molecules to the labels
	\item Fixing VTS by learning activity functions
	% \begin{enumerate}
	%   \item kcnf: low depth circuit.
	%   \item Boolean gates: And, Or.
	%   \item Boolean gates with linear combination.  
	% \end{enumerate}        
	% - Function dependence with var occurring once.
	\item  Fixing VTS by both adding/deleting components
\end{enumerate}

\subsubsection{Synthesis problem.}

We will do synthesis against the property that the VTS
is stable and 3-connected.
%
% The property is designed to balance the search space such that the synthesis procedure does not
% succeed with simply adding too many edges. 
%
%\begin{align*}
%\texttt{Property} =  \texttt{Stability} \land \texttt{Connected}(3) 
%%\land \texttt{DisConnected}(4)
%\end{align*}
This property was proposed in~\cite{shukla2017discovering}.
%
However, the biological relevance of the property is debatable and open for change.
%

%To handle this challenge, we used a combination of nondeterminism and Boolean enumeration at the C-source level to eliminate quantifiers, as explained further below.
%
%%% Local Variables:
%%% mode: latex
%%% TeX-master: "main"
%%% End:
          

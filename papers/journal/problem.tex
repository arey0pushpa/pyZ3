This section will present the different search problems of VTS.
%
We categorize the search problems into three different categories, search problem to solve the existential condition, search problem for universal condition and performing synthesis on the incomplete VTS. 

\subsection{Existential condition}
We employ two techniques to investigate the existential condition. 
%
First, using bounded model checking (BMC).
%
%The Bounded model checking technique
It performs a search for violation of the property in the program executions whose length is bounded by some fixed integer value, a \textit{depth}. 
%
In the case of no violation for a particular depth, one can iteratively increase the depth until either property is violated, the problem becomes intractable, or an upper bound of the system is reached.
%
For our case, the depth (called as \textit{unwinding} in the technical terms) is the length of the path in the underline graph.
%
%
For the experimentation, we will restrict ourselves to the elementary cycles (cycles of length one). 
%
This restriction puts an upper limit of $N$ on the length of each searched path. 
%
%
%and the upper bound is the number of nodes $N$.

Second, using constraint solving, where the rules are represented as a constraint over the variables defined in the system.
%
The search problem is to find an assignment of variables that satisfies all the generated constraints.
%
We have used a form of this technique called satisfiability modulo theories (SMT).
%
The SMT is a generalization of Boolean satisfiability problem with domain-specific reasoning, for example, support for uninterpreted functions, equality, arithmetic and data structures like arrays, bit-vectors etc. 
%
%It includes formulas with variables from other domains, including real numbers, integers, and support for various data structures such as lists, arrays, bit vectors.
% is satisfiability of formulas with respect to some background theory.
%
Throughout the paper we will address these two subcategories as $\fbmc$ and $\fsmt$ search problems.
%The existential condition search problem is represented as a bounded model checking and constraint solving. 
%
%Based upon earlier discussion, we need to answer the following search question among VTSs.
%
Here, we will present both $\fbmc$ and $\fsmt$ search problem for the existential condition.

\subsubsection{$\fbmc$ Search problem.}
For a given $k$, size $\nu$, molecule number $\mu$, and unwinding depth $w$,
we are searching for well-structured, stable, and well-fused VTS
$G = (\nodes,\mols,\edges,\nlabel,\pairs,\edgef,\nodef)$ such that
$|\nodes| = \nu$, $|\mols| = \mu$, and $G$ is not $k$-connected with w depth of unwinding.    

\subsubsection{$\fsmt$ Search problem.}
For a given $k$, size $\nu$, and molecule number $\mu$,
we are searching for well-structured, stable, and well-fused VTS
$G = (\nodes,\mols,\edges,\nlabel,\pairs,\edgef,\nodef)$ such that
$|\nodes| = \nu$, $|\mols| = \mu$, and
$G$ is not $k$-connected.    

%\noindent
%The notion of stability and connectivity varies slightly for BMC encoding. 
%%In this subsection, we will present the concept of stability and connectivity for BMC encoding. 
%%
%We call a graph $G$ {\em stable} if for each $(n,M',n') \in \edges$ and $m \in M'$,
%there exists a $m$-path in G with $n_1 = n'$ and $n_\ell = n$.
%%
%%
%We call a graph $G$ {\em connected} if for each $n,n' \in \nodes$,
%$n'$ there is a path from $n$ to $n'$ in G.
%
%We call a graph $G$ $k$-connected if for each $\edges' \subseteq \edges$ and $|\edges'| < k$,
%VTS $(\nodes,\mols,\edges-E',\nlabel,\pairs,\edgef,\nodef)$ is connected.
%%	
%We use $m$-${path}_i^j$ to denote an $m$-path between source node i and target node is j. 
%
%
%Creating a formula to search a path of length N-1 for each molecule m on every edge E will be extremely large and harder for the solvers to handle.  
%%
%Even if there exist shorter paths we have to create the formula of maximum size N-1. 
%% 
%To resolve these problems we will use a technique called bounded model checking, where the depth of the path can be specified and consequently increased or decreased.  
%

\subsection{Universal condition}
%Here we will present the QBF.
We will model the search problem for the universal condition as a quantified Boolean formula, and resultant formula will be solved with the help of a QBF solver.
%
The QBF is a generalization of the Boolean satisfiability problem in which every variable is quantified using either existential or universal quantifiers.
%
The QBFs enables potentially more succinct encodings than propositional logic.
 
%\ankit{This is a dummy template. need a rewrite.} 
\subsubsection{$\fqbf$ search problem.} For a given $k$, size $\nu$, and molecule number $\mu$, we are searching for a $k$-connected graph such that there is no assignment for well-structured, stable, and well-fused VTS $G = (\nodes,\mols,\edges,\nlabel,\pairs,\edgef,\nodef)$ with $|\nodes| = \nu$, $|\mols| = \mu$.    

%
%In this section, we will present a list of synthesis problems that may
%arise from the partially available information about a VTS and our synthesis method
%for the problems.

%\subsection{Problem Statements}
\subsection{Synthesis of VTS}
We will assume that we are given a VTS, whose all components
are not specified.
%
Our objective is to find the missing parts.
%
The missing parts can be in any of the components of VTS. 
%
For example, some undiscovered edges or nodes, or insufficient
knowledge about the presence of molecules in some part of the VTS.
%
To cover most of the likely variations of this missing information,
we have encoded the following variants of VTS synthesis problem.

\begin{enumerate}
	\item Fixing VTS by adding edges 
	\item Fixing VTS by adding molecules to the labels
	\item Fixing VTS by learning activity functions
	% \begin{enumerate}
	%   \item kcnf: low depth circuit.
	%   \item Boolean gates: And, Or.
	%   \item Boolean gates with linear combination.  
	% \end{enumerate}        
	% - Function dependence with var occurring once.
	\item  Fixing VTS by both adding/deleting parts
\end{enumerate}

\subsubsection{Encoding synthesis property}

We will do synthesis against the property that the VTS
is stable and 3-connected.
%
% The property is designed to balance the search space such that the synthesis procedure does not
% succeed with simply adding too many edges. 
%
%\begin{align*}
%\texttt{Property} =  \texttt{Stability} \land \texttt{Connected}(3) 
%%\land \texttt{DisConnected}(4)
%\end{align*}
This property was proposed in~\cite{shukla2017discovering}.
%
However, the biological relevance of the property is debatable and open for change.
%
Our tool is easily modifiable to support any other property that may be deemed 
interesting by the biologists.


%%% Local Variables:
%%% mode: latex
%%% TeX-master: "main"
%%% End:
             

\usepackage{mathtools}
\usepackage{makecell}
%\usepackage[]{graphics}
\usepackage[dvipdfmx]{graphicx}
\graphicspath{ {./} }

\usepackage{amsmath}
\usepackage{pdflscape}
\usepackage{rotating}
% \usepackage[top=0.85in,left=2.75in,footskip=0.75in]{geometry}

%\expandafter\def\csname ver@etex.sty\endcsname{3000/12/31}
%\let\globcount\newcount

% I ADDED FOR THE CHANGE IN ENUMERATE TO ALPHABBETICAL
\usepackage{enumitem}

%\DeclareFontShape{OT1}{cmr}{bx}{sc}{<-> cmbcsc10}{}

% amsmath and amssymb packages, useful for mathematical formulas and symbols
\usepackage{amsmath,amssymb}

%\renewcommand{\figurename}{Fig.{}}

% To add mail sign in the author 
\usepackage[]{ifsym}

% Use adjustwidth environment to exceed column width (see example table in text)
\usepackage{changepage}

% Use Unicode characters when possible
\usepackage[utf8x]{inputenc}

% textcomp package and marvosym package for additional characters
\usepackage{textcomp,marvosym}

% cite package, to clean up citations in the main text. Do not remove.
\usepackage{cite}

% Use nameref to cite supporting information files (see Supporting Information section for more info)
\usepackage{nameref}
\usepackage[breaklinks=true]{hyperref}
\usepackage{breakcites}

\setlength{\marginparwidth}{2cm}
% line numbers
%\usepackage[right]{lineno}

% ligatures disabled
\usepackage{microtype}
\DisableLigatures[f]{encoding = *, family = * }

% color can be used to apply background shading to table cells only
%\usepackage[table]{xcolor}
\usepackage{xargs}                      % Use more than one optional parameter in a new commands
\usepackage[pdftex,dvipsnames,table]{xcolor}  % Coloured text etc.
%\usepackage{todonotes}

% array package and thick rules for tables
\usepackage{array}

% Use package Listing to add code in our Manuscript
\usepackage{listings} 


% Added for the multi-column 
\usepackage[british]{babel}
\usepackage{hhline}
\usepackage{multirow}


%\usepackage[figurename=Fig]{caption}
\usepackage[labelfont=bf]{caption}
% Added for the sub-pictures
\usepackage{subcaption}
%\captionsetup[subfigure]{labelfont=rm}

%ADEDED BY ANKIT
% use (a) and i in the enumerate
\usepackage{enumitem}
%\usepackage[outdir=./images/]{epstopdf}
%% Save the class definition of \subparagraph
%\let\llncssubparagraph\subparagraph
%% Provide a definition to \subparagraph to keep titlesec happy
%\let\subparagraph\paragraph
%% Load titlesec
%\usepackage[compact]{titlesec}
%% Revert \subparagraph to the llncs definition
%\let\subparagraph\llncssubparagraph

%\titlespacing\section{0pt}{12pt plus 4pt minus 2pt}{0pt plus 2pt minus 2pt}
%\titlespacing{\section}{0pt}{*5}{*1.5}
%\titlespacing{\subsection}{0pt}{*4}{*1.5}
%\titlespacing\subsection{0pt}{12pt plus 4pt minus 2pt}{0pt plus 2pt minus 2pt}
%\titlespacing\subsubsection{0pt}{12pt plus 4pt minus 2pt}{0pt plus 2pt minus 2pt}


%Adjust the table 
\usepackage{adjustbox}


%add label refernce from another doc
\usepackage{xr}
\usepackage{catchfilebetweentags}

\usepackage{tkz-orm}
\usetikzlibrary{arrows,shapes,automata,backgrounds,petri,decorations.markings}
\usetikzlibrary{matrix,positioning,decorations.pathreplacing,calc,tikzmark}

\def\fsmt{\mathsf{SMT}}
\def\fbmc{\mathsf{BMC}}
\def\fqbf{\mathsf{QBF}}
\usepackage{hypcap} % fix the links

%\usepackage{lineno}
%\linenumbers

\usepackage{verbatim}
\usepackage{pgfplots}
%\usepackage{colortbl}
%\usepackage{pgfplotstable}
\pgfplotsset{compat=1.14}
	
\def\cca#1{\cellcolor{blue!#10}\ifnum #1>5\color{white}\fi{#1}}

\usepackage{tikz}
\usetikzlibrary{tikzmark,decorations.pathreplacing,arrows,shapes,positioning,shadows,trees,shapes.gates.logic.US,arrows.meta,shapes,automata,petri,calc}
%\usepackage{caption}
\usepackage{adjustbox}
\usepackage{todonotes}
% \usepackage[colorinlistoftodos,prependcaption,textsize=tiny]{todonotes}
% \newcommandx{\unsure}[2][1=]{\todo[linecolor=red,backgroundcolor=red!25,bordercolor=red,#1]{#2}}
% \newcommandx{\change}[2][1=]{\todo[linecolor=blue,backgroundcolor=blue!25,bordercolor=blue,#1]{#2}}
% \newcommandx{\info}[2][1=]{\todo[linecolor=OliveGreen,backgroundcolor=OliveGreen!25,bordercolor=OliveGreen,#1]{#2}}
% \newcommandx{\improve}[2][1=]{\todo[linecolor=Plum,backgroundcolor=Plum!25,bordercolor=Plum,#1]{#2}}
\usepackage{verbatim}
\usepackage{scalefnt}
\usepackage{wrapfig}
\usepackage{pgfplots}
\usepackage{tikz-qtree}
\usepackage{pgfplots}

\mathchardef\mhyphen="2D
\newcommand{\shorteq}{%
  \settowidth{\@tempdima}{-}% Width of hyphen
  \resizebox{\@tempdima}{\height}{=}%
}


%\definecolor{ai}{RGB}{179, 255, 179}
%\colorlet{fv}{red!55}
%\colorlet{ai}{blue!10}
%\colorlet{ar}{green!56}

\colorlet{fv}{gray!55}
\colorlet{ai}{gray!10}
\colorlet{ar}{gray!38}

\usepackage[edges]{forest}
\usepackage[T1]{fontenc}
%\usepackage{lmodern}


% argument #1: any options
\newenvironment{customlegend}[1][]{%
	\begingroup
	% inits/clears the lists (which might be populated from previous
	% axes):
	\csname pgfplots@init@cleared@structures\endcsname
	\pgfplotsset{#1}%
}{%
	% draws the legend:
	\csname pgfplots@createlegend\endcsname
	\endgroup
}%
\def\addlegendimage{\csname pgfplots@addlegendimage\endcsname}

\newcommand*{\equal}{=}
 \usepackage{tikz}
% \usetikzlibrary{arrows}
 \usepackage{xparse}
%\usetikzlibrary{}
\pgfdeclarelayer{myback}
\pgfsetlayers{myback,background,main}
\usetikzlibrary{tikzmark,decorations.pathreplacing,arrows,shapes,positioning,shadows,trees,shapes.gates.logic.US,arrows.meta,shapes,automata,petri,calc,matrix,backgrounds}
\tikzset{mycolor/.style = {line width=1bp,color=#1}}%
\tikzset{myfillcolor/.style = {draw,fill=#1}}%

\tikzset{%
  parent/.style={align=center,text width=3cm,rounded corners=3pt},
  child/.style={align=center,text width=3cm,rounded corners=3pt},
}

\NewDocumentCommand{\highlight}{O{blue!40} m m}{%
\draw[mycolor=#1] (#2.north west)rectangle (#3.south east);
}

\NewDocumentCommand{\fhighlight}{O{blue!40} m m}{%
\draw[myfillcolor=#1] (#2.north west)rectangle (#3.south east);
}


% create "+" rule type for thick vertical lines
\newcolumntype{+}{!{\vrule width 2pt}}
\renewcommand{\thesubfigure}{\Alph{subfigure}}
\renewcommand{\figurename}{Fig.}
\captionsetup[figure]{labelfont={bf},labelformat={default},labelsep=space,name={Fig.}}
\captionsetup[table]{labelsep=space}
\renewcommand{\thesubfigure}{\alph{subfigure}}


% create \thickcline for thick horizontal lines of variable length
\newlength\savedwidth
\newcommand\thickcline[1]{%
  \noalign{\global\savedwidth\arrayrulewidth\global\arrayrulewidth 2pt}%
  \cline{#1}%
  \noalign{\vskip\arrayrulewidth}%
  \noalign{\global\arrayrulewidth\savedwidth}%
}

\usepackage{array}
\newcolumntype{L}[1]{>{\raggedright\let\newline\\\arraybackslash\hspace{0pt}}m{#1}}
\newcolumntype{C}[1]{>{\centering\let\newline\\\arraybackslash\hspace{0pt}}m{#1}}
\newcolumntype{R}[1]{>{\raggedleft\let\newline\\\arraybackslash\hspace{0pt}}m{#1}}


% Remove comment for double spacing
%\usepackage{setspace} 
%\doublespacing

% Text layout
% \raggedright
% \setlength{\parindent}{0.5cm}
% \textwidth 5.25in 
% \textheight 8.75in
% create "+" rule type for thick vertical lines
% \newcolumntype{+}{!{\vrule width 2pt}}
%\renewcommand{\thesubfigure}{\Alph{subfigure}}

% \thickhline command for thick horizontal lines that span the table
\newcommand\thickhline{\noalign{\global\savedwidth\arrayrulewidth\global\arrayrulewidth 2pt}%
\hline
\noalign{\global\arrayrulewidth\savedwidth}}

% \raggedright
% \setlength{\parindent}{0.5cm}
% \textwidth 5.25in 
% \textheight 8.75in


% \usepackage[aboveskip=1pt,labelfont=bf,labelsep=period,justification=raggedright,singlelinecheck=off]{caption}
% \renewcommand{\figurename}{Fig}
% \usepackage{epstopdf}
% \AppendGraphicsExtensions{.tif}

%Added for correspondign author 
\newcommand{\ca}{$^{\textrm{\Letter}}$}

\newcommand{\booleans}{\mathbb{B}}
\newcommand{\naturals}{\mathbb{N}}
\newcommand{\integers}{\mathbb{Z}}
\newcommand{\ordinals}{\mathbb{O}}
\newcommand{\numarals}{\mathbb{I}}

\newcommand{\maps}{\rightarrow}

\newcommand{\union}{{\cup} }
\newcommand{\Union}{{\bigcup} }
\newcommand{\powerset}[1]{2^{#1}}
\newcommand{\intersection}{{\cap} }
\newcommand{\intersect}{\intersection}
\newcommand{\Intersection}{{\bigcap} }
\newcommand{\compose}{{\circ} }


\newcommand{\ltrue}{\mathbf{tt}}
\newcommand{\lfalse}{\mathbf{ff}}
\newcommand{\limplies}{\Rightarrow}
\newcommand{\lxor}{\oplus}
\newcommand{\Land}{\bigwedge}
\newcommand{\Lor}{\bigvee}
\newcommand{\Lxor}{\bigoplus}
\newcommand{\lequiv}{\Leftrightarrow}
\newcommand{\landplus}{\mathrel{:\hspace{-3pt}\land\hspace{-3pt}=}}
\newcommand{\lorplus}{\mathrel{:\hspace{-3pt}\lor\hspace{-3pt}=}}

\newcommand{\nodes}{N}
\newcommand{\mols}{M}
\newcommand{\nlabel}{L}
\newcommand{\edges}{E}
\newcommand{\pairs}{\mathcal{P}}
\newcommand{\nodef}{f}
\newcommand{\edgef}{g}


\newcommand{\lorem}{{\bf LOREM}}
\newcommand{\ipsum}{{\bf IPSUM}}

\newcommand{\npath}{\texttt{NPath}}

\newcommand{\vtstool}{\textsc{VTSArch}}
\newcommand{\zthree}{\textsc{Z3}}
\newcommand{\ourtool}{\textsc{VTSSynth}}
\newcommand{\sattool}{\textsc{VTSBMC}}
\newcommand{\smttool}{\textsc{VTSSMT}}
\newcommand{\qbftool}{\textsc{VTSQBF}}
\newcommand{\depqbf}{\textsc{DepQBF}}
\newcommand{\rarqbf}{\textsc{RaRQBF}}
\newcommand{\ashu}[1]{ {\textcolor{magenta} {Ashutosh: #1}} }
\newcommand{\mukund}[1]{ {\textcolor{blue} {Mukund: #1}} }
\newcommand{\somya}[1]{ {\textcolor{red} {Somya: #1}} }
\newcommand{\ankit}[1]{ {\textcolor{green!50!black}{Ankit: #1}} }

\newtheorem{df}{Definition}

%--------------------- DO NOT ERASE BELOW THIS LINE --------------------------

%%% Local Variables:
%%% mode: latex
%%% TeX-master: "main"
%%% End:

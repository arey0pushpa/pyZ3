\noindent 
We demonstrate in our work how local regulatory interactions between molecules of the VTS, translate into a global, observable properties of the VTS. 
%
Interactions between SNARE proteins and their regulators, which in its simplest form can be encoded in a pairing matrix, constrain the topological properties of the transport graph; all VTSs based on SNARE protein interactions must be 2-strongly connected. 
%
Additionally, we demonstrate how we can aid in predicting new vesicles given an incomplete VTS as input. 
%
Our approach demonstrates the power of SMT solvers in a novel application and the value of careful encoding of the problems into SMT queries. 
%
We manage to scale the tool up to the size that makes the tool biologically relevant.
%
%To conclude, we have established a connection between vesicle transport graph connectedness and underlying rules of SNARE pairing and regulation. 
%%
%The constraints implied by SNARE protein properties translates into
%mathematically elegant properties of the transport graph, namely its connectivity.
%%
%Our approach demonstrates the power of SMT solvers in a {\em novel} application and the value of careful encoding of the problems into SMT queries. 
%We manage to scale the tool up to the size that makes the tool biologically relevant.
%

However, the problem of searching for all $k$-connected VTSs induces quantifier alternation in the encodings. 
%
Therefore, a simple call to SAT solver may not work efficiently. 
%
We used QBF solvers for such queries. Our connectivity results contain many predictions which can in principle be tested through cell-biological experiments. 
%
For any input VTS, our tool outputs multiple possible solutions in the form of predictions of undiscovered vesicles and cargo molecules on specific vesicles. 
%
Such predictions could be used to guide biological experiments. Since a VTS that is close to completion is expected to have fewer alternative solutions to test, we expect that testing the predictions of our tool would become more tractable as more and more vesicles are discovered. 
%
One may also be interested in counting the number of graphs that satisfy a given property. 
%
Exact counting of the graphs using SAT solvers may prove to be very difficult. We are planning to employ approximate counting methods for finding the solution.

We also presented encodings of the synthesis problems that may arise from VTSs.
%
We demonstrated that our tool based on the encodings
scale up to the relevant sizes of the VTSs for some synthesis queries.
%
Our tool timed out on larger examples.
%
We are working to improve the performance of our tool.
%
%
With the tool, we made an attempt to present multiple new testable predictions. 
%
We will take this tool to the biologists and develop wet experiments that may
validate some synthesis results from the tool.
%
%These predictions may very likely increase to hasten discoveries in this field alongside pushing forward the use of formal methods in the biological field.
%

Since our current knowledge of VTSs is not extensive and there are many missing parts,
there is a huge space of potential predictions.
%
We also developing methods to prioritize predictions considering addition biological
information and using a refined model.
%
For example, in this work, we have only analyzed the VTS as static graphs. A major aspect of the biological cell’s VTS that the current model is missing is \textit{compartment maturation}, imbalances in incoming and outgoing molecular flux on vesicles.
%
In the future, we will study the dynamic behaviors of VTSs.

%%% Local Variables:
%%% mode: latex
%%% TeX-master: "main"
%%% End:

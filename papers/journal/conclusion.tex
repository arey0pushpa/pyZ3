\noindent 
%
\todo{copied frm the PLOS paper to Just a template}
We have established a connection between vesicle transport graph connectedness and underlying rules of SNARE pairing and regulation. 
%
Our results contain many predictions which can in
principle be tested through cell-biological experiments. 
%
However, our current knowledge of the actual vesicle transport graph of a cell is very poor: it is likely that there are many vesicle transport routes that are still uncharacterized. Even in those instances where vesicles are known to exist, the precise molecular drivers of their creation and fusion are often unknown.
%
Therefore, a direct verification of our predictions based on graph connectedness alone is not likely. 
%
However, more indirect predictions based on how SNAREs are regulated can be
explored. 
%
In this context, the Sec/MUNC proteins or SNARE longin domains are interesting
targets of study~\cite{van2010one,rossi2004longins}.

\todo{Copid just to create a template}
In the experiments, we have demonstrated the power of SMT solvers and the
value of careful encoding of the problems into SMT queries. 
%
We manage to scale the tool upto the size that makes tool biologically relevant. However, there are many further search problems or extensions that are of interest. 
%
For example, are there any k-connected graphs that can not be annotated into a VTS? 
%
This problem induces a quantifier alternation in an encoding. Therefore, a simple call to SAT solver will not work. 
%
We are planning to use QBF(quantified Boolean
formulas) solvers or develop iterative search algorithm for such queries.
%
One may be interested in counting the number of graphs that satisfy a given
property. 
%
Exact counting of the graphs using SAT solvers may prove to be very
difficult. 
%
We are also planning to employ some methods for approximate counting
of solutions.

We do not have a complete picture of the VTS graph, and so far, discoveries of new vesicles and molecular components are being made by groups interested in finding the path of a particular molecule, or they are chance discoveries. 
%
\todo{Copied till para end! Need revision}
In this paper, we presented encodings of the synthesis problems
that may arise from VTSs.
%
%
We demonstrated that our tool based on the encodings
scale up to the relevant sizes of the VTSs for some synthesis queries.
%
Our tool timed out on larger examples.
%
We are working to improve the performance of our tool.
%
%
With this tool, we made an attempt to present multiple new testable predictions. 
%
We will take this tool to the biologists and develop wet experiments that may validate some synthesis results from the tool.
%
These predictions may very likely increase to hasten discoveries in this field alongside pushing forward the use of formal methods in the biological field.

%Our model of VTSs is static graphs.
%
However, there are a few aspects in the modeling that we have relaxed.
%  \item What  needs  to  be  done  to  use  these  tools  to  real  world  BIO problems. In what aspect you want more, where it lagged.
%\end{enumerate}
%\textbf{Encode Dynamic behaviour of the VTS}
A major aspect of the biological cell’s VTS that the current model is missing is compartment maturation. 
%
We have only analyzed the VTS as a static graphs.
%
Compartments of the cell are not in steady state; rather they are changing in molecular composition due to imbalances in incoming and outgoing molecular flux on vesicles.
%
Maturation  is  evident  in  the  Golgi  apparatus  and  in  the  endocytic system. 
%
Including this behaviour of the compartments in the model would allow it to make more accurate predictions.
%%% Local Variables:
%%% mode: latex
%%% TeX-master: "main"
%%% End:

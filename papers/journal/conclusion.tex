\noindent 
%
We do not have a complete picture of the VTS graph, and so far, discoveries of new vesicles and molecular components are being made by groups interested in finding the path of a particular molecule, or they are chance discoveries. 
%
\todo{Copied! Need revision}In this paper, we presented encodings of the synthesis problems
that may arise from VTSs.
%
%
We demonstrated that our tool based on the encodings
scale up to the relevant sizes of the VTSs for some synthesis queries.
%
Our tool timed out on larger examples.
%
We are working to improve the performance of our tool.
%
%
With this tool, we made an attempt to present multiple new testable predictions. 
%
We will take this tool to the biologists and develop wet experiments that may validate some synthesis results from the tool.
%
These predictions may very likely increase to hasten discoveries in this field alongside pushing forward the use of formal methods in the biological field.

%Our model of VTSs is static graphs.
%
However, there are a few aspects in the modeling that we have relaxed.
%  \item What  needs  to  be  done  to  use  these  tools  to  real  world  BIO problems. In what aspect you want more, where it lagged.
%\end{enumerate}
%\textbf{Encode Dynamic behaviour of the VTS}
A major aspect of the biological cell’s VTS that the current model is missing is compartment maturation. 
%
We have only analyzed the VTS as a static graphs.
%
Compartments of the cell are not in steady state; rather they are changing in molecular composition due to imbalances in incoming and outgoing molecular flux on vesicles.
%
Maturation  is  evident  in  the  Golgi  apparatus  and  in  the  endocytic system. 
%
Including this behaviour of the compartments in the model would allow it to make more accurate predictions.
%%% Local Variables:
%%% mode: latex
%%% TeX-master: "main"
%%% End:

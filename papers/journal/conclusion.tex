\noindent In this paper, we presented encodings of the synthesis problems
that may arise from VTSs.
%
We demonstrated that our tool based on the encodings
scale up to the relevant sizes of the VTSs for some synthesis queries.
%
Our tool timed out on larger examples.
%
We are working to improve the performance of our tool.
%
We will take this tool to the biologists and develop wet experiments that may validate some synthesis results from the tool.
%
Our model of VTSs is static graphs.
%
In future, we will study the dynamic behaviors of VTSs.
%
It will allow us to predict behaviors after the perturbations in the VTSs
and more ways to test the predicted synthesis results.\\

\ankit{A rough outline of the answers needed in the future work section}
\begin{enumerate}
  \item why this tool was biologically helpful?
        \begin{itemize}
        	\item we do not have a complete picture of the VTS graph, and so far, discoveries of new vesicles and molecular components are being made by groups interested in finding the path of a particular molecule, or they are chance discoveries. 
        	\item With this tool, we make multiple new easily testable predictions. So it is very likely to hasten discoveries in this field.
        \end{itemize}
  \item What  needs  to  be  done  to  use  these  tools  to  real  world  BIO problems. In what aspect you want more, where it lagged.
\end{enumerate}


\textbf{Encode Dynamic behaviour of the VTS} A major aspect of the biological cell’s VTS that the current model is missing is compartment maturation. Compartments of the cell are not in steady state; rather they are changing in molecular composition due to imbalances in incoming and outgoing molecular flux on vesicles.
Maturation  is  evident  in  the  Golgi  apparatus  and  in  the  endocytic system. 
Including this behaviour of the compartments in the model would allow it to make more accurate predictions.
%%% Local Variables:
%%% mode: latex
%%% TeX-master: "main"
%%% End:

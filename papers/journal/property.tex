The conjecture relates graph-connectedness to a particular variation of SNARE pairing and regulation rules of the VTS.
%
We are interested in the following two type of properties of the transport graph:

\begin{enumerate}
\item Existential condition:
%the structural connectedness of the transport graph required for the system driven by a particular set of rules. 
structural connectedness of a transport graph which satisfies the stated set of rules. 

%there is a transport graph with the structural connectedness which satisfies the stated set of rules. 

%that a transport graph satisfies driven by a particular set of rules. 
%And we are interested in the least one. 

\item Universal condition: 
%every graph with the structural connectedness satisfies the stated set of rules. 
the structural connectedness that ensures every transport graph with that fundamental structure satisfies the stated set of rules. 
%The goal is to find the minimum graph connectivity that satisfies sufficient condition.
\end{enumerate}

The goal is to find the minimum graph connectivity that satisfies each of these condition.

%We express both these conditions as assertions in our program.

\subsubsection{Checking existential conditions} 
In our program, the rules VTS needs to follow 
%
%are we represent the stability, fusion, SNARE pairing and regulation rules, and the connectedness property that the system needs to follow, as 
%
are represented as constrained over the transport graphs (G), label on the nodes and edges and activity of those labels (determined by a Boolean function f). 
%
Each of these constrained functions is defined to be TRUE for a G and f: if and only if the corresponding condition holds for the given G and f. 
%
We use these functions to define the assertion that characterizes the existential condition to be checked. 
%
We will show in later sections how we encode these Boolean functions. 


To ensure that k-connectedness is an existential condition, we have to find a k-connected graph that satisfies all the rules (and constraints) of the system. 
%
We also have to ensure that the particular k is the least one. 
%
%Let's fix our conjecture: ``k-connectedness is a necessary condition for the system regulated by a Boolean function on the node and SNARE-SNARE inhibition on the edge". 
%
To find a k-connected graph (G) we specify our property as a query for the existence of a model for a conjunction of the formula representing stability, fusion and specific graph connectivity constraint. 
%

\begin{align*}
  &\texttt{LabeledVTS} = \texttt{Stability} \, \land \, \texttt{Fusion} \land \, \texttt{MinConnectivity(k)} 
\end{align*}

We will start with the value of k = 1 and check for the satisfiability of the above formula.
%
In case the formula is satisfiable the procedure terminates and we report the value of k. 
%
If the formula is not satisfiable we increment the k by 1 and perform the same procedure. 
%
In this way, we ensure that the reported k is the minimum connectivity of the VTS that satisfies the existential condition.  
% Used E as edge and L_e L_n as label for node and the edge,

\subsubsection{Checking universal conditions}

Similarly, for the conjecture that k-connectedness is a universal condition for the VTS, we have to ensure that every graph of k-connected structure satisfies all the rules (and constraints) of the system. 
%
We also have to ensure that the particular k is the least one. 
%
This condition can be specified as: for every k-connected graph, there exists a satisfiable assignment following the rules of the system. 

\begin{align*}
  & \forall G \, \texttt{Connectivity(k)} \implies  \exists
                        f,p: \texttt{Stability} \, \land \, \texttt{Fusion}  
\end{align*}

Note that the specification of this requires quantifier alternation. 
%
Most SAT solvers can check (at least efficiently) only quantifier-free formulas. 
%
%To handle this challenge, we used a combination of nondeterminism and Boolean enumeration at the C-source level to eliminate quantifiers, as explained further below.
We will use the same procedure as the existential condition to ensure the k to be the least one.



%%% Local Variables:
%%% mode: latex
%%% TeX-master: "main"
%%% End:
          

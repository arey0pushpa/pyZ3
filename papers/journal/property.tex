\noindent We analyze the model against our desired properties.
%
\ankit{Not completely correct}Since in biology there are no declared specifications or properties,
we conjecture some properties that appear to be true in all the known
biological systems.
%
In this paper, we choose graph connectedness to be a key property.
%
We will also seek two kinds of questions give the property.
% %
% In this paper,
% our conjecture relates graph-connectedness to a particular variation of SNARE pairing and
% regulation rules of the VTS.
% %
% We are interested in the following two type of properties of the transport graph:
\begin{enumerate}
\item {\em Existential condition:}
if there is a VTS that satisfies the connectedness property. 

\item {\em Universal condition:}
  if there is a connectedness property that is satisfied by each valid VTS.
\end{enumerate}
The goal is to find the minimum graph connectivity properties for each of the condition.
% In our program, the rules VTS needs to follow 
% %
% are represented as constrained over the transport graphs (G), label on the nodes and edges and activity of those labels (determined by a Boolean function f). 
% %
% Each of these constrained functions is defined to be TRUE for a G and f: if and only if the corresponding condition holds for the given G and f. 
% %
% We use these functions to define the assertion that characterizes the corresponding condition to be checked. 
% %
% We will show in later sections how we encode these Boolean functions. 
\subsubsection{Search for existential condition} 
%
% To ensure that k-connectedness is an existential condition,
%
We need to find a $k$-connected graph that satisfies all the rules (and constraints) of the system. 
%
We also aim to find the minimum $k$. 
%
%Let's fix our conjecture: ``k-connectedness is a necessary condition for the system regulated by a Boolean function on the node and SNARE-SNARE inhibition on the edge". 
%
To find a k-connected graph we specify our property as a query for the existence
of a model for a conjunction of the formula representing stability, fusion and specific
graph connectivity constraint. 
%
% \begin{align*}
%   &\texttt{LabeledVTS} = \texttt{Stability} \, \land \, \texttt{Fusion} \land \, \texttt{MinConnectivity(k)}
%  \tag{E}\label{eq:existcond}
% \end{align*}
%
We will start with the value of $k = 1$ and check for the satisfiability of the formula.
%
In case the formula is satisfiable the procedure terminates and we report the value of $k$. 
%
If the formula is not satisfiable we increment the $k$ by $1$ and repeat the same procedure. 
%
In this way, we ensure that the reported $k$ is the minimum connectivity of the VTS that
satisfies the existential condition.
%
% Used E as edge and L_e L_n as label for node and the edge,


\subsubsection{Search for universal condition}
%
% Similarly, for the conjecture that k-connectedness is a universal condition for the VTS,
For some $k>0$,
we have to ensure that every graph of $k$-connected structure satisfies all
the rules of the system.
%
We also aim to find the minimum $k$. 
% We also have to ensure that the particular k is the least one. 
%
This condition can be specified as: for every k-connected graph, there exists a satisfiable assignment following the rules of the system. 
%
% \begin{align*}
%   & \forall G \, \texttt{Connectivity(k)} \implies  \exists
%                         f,p: \texttt{Stability} \, \land \, \texttt{Fusion}  
%   \tag{U}\label{eq:univcond}
% \end{align*}
%
We will use the same procedure as the existential condition to ensure the k to be the least one.
%
The formula for the problem has quantifier alternation. 
%
Most SAT solvers can check (at least efficiently) only quantifier-free formulas.
%
We need a QBF solver for such queries.
%
%To handle this challenge, we used a combination of nondeterminism and Boolean enumeration at the C-source level to eliminate quantifiers, as explained further below.
%
%%% Local Variables:
%%% mode: latex
%%% TeX-master: "main"
%%% End:
          

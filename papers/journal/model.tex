In this section, we will present our formal model of VTS.
%
Our model attempts to capture only the relevant aspects of the
system to develop understanding the flow structure of VTSs.
%
The model assumes that a cell consists of clearly defined compartments
that are well stirred within themselves.
%
The model assumes that all chemical reactions get sufficient time to complete.
%
The model describes stable flow of molecules and does not
capture the transient nature of VTSs.
%
The model also does not capture concentrations of the molecules and
only captures their presence or absence.
%

In the model, the compartments are nodes of a graph
and vesicles are the edges between the nodes.
%
There is a set of relevant molecules that participate in
management of the VTS.
%
Both nodes and edges are labelled with subsets of the molecules.
%
The molecules are activity and their control is defined using
activity functions and a pairing matrix.
%

\begin{df}
  A VTS $G$ is a tuple $(\nodes,\mols,\edges,\nlabel,\pairs,\edgef,\nodef)$, where
  \begin{itemize}
  \item $\nodes$ is a finite set of nodes representing compartments in the VTS,
  \item $\mols$ is the finite set of molecules flowing in the system, 
  \item $\edges \subseteq \nodes \times (\powerset{\mols}-\emptyset) \times \nodes$ is the
    set of edges with molecule sets as labels,
  \item $\nlabel : \nodes \maps \powerset{\mols}$ defines the molecules present in the nodes,
  \item $\pairs \subseteq 2^{\mols}$ is pairing relation,
  \item $\nodef : \mols \maps \powerset{\mols} \maps B$ is activity maps for nodes, and
 \item $\edgef : \mols \maps \powerset{\mols} \maps B$ is activity maps for edges.
  \end{itemize}
\end{df}
$\nodes$, $\mols$, $\edges$, and $\nlabel$ define a labelled graph.
%
Additionally, $\pairs$ defines which molecules can fuse with which molecules,
and
$\nodef$ and $\edgef$ are the activity functions for molecules on
nodes and edges respectively.
%
% The model captures the steady state of a VTS.
% %
% The analysis of the model will inform us about the network/graph
% properties of VTSs.

% Let $\pairs(M')$ denote $\{P\setminus M'|  P \in \pairs \text{ and } P \intersect M' \neq \emptyset \}$.

% \textbf{Fusion condition in BMC and SMT:}
% $\pairs$ defines which molecules can fuse with which molecules.
% %
% Let $\pairs(M')$ denote $\{m|(m,m') \in P \text{ and } m' \in M'\}$.
% %
% $\nodef$ and $\edgef$ are used to define activity of molecules on
% nodes and edges respectively.
% %
% A molecule $k$ is {\em active} at node $n$ if $k \in \nlabel(n)$ and
% $\nodef(k,\nlabel(n))$ is true.
% %
% A molecule $k$ is {\em active} at edges $(n,M',n')$ if $k \in M'$ and
% $\edgef(k,M')$ is true.
% %
% We call $G$ {\em well-structured} if molecules $M$ is divided into
% two equal-sized partitions $M_1$ and $M_2$ such that
% $P \subseteq M_1 \times M_2$, and
% for each $(n,M',n') \in \edges$, $n \neq n'$, 
% $M' \subseteq \nlabel(n) \intersection \nlabel(n')$.
% %
% In other words, a well-structured VTS has no self loops, and 
% each edge carry only those molecules that are present in its source
% and destination nodes. 

% We will also consider several variations of the model.
% %
% For example, unique edge between two nodes, activity of molecules is
% not constrained by $\nodef$ and $\edgef$, etc.
% %


% \textbf{Fusion condition in QBF and synthesis:}
%
A molecule $k$ is {\em active} at node $n$ if $k \in \nlabel(n)$ and
$\nodef(k,\nlabel(n))$ is true.
%
A molecule $k$ is {\em active} at edges $(n,M',n')$ if $k \in M'$ and
$\edgef(k,M')$ is true.
%
We say $G$ is {\em fuse compatible} if molecules $M$ is divided into
two partitions $Q$ and $R$ such that
for each $P \in \pairs, |P \intersect Q| = 3 \land |P \intersect R| = 1 $.
%
In other words,
molecules are of two types $Q$ and $R$,
%
pairing relations have sets of four molecules such that three
are of type $Q$ and one of type $R$ or vice versa.
%
% This distinction of molecules is due to section~\ref{mol-types}.
%
We call $G$ {\em well-structured} if it is pair compatible, and
for each $(n,M',n') \in \edges$, $n \neq n'$ and
$M' \subseteq \nlabel(n) \intersection \nlabel(n')$.
%
In addition to fuse compatibility, the well-structured property requires that
here are no self loops, and 
each edge carry only those molecules that are present in its source
and destination nodes.
%
An edge $(n,M',\_) \in \edges$ {\em fuses} with a node $n'$
if there are non-empty set of molecules $M'' \subseteq M'$ and $M''' \subseteq \nlabel(n')$
such that $M''$ are active in the edge, $M'''$ are active in $n'$, and $M'' \union M''' \in \pairs$.
%
We call $G$ {\em well-fused} if each edge $(n,M',n') \in \edges$ fuses
with its destination node $n'$
and can not fuse with any other node.
%
$\pairs$ models biological possible quadruple of molecules that participate in fusion.
%
Each edge must be uniquely supported by such a quadruple of molecules.
%
We also require three of them to be on the node or edge, and the forth one
to be on the other side of the fusion.
% $\pairs(M'')$ are not active in any other node.

% We will also consider several variations of the model.
% %
% For example, the unique edge between two nodes, the activity of molecules is
% not constrained by $\nodef$ and $\edgef$, etc.
%

\vspace{0.2cm}
%\subsection{Uniform}
\noindent
A {\em path} in $G$ is a sequence $n_1,...,n_\ell$ of nodes 
such that $(n_i,\_,n_{i+1}) \in \edges$ for each $ 0 < i < \ell$.
%
For a molecule $m \in M$,
an {\em $m$-path} in $G$ is a sequence $n_1,...,n_\ell$ of nodes 
such that $(n_i,M',n_{i+1}) \in \edges$ and $m \in M'$ for
each $ 0 < i < \ell$.
%
A node $n'$ is {\em ($m$-)reachable} from node $n$ in $G$ if there is a ($m$-)path
$n,...,n'$ in $G$.
%
% A node $n'$ is {\em $m$-reachable} from node $n$ in $G$ if there is a
% $m$-path $n,...,n'$ in $G$.
%
We call $G$ {\em stable} if for each $(n,M',n') \in \edges$ and $m \in M'$,
$n$ is $m$-reachable from $n'$.
%
The definition of stability implies that if a molecule leaves a
compartment then there must be a path for it to come back.
%
It ensures that the flow of molecules does not deplete a
compartment.
%
Such a property is often expected in several physical systems
such that electromagnetic fields, fluid dynamics, etc.

%
We call $G$ {\em connected} if for each $n,n' \in \nodes$,
$n'$ is reachable from $n$ in $G$.
%
We call $G$ $k$-connected if for each $\edges' \subseteq \edges$ and
$|\edges'| < k$, VTS
$(\nodes,\mols,\edges-E',\nlabel,\pairs,\edgef,\nodef)$ is connected.
%
We will check if a VTS $G$ is $k$-connected or not.
%
The property of $k$-connectedness is a robustness property that one
expects from a biological systems.
%
The property says even after failure of a few edges the system will continue to work.

\ashu{Should be written bio people}
Let us discuss rough sizes of the system and expected behaviors.
%
A typical VTS may have about 10 compartments.
%
%Across eukaryotes, there are 20 broad SNARE varieties, though individual species can
%contain higher numbers (humans have 41)~\cite{kloepper2007elaborate}..
%
The VTS may transport a large number of molecules but the molecules that
are \todo{SA: commented line above} relevant of the control of VTS are about 50~\cite{kloepper2007elaborate}.
%
According to the definition the activity of molecules can be controlled by
the presence of all other molecules.
%
However, in practice the activity of a molecule is controlled by
the presence of ~2-3 other molecules.

%%% Local Variables:
%%% mode: latex
%%% TeX-master: "main"
%%% End:

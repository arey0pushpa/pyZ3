Vesicle Traffic Systems (VTSs) transport cargo among the intracellular compartments of eukaryotic biological cells.
%
The compartments are viewed as nodes that are labeled by their chemical identity and the transport vesicles are similarly viewed as labeled edges between the nodes.
%
Several interesting questions about VTSs translate to combinatorial search and synthesis problems. 
%
We present novel SAT, SMT and QBF based encodings of the properties over vesicle traffic systems.
%
We have implemented the presented encoding in a tool that searches for the networks that satisfy properties related to transport consistency conditions using these solvers. 
%
In our numerical experiments, we show that our tool can search for networks of sizes that are relevant to real cellular systems.
%
Understanding VTSs is an ongoing area of
research and for many cells they are partially known. 
%
For example, there
may be undiscovered edges, nodes, or their labels in a VTS of a cell. 
%
It has been speculated that there are properties that the VTSs must satisfy.
For example, stability, i.e., every chemical that is leaving a compartment
comes back. 
%
Many synthesis questions may arise in this scenario, where
we want to complete a partially known VTS under a given property.
%
In the paper, we present novel encodings of the above questions into
the QBF (quantified Boolean formula) satisfiability problems. 
%
We have implemented the encodings in a highly configurable tool and applied to a
couple of found-in-nature VTSs and several synthetic graphs.
%
Our results demonstrate that our method can scale up to the graphs of interest.
% Vesicle Traffic Systems(VTSs) are the material transport 
% mechanisms among the compartments inside the biological cells.

%%% Local Variables:
%%% mode: latex
%%% TeX-master: "main"
%%% End:

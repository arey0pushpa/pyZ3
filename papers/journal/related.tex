\noindent The previous analysis of VTS used simulation and statistical reasoning~\cite{mani2016stacking}. 
%
Such analysis is far from complete and can only make statistical claims about VTS based on a subset of samples. 
%
The properties of VTS discussed in this paper require precise analysis for all possible rules and variations, that make formal methods a suitable alternative.

The SAT solvers are used to solve a wide range of problems including
%
software and hardware verification~\cite{biere1999symbolic1, biere1999symbolic2, bjesse2001finding, velev2003effective}, equivalence checking~\cite{goldberg2001using}, automatic test case generation~\cite{stephan1996combinational},
AI planning~\cite{kautz1996pushing} and scheduling~\cite{gomes1998randomization}. 
%
The enormous progress in the performance of SAT and SMT solvers
%Improvement in SAT and SMT solvers
in the past decades have led to their use in modeling and understanding complex biological systems~\cite{heule2010exact,yordanov2013smt,mangla2010timing}. 
%
Many of which involve reasoning about graphs and networks rules~\cite{guerra2012reasoning,chin2008biographe} with possible exhaustive search.
% and hence can be reduced to solving a SAT/SMT question. 
%

One of the prominent areas of application of SAT and SMT solvers is modeling and understanding the gene regulatory networks (GRN)~\cite{giacobbe2015model,rosenblueth2014inference, yordanov2016method, dunn2014defining, paoletti2014analyzing, koksal2013synthesis}.
%
The gene regulatory network question is encoded either directly as a Boolean satisfiability problem or in addition with the combinations of background theories.
%
The generated formula is then solved with the help of a SAT solver. 
%
The~\cite{rosenblueth2014inference} uses symbolic model checking approach on the qualitative
models of regulatory networks.
%
Whereas,~\cite{dunn2014defining}  uses symbolic, SAT-based approach on the Boolean network model. 
%
The~\cite{giacobbe2015model, yordanov2016method} presents an SMT based encoding of the problem based on the presented Boolean model.
%

We have extended the application of SAT and SMT solvers to a comparatively more complex biological transport network~\cite{mani2016stacking} namely VTS. 
%
We have presented both SAT and SMT based encoding for our Boolean model of the VTS.
%
We have also introduced the use of QBF solvers and it's encoding in this context.
% 
%To our best knowledge, this series of work is the first application
%of formal methods to the vesicle traffic networks.

In recent years, there has been a wide range of methods
developed for similar synthesis problems~\cite{sketch,sygus,exampleSynth}.
%
They range from filling gaps an implementation of C programs from the pool of template predicates to learn a program from the example
runs of the program.
%
In the course of developing such methods,
the background technology, i.e. solving of quantified
constraints has been evolving rapidly~\cite{lonsing2010depqbf,z3Quant}.
%

There has been some work in applying synthesis technique in biology. %especially in gene regulatory networks~\cite{shavit2016automated, fisher2015synthesising}. 
A recent work by \cite{fisher2015synthesising} synthesize executable gene regulatory networks from single-cell gene expression data.
%
In \cite{shavit2016automated} synthesis of switching GRN (with changing topology) is performed.
%
The~\cite{fisher2015synthesising} uses constraint (satisfiability) solving techniques whereas \cite{shavit2016automated} uses SMT to perform synthesis. 
%
Synthesis technique is also used in optimal synthesis for chemical reaction networks~\cite{cardelli2017syntax}.
%
%
The work
%The paper \cite{cardelli2017syntax} in addition to using SMT over  ODE, 
use a template-guided approach in addition to using SMT over ODE. In our case queries contain quantifiers so we have employed QBF solving with Z3 for solving the synthesis problem. 
%
To our best knowledge, this is the first application of synthesis in VTS.
%%% Local Variables:
%%% mode: latex
%%% TeX-master: "main"
%%% End:

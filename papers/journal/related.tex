SAT and SMT solvers are a perfect fit for solving combinatorial problems due to their exhaustive nature of searching. 
%
Improvement in SAT and SMT solvers in the past decades have led to their use in modeling and understanding complex biological systems~\cite{heule2010exact,yordanov2013smt,mangla2010timing}. 
%
Many of which involve reasoning about graphs and networks rules with possible exhaustive search and hence can be reduced to solving a SAT/SMT question. 
%
Modeling and understanding the gene regulatory networks (GNR)~\cite{guerra2012reasoning,chin2008biographe}, modeling and understanding the gene regulatory networks (GNR) \cite{giacobbe2015model,rosenblueth2014inference, batt2010efficient, yordanov2016method, dunn2014defining, paoletti2014analyzing, koksal2013synthesis} is one such example. 
%
We have extended the application of SAT and SMT solvers to the most comple Biological transport network (VTS) \cite{mani2016stacking,shukla}. 

In recent years, there has been a wide range of methods
developed for the similar synthesis problems~\cite{sketch,sygus,exampleSynth}.
%
They range from filling gaps an implementation of C programs from the pool of template predicates to learn a program from example
runs of the program.
%
In the course of developing such methods,
the background technology, i.e. solving of quantified
constraints has been evolving rapidly~\cite{lonsing2010depqbf,z3Quant}.
%

There has been some work in
applying synthesis technique in biology especially in gene regulatory networks~\cite{shavit2016automated, fisher2015synthesising}. A very recent work~\cite{fisher2015synthesising} synthesize executable gene regulatory networks from single-cell gene expression data.
Synthesis technique is also used in optimal synthesis for chemical reaction networks~\cite{cardelli2017syntax}. The~\cite{fisher2015synthesising} uses constraint (satisfiability) solving techniques for the synthesis whereas ~\cite{shavit2016automated} uses SMT for synthesis. The paper~\cite{cardelli2017syntax} in addition to using SMT over ODE, uses a template-guided approach. In our case queries contain quantifiers so we have employed QBF solving with Z3 for the solving the synthesis problem. To our best knowledge, this is the first application of synthesis in VTS.

%%% Local Variables:
%%% mode: latex
%%% TeX-master: "main"
%%% End:
~        

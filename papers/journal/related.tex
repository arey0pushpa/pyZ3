\begin{table}[!ht]
	%\begin{adjustwidth}{-2.25in}{0in} % Comment out/remove adjustwidth environment if table fits in text column.
	\centering
	\begin{tabular}{|c|c|}
		\hline 
		\textbf{N} & \textbf{Total Number of graphs} \\ \hline
		1.         & 0                               \\ \hline
		2.         & 0                               \\ \hline
		3.         & 0                               \\ \hline
		4.         & 1                               \\ \hline
		5.         & 2                               \\ \hline
		6.         & 15                              \\ \hline
		7.         & 121                             \\ \hline
		8.         & 2159                            \\ \hline
		9.         & 68715                           \\ \hline
		10.        & 3952378                         \\ \hline
	\end{tabular}
	\label{tab-graphs}
     \caption{Number of simple 3-edge-connected unlabeled N-node graphs.}
	%\end{adjustwidth}
\end{table}

The analysis of vesicle traffic systems is a difficult problem
because of the combinatorial scaling of possible traffic topologies and regulatory rules. 
%
For example, we might want to check some conjecture of interest for all 3-connected graphs and
all possible variations of SNARE regulation rules. 
%
The number of graphs of specified connectivity grows exponentially with the number of nodes: Table~\ref{tab-graphs} shows how many 3-edge-connected graphs~\cite{a052448-oeis} exist (without parallel or self edges) as node number N increases.
%

Previous analyses of VTS have mostly been based on sampling approach~\cite{mani2016wine, mani2016stacking}. 
%
In these analyses, vesicle traffic is
modeled as a dynamical system. 
%
The traffic rule specifies how the system transitions from one
time point to the next. 
%
Given a traffic rule and an initial condition, the system is evolved over time until a steady state is reached. By studying a large sample of randomly generated traffic
rules, such analyses can make statistical claims about vesicle traffic.
%
Hence making precise general prediction about the properties of vesicle traffic networks over all possible traffic rules, not just for a sampled subset is not possible.
%
SAT and SMT solvers are a perfect fit for this situation, solving combinatorial problems with their exhaustive nature of searching. 
%
These tools check a specified logical property
on every possible state of a model constructed using variables ranging over Boolean or any
finite discrete type.

Improvement in SAT and SMT solvers in the past decades have led to their use in modeling and understanding complex biological systems~\cite{heule2010exact,yordanov2013smt,mangla2010timing}. 
%
Many of which involve reasoning about graphs and networks rules with possible exhaustive search and hence can be reduced to solving a SAT/SMT question. 
%
Modeling and understanding the gene regulatory networks (GNR)~\cite{guerra2012reasoning,chin2008biographe}, modeling and understanding the gene regulatory networks (GNR)~\cite{giacobbe2015model,rosenblueth2014inference, batt2010efficient, yordanov2016method, dunn2014defining, paoletti2014analyzing, koksal2013synthesis} is one such example. 
%
We have extended the application of SAT and SMT solvers to the most comple Biological transport network (VTS)~\cite{mani2016stacking,shukla}. 

In recent years, there has been a wide range of methods
developed for the similar synthesis problems~\cite{sketch,sygus,exampleSynth}.
%
They range from filling gaps an implementation of C programs from the pool of template predicates to learn a program from example
runs of the program.
%
In the course of developing such methods,
the background technology, i.e. solving of quantified
constraints has been evolving rapidly~\cite{lonsing2010depqbf,z3Quant}.
%

There has been some work in
applying synthesis technique in biology especially in gene regulatory networks~\cite{shavit2016automated, fisher2015synthesising}. A very recent work~\cite{fisher2015synthesising} synthesize executable gene regulatory networks from single-cell gene expression data.
Synthesis technique is also used in optimal synthesis for chemical reaction networks~\cite{cardelli2017syntax}. The~\cite{fisher2015synthesising} uses constraint (satisfiability) solving techniques for the synthesis whereas ~\cite{shavit2016automated} uses SMT for synthesis. The paper~\cite{cardelli2017syntax} in addition to using SMT over ODE, uses a template-guided approach. In our case queries contain quantifiers so we have employed QBF solving with Z3 for the solving the synthesis problem. To our best knowledge, this is the first application of synthesis in VTS.

%%% Local Variables:
%%% mode: latex
%%% TeX-master: "main"
%%% End:
~        

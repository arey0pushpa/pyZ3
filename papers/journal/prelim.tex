

% \subsection{VTS properties}

% For the synthesis of incomplete systems,
% we need properties against which we synthesize the missing parts.
% %
% Here we will discuss two such properties proposed in earlier
% works~\cite{smtVTS}, namely stability and $k$-connectedness.
% %

% \paragraph{Stability property}
% %
% We use Boolean variable $r_{i,j,m,p}$ to indicate if there is an
% $m$-path from $i$ to $j$ of length less than or equal to $p$.
% %
% We use $m$-reachability to encode the stability condition in VTSs.
% %
% The following constraint recursively encodes that node $j$ is
% $m$-reachable from node $i$ in less than $p$ steps.
% %
% Subsequently, we encode stability condition using the reachability variables.
% \begin{align*}
%   \texttt{Paths}(r) &= \bigwedge\limits_{\mathclap{i,j,m,p}} r_{i,j,m,p} \limplies (\bigvee_{q} \, e_{i,j,q,m} \lor \bigvee_{i\neq i^{\prime}} ( \, \bigvee_{q} e_{i,i^{\prime},q,m}) \land r_{i^{\prime},j,m,p-1} )
%   \\
%   \texttt{Loop}(r) &= \bigwedge\limits_{i,j,m} (\bigvee_{q} e_{i,j,q,m}) \limplies r_{j,i,m,\nu}
%   \\
%   \texttt{Stability} &= \exists r. \; \texttt{Paths}(r) \land \texttt{Loop}(r)
% \end{align*}

% \paragraph{$k$-connected property}
% %
% $k$-connectedness expresses robustness against failure of few edges.
% %
% Let us use $d_{i,j,q}$ to indicate $q$th edge between $i$ and $j$ is failed
% and $r'_{i,j}$ to indicate if there is a path from $i$ to $j$ in
% the modified VTS.
% %
% % To check whether $k$-connected is a
% % necessary condition, we remove (drop) $k-1$ edges from the graph and
% % if it disconnects the graph, and we get an assignment.
% % %
% % We have a graph that is not $k$-connected.
% %
% In the following, $\texttt{Fail}(d,k)$ encodes that only
% existing edges can be failed and exactly $k-1$ edges are failed.
% %
% $\texttt{FReach}(d,r')$ defines reachability in the modified VTS.
% %
% We use a new variable $r'_{i,j,p}$ to encode reachability from
% $i$ to $j$ in at most $p$ steps.
% %
% $\texttt{Connected}(r')$ says that all nodes are reachable from any
% other node.
% \begin{align*}
%   \texttt{Fail}(d,k) = & 
%   \bigwedge\limits_{i,j,q} d_{i,j,q} \limplies e_{i,j,q}  \land 
%   \sum_{i,j,q} d_{i,j,q} = k-1\\
%   % \texttt{ReachAbove}(d,r') = &
%   % \bigwedge\limits_{i,j}  [\bigvee_{q} (e_{i,j,q} \land  \neg d_{i,j,q}) \lor  (\bigvee_{i' \neq i}  r^{\prime}_{i',j} \land  \bigvee_{q} (e_{i,i',q} \land \neg d_{i,i',q}) ] \limplies r^{\prime}_{i,j}\\
%   \texttt{FReach}(d,r') = &\hspace{-1ex}
%    \bigwedge\limits_{i,j,p}  \hspace{-1ex}r^{\prime}_{i,j,p} \hspace{-1ex}\limplies\hspace{-1ex} [\bigvee_{q} (e_{i,j,q} \land  \neg d_{i,j,q}) \hspace{-1pt}\lor \hspace{-2pt} (\bigvee_{\mathclap{i' \neq i}}  r^{\prime}_{i',j,p-1} \land  \bigvee_{q} (e_{i,i',q} \land \neg d_{i,i',q}) ]\\
%   \texttt{Connected}(r') = & \Land\limits_{i,j} (r^{\prime}_{i,j,\nu} \lor r^{\prime}_{j,i,\nu})
% \end{align*}
% We will be synthesizing $k$-connected graphs.
% %
% We define $\texttt{Connected}(k)$ that says for all possible valid failures
% the graph remains reachable. 
% \begin{align*}
%   \texttt{Connected}(k) = & \forall d.\;
%           (\texttt{Fail}(d,k) \limplies \exists r'.\;\texttt{FReach}(d,r')
%                                 \land \texttt{Connected}(r'))
% % \\
%   % \texttt{Disconnected}(k) = & \exists d.\;
%   %         (\texttt{Drop}(d,k) \land \exists r'.\;\texttt{ReachAbove}(d,r')
%   %                               \land \lnot \texttt{Connected}(r'))
% \end{align*}
% Since $d$ variables in $\texttt{Connected}(k)$ are universally
% quantified, $\texttt{Connected}(k)$ introduces quantifier alternations.
% %
% Therefore, synthesis against this property will require QBF reasoning.
% %
% We may make the formula quantifier free  by considering all possible failures
% separately and introducing a vector of reachability variables for each
% failure.
% %
% However, this will blow up the size of the formula and may not be
% solvable by a SAT solver.
% %


\subsection{Solvers}
Due to the combinatorial nature of the graphs, the search space is huge
and often hard to enumerate naively.
%
We need sophisticated solvers 

%%% Local Variables:
%%% mode: latex
%%% TeX-master: "main"
%%% End:

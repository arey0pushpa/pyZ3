%\externaldocument{sat-table}

%We have developed four different tools (\sattool, \smttool, \qbftool, \ourtool) for implementation of proposed encodings.
%
We have developed the {\vtstool} tool to conjecture the properties of the VTS (explained in Section 6). 
%
%
The {\vtstool} tool includes about 10,000 lines of code.
%
The tool is divided into three different blocks,
%
checking existential condition, checking the universal condition and performing synthesis on the incomplete VTS. 
%
The existential condition is subdivided into two sub-block, bounded model checking and constraint solving. 
%
We will refer to existential subblocks as {\sattool} and {\smttool}, universal block as {\qbftool} and synthesis block as {\ourtool}.
%


\subsection{Implementation}
%The existential condition problem is implemented with two separate encodings.
We have implemented two separate encodings to determine the existential condition of the VTS. 
%To solve the existential condition we have implemented two separate encodings. 
%
The first encoding is implemented in {\sattool}
%One 
%with a successful bounded model checker, CBMC, 
and perform path enumeration to encode the reachability. 
%
%This encoding is implemented in {\sattool}.
% and uses a successful bounded model checker, CBMC.
%which is based on the architecture of a
%
%The second encoding is implemented in {\smttool} and uses a
%is based on 
%industrial strength SMT solver, Z3, that uses a 
%recursive encoding of finding a least fixed point to encode the same. 
The second encoding is implemented in {\smttool} and employs a recursive encoding of finding a least fixed point for the reachability. 

%
%The encoding is implemented in {\smttool}.
% and uses industrial strength SMT solver, Z3.
%The {\sattool} is based on the architecture of a bounded model checker (CBMC) and uses an enumeration of paths for encoding the reachability problem.
%
%The {\smttool} is based on the SMT solver Z3 and uses the efficient encoding of finding a least fixed point to solve reachability problem. 
%
%The {\qbftool} is based on the QBF encoding of the universal condition and uses {\depqbf} solver to solve the generated QBF. 
The QBF encoding of the universal condition is implemented in {\qbftool}.
%
The tool uses {\depqbf} solver to solve the generated QBF formula. 
%
The {\ourtool} is a synthesis tool for synthesizing incomplete information in the input VTS using {\qbftool} interface.
%

All the experiments were done
%Each test was run 
on a machine with Intel(R) Core(TM) i7-2600 CPU @ 3.40GHz processor and 16GB RAM.
%
A 120 min (7200 sec) timeout was used for the tool.

%%\begin{table}[!ht]
%%\begin{adjustwidth}{-2.25in}{0in} % Comment out/remove adjustwidth environment if table fits in text column.
%\centering
%\def\arraystretch{1.6}
%\caption{
%{\bf SNARE regulation and graph connectedness.}}
%\begin{tabular}{|c|c|c|c|c|c|c|}
%\hline
%\bf{Sr.No} & 	{\multirow{2}{*}{\textbf{Regulation}}} & 
%  {\multirow{2}{*}{\textbf{Existential}}} &
%{\multirow{2}{*}{\textbf{Universal}}}  \\ 
%%	\cline{2-3} &  \cline{Connect & Timing & \\ \hline
%%\thickhline
% 
%1. & Boolean function & Boolean function & 2C \\ \hline
%2. & None & Boolean function & 3C \\ \hline
%3. & Boolean function & SNARE-SNARE inhibition & 4C \\ \hline
%4. & None & SNARE-SNARE inhibition & No graphs found \\ \hline
%5. & Boolean function & None & No graphs found \\ \hline
%6. &  None & None & No graphs found \\ \hline
%
%%6. & Arbitrary Boolean & Specificity matrix and every edge is distinct & N \& S = 3C\\ \hline
%
%\end{tabular}
%\label{tab:smt-grph}
%%\end{adjustwidth}
%\end{table}

\begin{table}[t]
	\centering
	\caption{{\bf SNARE regulation and graph connectedness.}}
%	\adjustbox{max height=\dimexpr\textheight-5.5cm\relax,
%		max width=\textwidth}{
		\begin{tabular}[t]{|c|c|c|c|c|c|c|}\hline
			
			{\multirow{2}{*} {Sr.No}}  & \multicolumn{2}{c|}{Regulation} & \multicolumn{2}{c|}{Existential} & \multicolumn{2}{c|}{Universal} 
			
		%	\\\cline{2-7}
			
%			{} & \multicolumn{2}{c|}{\textbf{2-connected} } & \multicolumn{2}{c|}{\textbf{3-connected} } & \multicolumn{2}{c|}{\textbf{2-connected}}  &  \multicolumn{2}{c|}{\textbf{4-connected}}

			%{} & \multicolumn{2}{c|}{2-connected} & \multicolumn{2}{c|}{3-connected} & \multicolumn{2}{c|}{2-connected}  &  \multicolumn{2}{c|}{4-connected}
			
			\\\cline{2-7}
			{} & {Node} & {Edge} & {Connectivity} & {Timing} & {Connectivity} & {Timing}
			
			\\\hline
			1. & Bf & Bf & 2C & 2.12 & 3C & \cellcolor{red!25}!1.89  \\\hline
			2. & None & Bf & 3C & 2.12 & 3C & \cellcolor{red!25}!1.89  \\\hline
			3. & Bf & SNARE inhib & 4C & 7.65 & 4C & 7.66 \\\hline
			4. & None & SNARE inhib & No graphs & 22.74 & 2.85 & 48.35 \\\hline
			5. & Bf & None & No graphs & 500.03 & No graphs & 890.84 \\\hline
			6. &  None & None & No graphs  & M/O & No graphs & M/O \\\hline
			
	\end{tabular}
%}
	% \caption{Run-times for searching for models (in secs).}
	\label{tab:sat-graph}
\end{table}

%\input{tab:satqbf-grabh}
%\ExecuteMetaData[result-table.tex]{tab:satqbf-grabh}

\textbf{{\sattool} (bounded model checker based tool)}:
The enumeration encoding (presented in section 6.2) for each variants of VTS is implemented in {\sattool}.
	%We have implemented the encodings for different variants of VTS (presented in section 6.2) using
The tool is based on CBMC, a successful bounded model checker, for C and C++ programs. 
%
%Unlike a general model checker~\cite{cimatti2002nusmv}, CBMC
%
It verify property for all the possible states, up to a certain “depth,” a parametric limit on the size of the model.
%
The depth of model is specified by the user and can be  incrementally increased to a desired value.
%
%The CBMC architecture consists of a C-language front-end (CFE) and a SAT Solver back-end (SAT). 
%
%CBMC front end accepts 

%We encode the problem using CBMC C-language front-end (CFE) interface with some special annotations called \textit{assumes} and \textit{assertions}.
%
%These annotations express constraints on the model and properties to be checked. 
%	
The vesicle transport network was modelled as a non-deterministic C-program manipulating an annotated graph. 
%
The special annotations \textit{assumes} and \textit{assertions} of CBMC's front end were used to express constraints on the model and properties to be checked. 
%
%The fusion rules and steady state properties were expressed as constraints using special annotations \textit{assumes}. 
%		
%The correspondence relationship between connectedness conditions and steady state were expressed as annotations \textit{assertions} to be checked. 
Using the tool's interface user can select the variation of the VTS, connectivity of the graph and other required parameters like an exploration depth.
%
The {\sattool} verifies the properties by first converting the constraints and properties into a Boolean formula and determine the satisfiability of the formula with the help of a SAT solver backend.
%
The satisfying model is translated into a human readable format and in case of a violation, the tool produces a readable counterexample.
%  following steps:
%	\begin{itemize}
%	\item Convert (using CFE) the model and the properties into a Boolean formula (verification condition) such that the property is true of all behaviors of the model up to the specified
%depth iff the Boolean formula is valid.
%	\item Check (using a SAT solver) validity of the verification condition. {\sattool} reports successful verification if the formula is found to be valid. If not, then {\sattool} produces a counterexample,
%an assignment to the variables of the model, that is a witness to the violation of the property.
%	\end{itemize}
For our experimentation purpose, the SAT solver backend used were MiniSat~\cite{sorensson2005minisat}, and CryptoMiniSat~\cite{soos2016cryptominisat}.
% the winner of SAT 2015 Competition~\cite{balyo2016sat}. 

%CBMC has a built-in SAT solver called MiniSat~\cite{sorensson2005minisat}, but it is also possible to use various other SAT solver blackboxes for property verification. Besides using MiniSat as a default SAT solver for our model, we have used different SAT solvers for the verification of the property, particularly CryptoMiniSat~\cite{soos2016cryptominisat} which was the winner of SAT 2015 Competition~\cite{balyo2016sat}. 
%		MiniSat performed satisfactorily in comparison to other SAT solvers.

\textbf{{\smttool} (SMT based tool)}:
The recursive encoding (presented in section 6.1)  
%We have implemented the encodings 
for each variant is implemented in {\smttool}.
%
The tool is based on industrial strength SMT solver, Z3. 
%
The tool allows the user to choose a model and the size
of the network besides other parameters like connectivity and number of parallel edges. 
%		
It uses Z3 Python interface to build the needed constraints and applies Z3 solver on the constraints to find a model (a satisfying assignment that follows the constraints). 
%
This tool also translates the satisfying model found by Z3 into
a VTS and represent a visual output to the user in form of coloured annotated graph. 
%
%We also visually report the dropped edges required to disconnect the graph, it gives information about the connectivity of the graph.
%
 

\textbf{{\qbftool} (QBF based universal condition):}
We have implemented the encodings of the VTS in QBF form (presented in section 6.5) in
{\qbftool}. 
%
The tool is developed in C++ and uses~\zthree~\cite{z3} infrastructure to encode the constraints.
%
Since some of the formulas involve alternation of quantifiers over Boolean variables Z3 is not a suitable choice for those examples.
%
We translate the formulas created by Z3 tool into a standard
QDIMACS~\cite{qdimacs} format and use QBF solvers to solve the formula. 
%
We have used~\depqbf~\cite{lonsing2010depqbf} for solving QBF formulas with alternation of quantifiers. 

\textbf{{\ourtool} (QBF based synthesis tool):}
\begin{sidewaysfigure}[t]
	\centering
	\begin{tabular}[t]{|c@{}|@{}c@{}|@{}c@{}|@{}c@{}|@{}c@{}|@{}c@{}|@{}c@{}|@{}c@{}|@{}c@{}|@{}c@{}|@{}C{4cm}@{}|}\hline
		% \begin{table}[t]
		%   \centering
		%   \begin{tabular}[t]{|c@{}|@{}c@{}|@{}c@{}|@{}c@{}|@{}c@{}|@{}c@{}|@{}c@{}|@{}c@{}|@{}c@{}|@{}c@{}|@{}c@{}|}\hline
		{\multirow{2}{*} \textbf{}}  & \multicolumn{2}{c|}{\textbf{Add}} & \multicolumn{2}{c|}{\textbf{Add}} & \multicolumn{2}{c|}{\textbf{Learning NNF}}  &  \multicolumn{2}{c|}{\textbf{Learning}} &  \multicolumn{2}{c|}{\textbf{Add/Delete}} \\
		{\multirow{2}{*} \textbf{Table a}}  & \multicolumn{2}{c|}{\textbf{edge}} & \multicolumn{2}{c|}{\textbf{molecules}} & \multicolumn{2}{c|}{\textbf{(only $\land$ and $\lor$)}}  &  \multicolumn{2}{c|}{\textbf{k-CNF}} &  \multicolumn{2}{c|}{\textbf{parts}} \\
		\cline{2-11}
		{} & {\textbf{Time}} & {\textbf{\#C}} & {\textbf{Time}} & {\textbf{\#C}} & {\textbf{Time}} & {\textbf{\#C}} & {\textbf{Time}} & {\textbf{\#C}} & {\textbf{Time}} & {\textbf{\#C}} \\
		\hline
		
		plos1-dia[3C]& 0.02 &$\infty$& 0.37 &$\infty$& 0.46 & $\infty$ & 0.63 &$\infty$& 0.21 & -1 E, -1AE, -1AN. +1E, +1N. \\\hline
		plos2-dia[4C] & 2.02 & 0   & 2.38 & 0  & 4.27  & 0 & 5.94 & 0 &  3.30 & 0 \\\hline
		sub-mammal[3C]  & 4.61 & 1 E  & 5.98 & 7 PE & 27.33 & 1 E & 14.75 & 1 E & 6.82  & -1E, -2PE, -4AN. +1 E, +5PE, +4N, +3AN, +3AE. \\\hline
		node4[3C]  & 5.93  & 1 E   &  13.72 & 12 PE  &  5.09  & $\infty$ & 4.30 &$\infty$& 7.21  & -2 E, -2PE, -1 N, -1AN, -1 AE. +18N, +8E, +1PE.\\\hline
		%   yeast-graph[3C]   & 95.016    & 1 E   & 94.520   & 1 E   & 169.430  & 1 E & 172.35   & 1E   & 107.43  &  -1 E, -2 N, -2 AE, -2 AN. +2 E, 12 PE, 7 N. \\\hline
		yeast-graph[3C]   & 105.15    & 2 E  &   247.04  & 15 PE   & 1381.24  & 2 E  & 541.66   & 2 E & 326.09  &  -1E, -1N, -1AE, -1AN, -1PE. +2E, +11PE, +20N. \\\hline
	   	node6[3C]  &  454.98  & 2 E  &  1209.30    & 21 PE    &  T/O  & N/A      &  T/O   &  N/A   &  T/O     & N/A \\\hline
		mammal-graph[3C]  &  T/O     & N/A  &  T/O     & N/A    &  T/O         & N/A      &  T/O    &  N/A    &  T/O     & N/A\\\hline
		%    & 0.0    & 0.0    & 0.0    & 0.0    & 0.0         & 0.0      & 0.0   & 0.0    & 0.0    & 0.0\\\hline
	\end{tabular}
	% \caption{Run-times for searching for models (in secs). \#C  stands for minimum changes.
	% Time is reported in seconds.}
	% \label{tab:qf-graph}
	% \end{table}
	\begin{tabular}[t]{|c@{}|@{}c@{}|@{}c@{}|@{}c@{}|@{}c@{}|@{}c@{}|@{}c@{}|@{}c@{}|@{}c@{}|@{}c@{}|@{}C{4cm}@{}|}\hline
		{\multirow{2}{*} \textbf{}}  & \multicolumn{2}{c|}{\textbf{Add}} & \multicolumn{2}{c|}{\textbf{Add}} & \multicolumn{2}{c|}{\textbf{Learning NNF}}  &  \multicolumn{2}{c|}{\textbf{Learning}} &  \multicolumn{2}{c|}{\textbf{Add/Delete}} \\
		{\multirow{2}{*} \textbf{Table b}}  & \multicolumn{2}{c|}{\textbf{edge}} & \multicolumn{2}{c|}{\textbf{molecules}} & \multicolumn{2}{c|}{\textbf{(only $\land$ and $\lor$)}}  &  \multicolumn{2}{c|}{\textbf{k-CNF}} &  \multicolumn{2}{c|}{\textbf{parts}} \\
		\cline{2-11}
		{} & {\textbf{Time}} & {\textbf{\#C}} & {\textbf{Time}} & {\textbf{\#C}} & {\textbf{Time}} & {\textbf{\#C}} & {\textbf{Time}} & {\textbf{\#C}} & {\textbf{Time}} & {\textbf{\#C}} \\
		\hline
		
		plos1-dia & 0.04&$\infty$& 0.31 &$\infty$& 0.20 & $\infty$ & 0.27&$\infty$& 3.47 & -1E, -1PE, - 1N, -1PE. +1AE, +1PE, +1N\\\hline
		plos2-dia & 3.97 & 0 &  2.26 & 0  & 5.46 & 0 & 5.35 & 0 & 3.24 & 0 \\\hline
		sub-mammal & 3.22 & 1 E  & 4.18 & 7 PE  & 24.73 & 1 E  & 9.21 & 1 E & 5.36  & -1E, -2PE, -4AN. +1 E, +5PE, +4N, +3AN, +3AE \\\hline
		node4  & 4.01  & 1 E  & 10.19  & 12 PE & 4.26  & $\infty$ & 4.13 &$\infty$&  6.89  & -2 E, -2PE, -1N, -1AN, -1AE. +18N, +8E, +1PE \\\hline
		yeast-graph & 38.68  & 2 E  &   127.45  & 15 PE   & 1002.41  & 2 E  & 434.08   & 2 E & 197.82  &  -1E, -1N, -1AE, -1AN, -1PE. +2E, +11PE, +20N. \\\hline
		node6  &  235.97  & 2 E  &  549.25    & 21 PE    &  T/O  & N/A      &  T/O   &  N/A   &  T/O     & N/A \\\hline
		mammal-graph   &  T/O     & N/A  &  T/O    & N/A    &  T/O  & N/A      &  T/O   &  N/A   &  T/O     & N/A\\\hline
	\end{tabular}
	\caption{Run-times for synthesis queries. \#C stands for minimum changes in the synthesized VTS in comparison with the given partial VTS. Time is reported in seconds. (a) The solver used is DepQBF (b) The solver used is Z3. The sub-mammal is a subgraph of the complete mammal-graph. In the Add/Delete parts column, ‘+’n sign is used to show the addition of n number of the molecules, similarly ‘-’n is used to show the removal of n number of molecules. In the table, N is node labels, AN is active node molecules, E is edges, PE is molecule presence on the edge and AE is active molecules on the edge. The [kC] stands for k graph connectedness which is part of only DepQBF experiments.}
	
	% \caption{Run-times for searching for Z3 models (in secs). \#C  stands for minimum changes.
	% Time is reported in seconds.}
	\label{tab:synth-graph}
\end{sidewaysfigure}

We have implemented the synthesis encodings (presented in section 6.6) in~\ourtool\footnote{{\url{https://github.com/arey0pushpa/pyZ3}}}.
%
The tool takes a partially defined VTS as input in a custom designed
input language.
%
The input is then converted to the constraints over VTS using the {\qbftool} interface. 
%
The tool can synthesize the queries discussed earlier and also their combinations.
%The tool can not only synthesize the discussed queries, but also their combinations.
%
For example, our tool can modify labels of nodes or edges while
learning activity functions.
%
The tool returns a complete VTS with visual graph representation as an output to the user.  

\subsection{Results}
The results and timings of three blocks of the {\vtstool} are presented in the table~\ref{tab:smt-grph} and table~\ref{tab:satsmt-graph}, for both {\sattool} and {\smttool}, in table~\ref{tab:qbf-graph} for {\qbftool} and in table~\ref{tab:synth-graph} for~{\ourtool}.


%To illustrate usability of our tool in the last column of the Table~\ref{tab:smt-grph}, we present the minimum connectedness needed for the different variants after applying our tool for sizes from 2 to 10. 
%%
%We found no graph for the variant A with constraints Ann and Aen. 
%%
%Replacing constraint on the node of Ann with Anb (variant B)
%does not affect the outcome. 
%%
%If we allow every present molecule to stay active
%(Ann) but constraint the edge by a boolean function (Aeb) the resultant VTS has to be at least 3-connected. 
%%
%Similarly, the results for other cases are presented in the table.



\begin{table}[t]
	\centering
    \def\arraystretch{1.2}
	\begin{tabular}{ |c|c|c|c| }
		\hline
		{\multirow{2}{*}{Variant}}  &
		\multicolumn{2}{c|} {Constraints} & 
	%	{\multirow{2}{*}{\textbf{Results}}}
	    {\multirow{2}{*}{Graph connectivity }} 
		\\
		\cline{2-3}
		&  Rest & Activity & 
	\\ \hline
	
	A. & \multirow{5}{*}{
		\makecell{ \ref{eq:f0}--\ref{eq:fuse2},\\
			\ref{eq:reach1},\ref{eq:reach2},\\
			\ref{eq:drop1}--\ref{eq:drop4}}
	}
	& \ref{eq:ann},\ref{eq:aen} & No graph   \\
	B. & & \ref{eq:anb},\ref{eq:aen} & No graph \\
	C. & & \ref{eq:ann},\ref{eq:aeb} & 3-connected \\
	D. & & \ref{eq:anb},\ref{eq:aeb} & 2-connected \\
	E. & & \ref{eq:ann},\ref{eq:aep} & No graph    \\
	F. & & \ref{eq:anb},\ref{eq:aep} & 4-connected \\\hline
	
	% C_es :Self edges are allowed
	% C_ed : Every edge is distinct 
\end{tabular}
\caption{ Activity regulation of molecules vs. graph connectivity.
}
\label{tab:smt-grph}
\end{table}

\begin{table}[t]
	\centering
	\adjustbox{max height=\dimexpr\textheight-5.5cm\relax,
		max width=\textwidth}{
		\begin{tabular}[t]{ c|c c|c c|c c|c c}\hline
			
			{\multirow{2}{*} {Size}}  & \multicolumn{2}{c|}{Variant A} & \multicolumn{2}{c|}{Variant C} & \multicolumn{2}{c|}{Variant D}  &  \multicolumn{2}{c}{Variant F} \\ 
			%\cline{2-9}
			
			{} & \multicolumn{2}{c|} {2-connected}  & \multicolumn{2}{c|} {3-connected}  & \multicolumn{2}{c|} {2-connected}  &  \multicolumn{2}{c} {4-connected}
			%{} & \multicolumn{2}{c|}{2-connected} & \multicolumn{2}{c|}{3-connected} & \multicolumn{2}{c|}{2-connected}  &  \multicolumn{2}{c|}{4-connected}
			\\
			%\cline{2-9}
			{} & {\sattool}  & {\smttool} &  {\sattool}  & {\smttool} &  {\sattool}  & {\smttool} &  {\sattool}  & {\smttool} \\\hline
			2 & M/O & !0.0569 &  M/O & 0.050 & !0.13 & !1.89 & 0.35 & 5.12 \\\hline
			3 & M/O & !0.364 &  M/O  & 0.336 & 0.62 &  0.345  & 1.36 & 23.94\\\hline
			4 & M/O & !1.789 &  M/O & 1.641 & 2.85 & 1.599  & 4.81 & 123.34\\\hline
			5 & M/O & !6.278 & M/O & 5.589 & 10.27 & 890.84 & 33.36  & 2482.71 \\\hline
			6 & M/O & !22.642 & M/O & 17.295 & 30.81 & M/O  & 147.52 & M/O\\\hline
			7 & M/O & !60.243 &  M/O &   37.725 & & M/O & 522.26  & M/O \\\hline
			8 & M/O & !195.64 &  M/O &  185.457 & 303.19 & M/O & 2142.76 & M/O\\\hline
			9 & M/O & !378.93 &  M/O &  499.414 & 971.01 & M/O & 4243.34 & M/O\\ \hline
			10 & M/O & !1029.79 & M/O & 1977.720 & M/0 & 288.780 & 7786.82 & M/O\\\hline
	\end{tabular}}

	\caption{{\sattool} and {\smttool} run-times for searching for models (in secs).}
	
	\label{tab:satsmt-graph}
\end{table}


The minimal connectivity results for each variation of the existential condition are presented in table~\ref{tab:smt-grph}.
%
%No matter what the rules of molecular interactions and regulation, steady state requires the graph structure to be at least strongly connected, and therefore 2-connected.
The steady-state constraint requires that structure of the graph is at least 2-connected. 
%
We found that 3-connectedness is the existential condition for the Boolean regulation model on vesicles, while 4-connectedness is the existential condition for the more constrained SNARE-SNARE inhibition on vesicles and Boolean regulation on compartments.
%Beyond this, we find that 3-connectedness is existential condition for the Boolean regulation model on vesicles, while 4-connectedness is existential condition for the more constrained SNARE-SNARE inhibition on vesicles and Boolean regulation on compartments.
%
For all other constrained regulation models of VTS, the experimentation was unable to find any graphs representing the vesicle transport networks in the stable state.
%(including the case where SNAREs are always active) we did not find any possible graphs representing stable vesicle transport networks. 
%
This strongly suggests that no such graphs exist, as our graphs are of finite size, this does not amount to a  proof. 

For biological cells, the steady state condition implies that every compartment that gives a vesicle necessarily receives a vesicle too.
%
On top of this, for the case where traffic molecules are regulated in some form, every pair of compartments with a vesicle going between them, have two additional vesicle paths connecting them; where these two paths go in opposite directions.
%
Finally, when we take into account the exact manner in which SNAREs interact in real cells, we find that the biological VTS must conform to a very specific structure; any two compartments that are connected by a vesicle, must be connected by 4 vesicle paths, two of them going in the same direction, and the other two in the opposite direction. 
%
No graph results suggests that a functional VTS necessarily needs some form of regulation, where the molecules of the VTS are kept inactive in places where they might disrupt other interactions

%This strongly suggests (but does not prove, since we only checked graphs of finite size) that no such graphs exist.
%We have found no graphs for Variants A where no regulation is enforced on the activity of the molecules both on the edges and nodes.
%%
%Two other cases Variant B, Variant E also results in the no graph satisfying the constraints.
%%  
%2-connectedness is the existential condition for Variants D where activity of the molecule is dependent on the presence of the other molecules. 
%%
%Variants C has 3-connectedness as


%
Table~\ref{tab:satsmt-graph} presents the comparison of the running times of {\smttool} and {\sattool} on VTS variations of sizes 2 to 10.
%
For this comparison we have fixed the total number of molecules ($|M|$) to be $2|N|$ 
%for $ |N|> 2$ and $2|N| + 1$ for $|N| = 2$
%
%For each variant, we fix 1
and maximum number of parallel
edges to be 2.
%
%
In the table ``!'' indicate that no model exists for the constraint  (constraints were unsatisfiable).
%In the table we have shown comparison for specific connectivity, for example variant A is checked against any graph with connectivity 2, variant B with connectivity 3 similarly for the rest of the variants.
%
%We have compared our performance with the performance of our earlier CBMC based implementation (old-encoding).
%
As an example, the result of verifying the formula for variation F of 2-connected, 10 compartment graph took 129.78 minutes (7786.8 secs) with a satisfying result.
%
In comparison, {\sattool} was unable to solve the instance and ran OUT OF MEMORY.
% for $|N|$ greater than 5.
%
% 
The recursive encoding also performed better for the cases where the instance was unsatisfiable (table~\ref{tab:satsmt-graph} Variant A timing comparison and Variant D with N =2).
%instead of enumeration, not only we got efficiency improved for finding a SAT model but also did better in the case of refutation that no model exist (table~\ref{tab:satsmt-graph} Variant A timing comparison and Variant D with N =2). 
%
Hence with the use of improved encoding of reachability, we were able to scale the system to the biologically relevant compartment size, the
%a much larger compartmentalized system, especially to interesting 
the case of eukaryotic cells with a total number of 10 compartments.
%
Furthermore, we experimented with limits of our tools and found that {\smttool} was able to solve the constraints up to $20$ nodes.

\begin{table}[t]
	\centering
	\def\arraystretch{1.2}
%	\begin{tabular}{ c|c|c|c }
%		\hline
%		{\multirow{2}{*}{Variant}}  &
%		\multicolumn{2}{c|} {Constraints} & 
%		%	{\multirow{2}{*}{\textbf{Results}}}
%		{\multirow{2}{*}{Graph connectivity }} 
%		\\
%		\cline{2-3}
%		&  Rest & Activity & 
%		\\ \hline
%		
%		A. & \multirow{5}{*}{
%			\makecell{ \texttt{Consistancy}, \\  \texttt{Stability}, \\
%				 \texttt{Connected(k)}  }
%		}
%		& \ref{eq:ann},\ref{eq:aen} & No graph   \\
%		B. & & \ref{eq:anb},\ref{eq:aen} & No graph \\
%		C. & & \ref{eq:ann},\ref{eq:aeb} & 3-connected \\
%		D. & & \ref{eq:anb},\ref{eq:aeb} & 3-connected \\
%		E. & & \ref{eq:ann},\ref{eq:aep} & No graph    \\
%		F. & & \ref{eq:anb},\ref{eq:aep} & 4-connected \\\hline
%		
%		% C_es :Self edges are allowed
%		% C_ed : Every edge is distinct 
%	\end{tabular}
	\begin{tabular}{ c|c | c| c|c }
	\hline
	{\multirow{3}{*}{Variant}}  &
	\multicolumn{3}{c|} {Constraints} & 
	%	{\multirow{2}{*}{\textbf{Results}}}
	{\multirow{3}{*}{Graph connectivity} } 
	\\
	\cline{2-4}
	& 	{\multirow{2}{*}{Rest} }  & 	\multicolumn{2}{c|} {Activity} & 
	\\ 
	\cline{3-4}
	&  & Node  & Edge & 
	\\\hline
	
	A. & \multirow{5}{*}{
					\makecell{ \texttt{Consistancy}, \\  \texttt{QStability}, \\
						 \texttt{QConnected(k)}  }
	}
	& \texttt{FunOnN} & \texttt{FunOnE}  & No graph \\
B. & & \texttt{FunBfN} & \texttt{FunOnE}  & No graph  \\
C. & & \texttt{FunOnN} &  \texttt{FunBfE} & 3-connected  \\
D. & & \texttt{FunBfN} & \texttt{FunBfE} & 3-connected\\
E. & & \texttt{FunOnN} & \texttt{FunPairE}  & No graph  \\
F. & & \texttt{FunBfN} &  \texttt{FunPairE} & 4-connected  \\\hline
	
	% C_es :Self edges are allowed
	% C_ed : Every edge is distinct 
\end{tabular}
	\caption{Universal condition result for the graph connectivity}
	\label{tab:qbf-graph}
\end{table}

%\textbf{Universal Condition:}

Table~\ref{tab:qbf-graph} presents the result of the QBF tool.
%
The table represents the result of the graph connectivity for each variation of the VTS.  
%
For example, we found that the 3-connected graph satisfies the universal condition for the variation 2. 
%
Note that for this variation existential condition result was the 2-connected graph. 
%
For every other version, the results remained the same.
This implies that for every 3-connected graph, a valid VTS that satisfies all constraints \todo{ elaborate a bit on the constraints} exists.
% 
For biological cells, this means that the whole space of 3-connected graphs is available for evolutionary exploration
%
%The experimentations were done using two QBF solvers: \depqbf and \rarqbf.


%(there is a satisfying assignments to satisfy constraints of VTS) is valid VTS.   
%\textbf{Synthesis}:

We have applied~\ourtool~on six partially defined VTSs.
%
The results are presented in Fig~\ref{tab:synth-graph} for both the solvers
\depqbf and \zthree.
%
To use~\zthree, we remove \texttt{Connected} constraints, such that the queries becomes
quantifier-free.
%
% The experiments were conducted on a machine,
% with \ashu{?}MHz processor, \ashu{?}GB memory, and 900s timeout.
%
%The experiments were conducted with a 900s timeout.
%
The first four VTSs are synthetic but inspire from literature for
typical motifs in VTSs. 
The third VTS is a subgraph of the last VTS.
%
%
The fifth VTS is taken from~\cite{burri2004complete}.
%
The last VTS represent mammalian SNARE map created by studying the literature references.  
%
The fig shows timing for various synthesis queries.
%
For each synthesis query, we have two columns.
%
One column reports the timing and the other reports the minimum changes needed to obtain a valid VTS.
%
$\infty$ indicates that any number of changes with the synthesis query
search space can obtain the VTS.
%
%\ashu{@ankit: please discuss all the synthesis queries in the table.}

In the Fig~\ref{tab:synth-graph}, we report five synthesis queries.
%
The first one only adds new labelled edges to the graph.
%
We have ranked the all possible graph edits with the simple rank of
minimum updates.
% %
% Our tools were able to complete the graphs by adding five new edges
% and nine new molecules in 210s.
%
The second query adds new labels to the edge.
%  and was able to
% fix the graphs with eleven new molecule labels in 180s.
%
The third query synthesizes NNF Boolean functions only containing
$\land$ and $\lor$ gates for activity functions, while allowing
more edges to be added.
%
The result shows the basic template of 4 leaves and 3 gates.
%
% The tool was able to synthesize the Boolean functions in 470s.
%
To illustrate the versatility of our tool, the fourth query
synthesizes $3$-CNF functions (encoding not presented).
%  and was able to
% synthesis the results for all graphs in 500s.
%
Finally, we report queries that allows both addition and deletion of edges, and labels
of node and labels. 

%
%Our experiments suggest that the synthesis problems are solvable by modern 
%solvers and the synthesis technology may be useful for biological research.
%

%%% Local Variables:
%%% mode: latex
%%% TeX-master: "main"
%%% End:

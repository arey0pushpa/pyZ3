We have developed three different tools (\sattool, \smttool, \ourtool) for implementation of proposed encodings. {\sattool} is based on the architecture of a Bounded model checker (CBMC) and uses an enumeration of paths for encoding the reachability problem. {\smttool} is based on the SMT solver Z3 and uses the efficient encoding of finding a least fixed point to solve reachability problem. The {\ourtool} is a synthesis tool for synthesizing incomplete information in the input VTS. It is based on the QBF encoding and uses {\depqbf} solver to solve the generated QBF. All the experiments were done on a machine with Intel(R) Core(TM) i3-4030U CPU @ 1.90GHz processor and 4GB RAM with 30 min (1800 sec) timeout.

\begin{table}[!ht]
%\begin{adjustwidth}{-2.25in}{0in} % Comment out/remove adjustwidth environment if table fits in text column.
\centering
\def\arraystretch{1.6}
\caption{
{\bf SNARE regulation and graph connectedness.}}
\begin{tabular}{|c|l|l|c|}
\hline
\bf{Sr.No} & \multicolumn{1}{c|}{\bf{Regulation on node}} & \bf{Regulation on edge} & \bf{Connectedness}\\ \thickhline
1. & Boolean function & Boolean function & N = 2C S = 3C \\ \hline
2. & None & Boolean function & N \& S = 3C\\ \hline
3. & Boolean function & SNARE-SNARE inhibition & N \& S = 4C\\ \hline
4. & None & SNARE-SNARE inhibition & No graphs found\\ \hline
5. & Boolean function & None & No graphs found\\ \hline
6. &  None & None & No graphs found\\ \hline

%6. & Arbitrary Boolean & Specificity matrix and every edge is distinct & N \& S = 3C\\ \hline

\end{tabular}
\label{tab:smt-grph}
%\end{adjustwidth}
\end{table}

\begin{table}[t]
	\centering
	% \def\arraystretch{1.6}
	\begin{tabular}{|c|c|c|c|}
		\hline
		{\multirow{2}{*}{\textbf{Variant}}}  &
		\multicolumn{2}{c|}{\textbf{Constraints}} &
		{\multirow{2}{*}{\textbf{Graph connectivity}}}
		% \multicolumn{2}{c|}{\textbf{Graph connectivity}}
		\\
		\cline{2-3}
		&  \textbf{Rest} & \textbf{Activity} &  % & \textbf{Sampling: Guarantee}
	\\ \hline
	
	A. & \multirow{5}{*}{
		\makecell{ \ref{eq:f0}--\ref{eq:fuse2},\\
			\ref{eq:reach1},\ref{eq:reach2},\\
			\ref{eq:drop1}--\ref{eq:drop4}}
	}
	& \ref{eq:ann},\ref{eq:aen} & No graph     \\\cline{3-4}
	B. & & \ref{eq:anb},\ref{eq:aen} & No graph     \\\cline{3-4}
	C. & & \ref{eq:ann},\ref{eq:aeb} & 3-connected  \\\cline{3-4}
	D. & & \ref{eq:anb},\ref{eq:aeb} & 2-connected  \\\cline{3-4}
	E. & & \ref{eq:ann},\ref{eq:aep} & No graph     \\\cline{3-4}
	F. & & \ref{eq:anb},\ref{eq:aep} & 4-connected  \\\hline
	
	% C_es :Self edges are allowed
	% C_ed : Every edge is distinct 
\end{tabular}
\caption{{\bf Activity regulation of molecules vs. graph connectivity.
}}
\label{tab:smt-grph}
\end{table}

\begin{sidewaysfigure}[t]
	\centering
	\begin{tabular}[t]{|c@{}|@{}c@{}|@{}c@{}|@{}c@{}|@{}c@{}|@{}c@{}|@{}c@{}|@{}c@{}|@{}c@{}|@{}c@{}|@{}C{4cm}@{}|}\hline
		% \begin{table}[t]
		%   \centering
		%   \begin{tabular}[t]{|c@{}|@{}c@{}|@{}c@{}|@{}c@{}|@{}c@{}|@{}c@{}|@{}c@{}|@{}c@{}|@{}c@{}|@{}c@{}|@{}c@{}|}\hline
		{\multirow{2}{*} \textbf{}}  & \multicolumn{2}{c|}{\textbf{Add}} & \multicolumn{2}{c|}{\textbf{Add}} & \multicolumn{2}{c|}{\textbf{Learning NNF}}  &  \multicolumn{2}{c|}{\textbf{Learning}} &  \multicolumn{2}{c|}{\textbf{Add/Delete}} \\
		{\multirow{2}{*} \textbf{Table a}}  & \multicolumn{2}{c|}{\textbf{edge}} & \multicolumn{2}{c|}{\textbf{molecules}} & \multicolumn{2}{c|}{\textbf{(only $\land$ and $\lor$)}}  &  \multicolumn{2}{c|}{\textbf{k-CNF}} &  \multicolumn{2}{c|}{\textbf{parts}} \\
		\cline{2-11}
		{} & {\textbf{Time}} & {\textbf{\#C}} & {\textbf{Time}} & {\textbf{\#C}} & {\textbf{Time}} & {\textbf{\#C}} & {\textbf{Time}} & {\textbf{\#C}} & {\textbf{Time}} & {\textbf{\#C}} \\
		\hline
		
		plos1-dia[3C]& 0.326 &$\infty$& 0.312 &$\infty$& 0.669 & $\infty$ & 0.966 &$\infty$& 0.277 & -1 E, -1 AE, -1 AN. +1 E, +1 N. \\\hline
		plos2-dia[4C] & 0.266 & 0   & 0.322 & 0  & 1.409  & 0 & 2.114 & 0 &  0.337 & 0 \\\hline
		sub-mammal[3C]  & 0.767 & 1 E  & 1.049 & 5 PE & 3.523 & 1E & 4.961 & 1E & 1.172  & -1 E, -2 PE, -1 AN. +1 E, +4 PE, +4 N, +2 AN, +2 AE. \\\hline
		node4[3C]  & 1.554  & 1 E   &  3.859 & 12 PE  &  5.286  & $\infty$ & 4.502 &$\infty$& 2.194  & -2 E, -2 PE, -1 N, -1 AN, -1 AE. +12 N, +8 E, +1 PE.\\\hline
		%   yeast-graph[3C]   & 95.016    & 1 E   & 94.520   & 1 E   & 169.430  & 1 E & 172.35   & 1E   & 107.43  &  -1 E, -2 N, -2 AE, -2 AN. +2 E, 12 PE, 7 N. \\\hline
		yeast-graph[3C]   & 95.016    & 2 E  &   timeout  & N/A   & 1571.42  & 2 E  & 530.210   & 2 E & 72.316  &  -1 E, -1 N, -1 AE, -1 AN, -1PE. +2 E, 7 PE, 8 N. \\\hline
		
		mammal-graph[3C]  &  timeout     & N/A  &  timeout     & N/A    &  timeout         & N/A      &  timeout    &  N/A    &  timeout     & N/A\\\hline
		%    & 0.0    & 0.0    & 0.0    & 0.0    & 0.0         & 0.0      & 0.0   & 0.0    & 0.0    & 0.0\\\hline
	\end{tabular}
	% \caption{Run-times for searching for models (in secs). \#C  stands for minimum changes.
	% Time is reported in seconds.}
	% \label{tab:qf-graph}
	% \end{table}
	\begin{tabular}[t]{|c@{}|@{}c@{}|@{}c@{}|@{}c@{}|@{}c@{}|@{}c@{}|@{}c@{}|@{}c@{}|@{}c@{}|@{}c@{}|@{}C{4cm}@{}|}\hline
		{\multirow{2}{*} \textbf{}}  & \multicolumn{2}{c|}{\textbf{Add}} & \multicolumn{2}{c|}{\textbf{Add}} & \multicolumn{2}{c|}{\textbf{Learning NNF}}  &  \multicolumn{2}{c|}{\textbf{Learning}} &  \multicolumn{2}{c|}{\textbf{Add/Delete}} \\
		{\multirow{2}{*} \textbf{Table b}}  & \multicolumn{2}{c|}{\textbf{edge}} & \multicolumn{2}{c|}{\textbf{molecules}} & \multicolumn{2}{c|}{\textbf{(only $\land$ and $\lor$)}}  &  \multicolumn{2}{c|}{\textbf{k-CNF}} &  \multicolumn{2}{c|}{\textbf{parts}} \\
		\cline{2-11}
		{} & {\textbf{Time}} & {\textbf{\#C}} & {\textbf{Time}} & {\textbf{\#C}} & {\textbf{Time}} & {\textbf{\#C}} & {\textbf{Time}} & {\textbf{\#C}} & {\textbf{Time}} & {\textbf{\#C}} \\
		\hline
		
		plos1-dia & 0.041&$\infty$& 0.320 &$\infty$& 0.225 & $\infty$ & 0.33&$\infty$& 3.74 & -1 E, -1 PE, - 1 N, -1 PE. +1 AE, +1 PE, +1 N\\\hline
		plos2-dia & 3.97 & 0 &  2.647 & 0  & 5.941 & 0 & 5.680 & 0 & 3.56 & 0 \\\hline
		sub-mammal & 3.483 & 1 E  & 4.379 & 5 PE  & 29.980 & 1 E  & 10.405 & 1 E & 3.650  & -1 E, -2 PE, -1 AN. +1 E, +4 PE, +4 N, +2 AN, +2 AE \\\hline
		node4  & 4.150  & 1 E  & 10.562  & 12 PE & 3.401  & $\infty$ & 4.760 &$\infty$&  5.05  & -2 E, -2 PE, -1 N, -1 AN, -1 AE. +12 N, +8 E, +1 PE \\\hline
		yeast-graph & 40.225  & 2 E  &   timeout  & N/A   & 1393.84  & 2 E  & 468.161   & 2 E & 69.81  &  -1 E, -1 N, -1 AE, -1 AN, -1PE. +2 E, 7 PE, 8 N. \\\hline
		mammal-graph   &  timeout     & N/A  &  timeout     & N/A    &  timeout         & N/A      &  timeout    &  N/A   &  timeout     & N/A\\\hline
	\end{tabular}
	\caption{Run-times for synthesis queries. \#C stands for minimum changes in the synthesized VTS in comparison with the given partial VTS. Time is reported in seconds. (a) The solver used is DepQBF (b) The solver used is Z3. The sub-mammal is a subgraph of the complete mammal-graph. In the Add/Delete parts column, ‘+’n sign is used to show the addition of n number of the molecules, similarly ‘-’n is used to show the removal of n number of molecules. In the table, N is node labels, AN is active node molecules, E is edges, PE is molecule presence on the edge and AE is active molecules on the edge. The [kC] stands for k graph connectedness which is part of only DepQBF experiments.}
	
	% \caption{Run-times for searching for Z3 models (in secs). \#C  stands for minimum changes.
	% Time is reported in seconds.}
	\label{tab:qbf-graph}
\end{sidewaysfigure}


\subsection{SATVTS (Bounded model checker based tool)} 
The first tool is implemented using model checking tool for C program CBMC~\cite{clarke2004tool}.
%
CBMC is a Bounded Model Checker for C and C++ programs. 
%
Unlike a general model checker~\cite{cimatti2002nusmv}, CBMC checks the verified property for all the possible states, only up to a certain “depth,” a parametric limit on the size of the model.
%
The depth of model is specified by the user and can be  incrementally increased to a desired value.

The CBMC architecture consists of a C-language front-end (CFE) and a SAT Solver back-end (SAT). 
%
CBMC front end accepts ANSI-C programs with some special annotations (called assumes and assertions) to express constraints on the model and properties to be checked. 
%		
In our experiments, the vesicle transport network was modelled as a non-deterministic C-program manipulating a graph with labelled edges with the fusion rules and steady state properties expressed as constraints using assumes. 
%		
The correspondence relationship between connectedness conditions
and guarantees for steady state were expressed as assertions to be checked. 
%		
Given an exploration depth provided by the user, which in our case corresponds to the size of the graph, CBMC verifies the properties by executing the following steps:
	\begin{itemize}
	\item Convert (using CFE) the model and the properties into a Boolean formula (verification condition) such that the property is true of all behaviors of the model up to the specified
depth iff the Boolean formula is valid.
	\item Check (using a SAT solver) validity of the verification condition. CBMC reports successful verification if the formula is found to be valid. If not, then CBMC produces a counterexample,
an assignment to the variables of the model, that is a witness to the violation of the property.
	\end{itemize}

CBMC has a built-in SAT solver called MiniSat~\cite{sorensson2005minisat}, but it is also possible to use various other SAT solver blackboxes for property verification. Besides using MiniSat as a default SAT solver for our model, we have used different SAT solvers for the verification of the property, particularly CryptoMiniSat~\cite{soos2016cryptominisat} which was the winner of SAT 2015 Competition~\cite{balyo2016sat}. 
%		
MiniSat performed satisfactorily in comparison to other SAT solvers.

\subsection{SMTVTS/pyVTS (SMT based tool)}
	We have implemented the encodings for each variants using the python interface of Z3 in a tool (MAA). 
%
Our tool allows the user to choose a model and the size
of the network besides other parameters like connectivity and number of parallel edges. 
%		
It uses Z3 Python interface to build the needed constraints and applies Z3 solver on the constraints to find a model ( a satisfying assignment that respects the constraints). 
%
This tool also translates the satisfying model found by Z3 into
a VTS and presents a visual output to the user in form of annotated graph. 
%
We also visually report the dropped edges required to disconnect the graph, it gives information about the connectivity of the graph.
%
To illustrate usability of our tool in the last column of the table 1, we present the minimum connectedness needed for the different variants after applying our tool for sizes from 2 to 10. 
%
We found no graph for the variant A with constraints Ann and Aen. 
%
Replacing constraint on the node of Ann with Anb (variant B)
does not affect the outcome. 
%
If we allow every present molecule to stay active
(Ann) but constraint the edge by a boolean function (Aeb) the resultant VTS has to be at least 3-connected. 
%
Similarly, the results for the other cases are presented in the table.

In Table~\ref{tab:qf-grabh}, we present the running times for the sear
ch of
VTSs of sizes 2 to 10 that satisfy the variants.
and compare with our old encoding
({Old-e}) from~\cite{shukla}.
%
For the comparison between both encodings we have fixed the total
number of molecules to be $|M| = 2|N|$ for $ |N|> 2$ and
$|M| = 2|N| + 1$ for $|N| = 2$.
%
For each variant, we fix maximum number of parallel
edges to 2.
%
In the table we have shown comparison for specific connectivity, for
example variant A is checked against any graph with connectivity 2,
variant B with connectivity 3 similarly for the rest of the Variants.
%

We have compared our performance with the performance of our earlier
CBMC based implementation (old-encoding).
%
For example, the formula for variation F, the total number of
compartments ($|N|$) equals to 10, returns in 129.78 minutes (7786.8 s
ecs)
with a SAT result.
%
In comparison, CBMC results in OUT OF MEMORY for $|N|$ greater than 5.
%
``!'' indicate that the constraints were unsatisfiable.
%
Using this encoding in comparison to the old one, not only we got effi
ciency improved for finding a SAT model but also did better in the cas
e of refutation that no model exist (Table 2 Variant A timing comparis
on and Variant D with N =2). 
%
Hence with the use of this novel encoding, we are able to scale the system to a much larger compartmentalized system, especially to
eukaryotic cells with a total number of 10 compartments.
%
Furthermore, we experimented with limits of our tools and found
that $\zthree$ was able to solve the constraints up to $\sim{14-18}$ n
odes.



\subsection{SynthVTS (QBF based synthesis tool)} 
We have implemented the encodings in a tool
called~\ourtool\footnote{{\url{https://github.com/arey0pushpa/pyZ3}}}.
%
The tool takes a partially defined VTS as input in a custom designed
input language.
%
The input is then converted to the constraints over VTS. 
%
The tool can not only synthesize the above-discussed queries, but also their
combinations.
%
For example, our tool can modify labels of nodes or edges while
learning activity functions.
%
Our tool is developed in C++ and uses~\zthree~\cite{z3} infrastructure for
processing formulas. 
%
Since some of the formulas involve alternation of quantifiers over
Boolean variables Z3 is not a suitable choice for those examples.
%
We translate the formulas created by Z3 tool into a standard
QDIMACS~\cite{qdimacs} format and use as an input for QBF solvers. 
%
We use~\depqbf~\cite{lonsing2010depqbf} for solving of QBF formulas. 
%
Our tool includes about 7000 lines of code.

We have applied~\ourtool~on six partially defined VTSs.
%
The results are presented in table~\ref{tab:qf-graph} for both the solvers
\depqbf and \zthree.
%
To use~\zthree, we remove \texttt{Connected} constraints, such that the queries becomes
quantifier-free.
%
% The experiments were conducted on a machine,
% with \ashu{?}MHz processor, \ashu{?}GB memory, and 900s timeout.
%
%The experiments were conducted with a 900s timeout.
%
The first four VTSs are synthetic but inspire from literature for
typical motifs in VTSs. 
The third VTS is a subgraph of the last VTS.
%
%
The fifth VTS is taken from~\cite{burri2004complete}.
%
The last VTS represent mammalian SNARE map created by studying the literature references.  

The table shows timing for various synthesis queries.
%
For each synthesis query, we have two columns.
%
One column reports the timing and the other reports the minimum changes
needed to obtain a valid VTS.
%
$\infty$ indicates that any number of changes with the synthesis query
search space can obtain the VTS.
%
%\ashu{@ankit: please discuss all the synthesis queries in the table.}
In the table, we are reporting five synthesis queries
%
The first one only adds new labelled edges to the graph.
%
We have ranked the all possible graph edits with the simple rank of
minimum updates.
% %
% Our tools were able to complete the graphs by adding five new edges
% and nine new molecules in 210s.
%
The second query adds new labels to the edge.
%  and was able to
% fix the graphs with eleven new molecule labels in 180s.
%
The third query synthesizes NNF Boolean functions only containing
$\land$ and $\lor$ gates for activity functions, while allowing
more edges to be added.
%
The result shows the basic template of 4 leaves and 3 gates.
%
% The tool was able to synthesize the Boolean functions in 470s.
%
To illustrate the versatility of our tool, the fourth query
synthesizes $3$-CNF functions (encoding not presented).
%  and was able to
% synthesis the results for all graphs in 500s.
%
Finally, we report queries that allows both addition and deletion of edges, and labels
of node and labels. 

%
%Our experiments suggest that the synthesis problems are solvable by modern 
%solvers and the synthesis technology may be useful for biological research.
%

%%% Local Variables:
%%% mode: latex
%%% TeX-master: "main"
%%% End:
